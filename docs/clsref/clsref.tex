\documentclass[11pt]{article}

\usepackage{amsmath} 
\usepackage{amsfonts}
\usepackage[colorlinks=true]{hyperref}
\usepackage{verbatim}
\usepackage{setspace}
\usepackage{graphicx}

%% Autoscaled figures
\newcommand{\incfig}{\centering\includegraphics}
\setkeys{Gin}{width=0.9\linewidth,keepaspectratio}

%% Commonly used macros
\newcommand{\eqr}[1]{Eq.\thinspace(#1)}
\newcommand{\pfrac}[2]{\frac{\partial #1}{\partial #2}}
\newcommand{\pfraca}[1]{\frac{\partial}{\partial #1}}
\newcommand{\pfracb}[2]{\partial #1/\partial #2}
\newcommand{\mvec}[1]{\mathbf{#1}}
\newcommand{\gvec}[1]{\boldsymbol{#1}}
\newcommand{\script}[1]{\mathpzc{#1}}

\title{Lucee Class Reference Manual}
\author{Ammar Hakim}
\date{}

\begin{document}
\maketitle

\section{Classes to read and manipulate input data}

Input data in \textsc{Lucee} is represented as a hierarchical tree of
key-value pairs by the class \verb=Lucee::KeyValTree=. Each node in
the tree can store key-value pairs, where the keys are strings and
values are basic types (\verb=int=, \verb=float=, \verb=double=,
\verb=std::string=, and \verb=std::vectors= of these). In addition,
each node in the tree can have any number of named child
\verb=Lucee::KeyValTree= objects.

\end{document}
