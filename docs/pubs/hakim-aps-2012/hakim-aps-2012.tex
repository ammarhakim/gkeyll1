\documentclass[pdf]{beamer}
\usepackage{biblatex}
\usepackage{amsmath}
\usepackage{amsfonts}

\DeclareMathAlphabet{\mathpzc}{OT1}{pzc}{m}{it}

\newcommand{\eqr}[1]{Eq.\thinspace(#1)}
\newcommand{\pfrac}[2]{\frac{\partial #1}{\partial #2}}
\newcommand{\pfracc}[2]{\frac{\partial^2 #1}{\partial #2^2}}
\newcommand{\pfraca}[1]{\frac{\partial}{\partial #1}}
\newcommand{\pfracb}[2]{\partial #1/\partial #2}
\newcommand{\pfracbb}[2]{\partial^2 #1/\partial #2^2}
\newcommand{\spfrac}[2]{{\partial_{#1}} {#2}}
\newcommand{\mvec}[1]{\mathbf{#1}}
\newcommand{\gvec}[1]{\boldsymbol{#1}}
\newcommand{\script}[1]{\mathpzc{#1}}
\newcommand{\eep}{\mvec{e}_\phi}
\newcommand{\eer}{\mvec{e}_r}
\newcommand{\eez}{\mvec{e}_z}
\newcommand{\iprod}[2]{\langle{#1}\rangle_{#2}}

\newtheorem{thm}{Theorem}
\newtheorem{lem}{Lemma}

\theoremstyle{definition}
\newtheorem{dfn}{Definition}
\newtheorem{proposed}{Proposed Answer}
\DeclareMathOperator{\spn}{span}

%% autoscaled figures
\newcommand{\incfig}{\centering\includegraphics}
\setkeys{Gin}{width=0.7\linewidth,keepaspectratio}

\usepackage{beamerthemesplit}
% \usepackage{palatino} % use palatino as the default font
\setbeamercovered{transparent}
%\usetheme{Berkeley}
\usetheme{lined}
% \usetheme{Frankfurt}
% \usetheme{Copenhagen}
\usecolortheme{dolphin}

% Setup TikZ
\usepackage{tikz}
\usetikzlibrary{arrows}
\tikzstyle{block}=[draw opacity=0.7,line width=1.4cm]

\title[Continuum Discontinuous Galerkin Algorithms]{High-Order,
  Conservative Discontinuous Galerkin Algorithms for (Gyro) Kinetic
  Simulations of Edge Plasma}%
\author{A. H. Hakim \and G. W. Hammett}%

\institute[http://www.ammar-hakim.org] % (optional, aber oft n�tig)
{
  Princeton Plasma Physics Laboratory, Princeton, NJ\\
  ammar@princeton.edu\\
  \url{http://www.ammar-hakim.org}
}

\date[APS/DPP 2012]{American Physical Society, Division of Plasma
  Physics, 29$^\mathrm{th}$~October- 2$^\mathrm{nd}$~November 2012}

\bibliography{gke}
%\footfullcite{jones00}
\begin{document}

\begin{frame}
  \titlepage
\end{frame}

% ----------------------------------------------------------------
\begin{frame}{Long term goal: Accurate and stable continuum schemes
    for full-F edge gyrokinetics in 3D geometries}

  Question: Can one develop accurate and stable schemes that conserve
  invariants, maintain positivity and use as few grid points as
  possible?

  \begin{proposed}
    Explore high-order hybrid discontinuous/continuous Galerkin
    finite-element schemes, enhanced with flux-reconstruction and a
    proper choice of velocity space basis functions.
  \end{proposed}

\end{frame}
% ----------------------------------------------------------------

% ----------------------------------------------------------------
\begin{frame}{Several fluid and kinetic problems are described by a
    Hamiltonian}%

  \begin{align*}
    \pfrac{f}{t} = \{H,f\}
  \end{align*}
  where $H(z^1,z^2)$ is the Hamiltonian and canonical Poisson bracket
  is
  \begin{align*}
    \{g,h\} \equiv \pfrac{g}{z^1}\pfrac{h}{z^2} -
    \pfrac{g}{z^2}\pfrac{h}{z^1}.
  \end{align*}
  Defining phase-space velocity vector $\gvec{\alpha} = (\dot{z}^1,
  \dot{z}^2)$, with $\dot{z}^i = \{z^i,H\}$ leads to \emph{phase-space
    conservation form}
  \begin{align*}
    \pfrac{f}{t} + \nabla\cdot\left(\gvec{\alpha}f\right) = 0.
  \end{align*}
  Additionally $\nabla\cdot\gvec{\alpha} = 0$ (Liouville theorem).
\end{frame}
% ----------------------------------------------------------------

% ----------------------------------------------------------------
\begin{frame}{Example: Incompressible Euler equations in two
    dimensions}%

  A basic model problem is the \emph{incompressible} 2D Euler
  equations written in the stream-function ($\phi$) vorticity ($f$)
  formulation. Here the Hamiltonian is simply $H(x,y) = \phi(x,y)$.

  \begin{align*}
    \pfrac{f}{t} + \nabla\cdot(\mvec{u}f) = 0
  \end{align*}
  where $u_x = \{x,H\} = \partial\phi/\partial y$ and $u_y = \{y,H\} =
  -\partial\phi/\partial x$. The potential is determined from
  \begin{align*}
    \nabla^2 \phi = -f.
  \end{align*}
\end{frame}
% ----------------------------------------------------------------

% ----------------------------------------------------------------
\begin{frame}{Example: Vlasov equation for electrostatic plasmas}%
  The Vlasov-Poisson system has the Hamiltonian
  \begin{align*}
    H(x,p) = \frac{1}{2m}p^2 + q\phi(x)
  \end{align*}
  where $q$ is species charge and $m$ is electron mass and $p=mv$ is
  momentum. With this $\dot{x} = v$ and $\dot{v} =
  -q\partial\phi/\partial x$ leading to
  \begin{align*}
    \pfrac{f}{t} + v\pfrac{f}{x} -
    \frac{q}{m}\pfrac{\phi}{x}\pfrac{f}{v} = 0
  \end{align*}
\end{frame}
% ----------------------------------------------------------------

% ----------------------------------------------------------------
\begin{frame}{For Vlasov equation potential can be determined in two
    different ways}%

  For electron plasma waves use full Poisson equation
  \begin{align*}
    \frac{\partial^2 \phi}{\partial x^2} = -\frac{\rho_c}{\epsilon_0}
  \end{align*}
  where $\rho_c = |e| (n_{io}(x) - n(x,t))$ is total charge
  density. For ion-acoustic waves use quasi-neutrality
  \begin{align*}
    n_{i}(x) = n_{eo}\left(1 + \frac{|e|\phi}{T_e}\right)
  \end{align*}
  where $n_{eo}$ is the constant electron initial density and $T_e$ is
  the fixed electron temperature. This determines potential without
  the need to solve a Poisson equation and is a model of parallel
  dynamics in gyrokinetics.

\end{frame}
% ----------------------------------------------------------------

% ----------------------------------------------------------------
\begin{frame}{Gyrokinetic equation can also be derived from
    gyro-center Hamiltonian}

  In the Hamiltonian gyrokinetic theory\footfullcite{Brizard:2007fs}
  the gyrokinetic equation is derived from the gyrocentre Hamiltonian
  in gyro-center coordinates $(\mvec{R}, v_{\parallel}, \mu, \alpha)$
  \begin{align*}
    H = \frac{1}{2}m_i v_{\parallel}^2 + \mu B + e_i \langle\phi \rangle_\alpha
  \end{align*}
  where $v_{\parallel}$ is the parallel velocity, $\mu$ is the
  magnetic moment, $\alpha$ is gyro-angle and $\phi$ is the
  electrostatic potential. Poisson bracket is no longer canonical, but
  gyrokinetic Vlasov equation can still be written as a conservation
  equation in phase-space.
\end{frame}
% ----------------------------------------------------------------

% ----------------------------------------------------------------
\begin{frame}{Invariants for Hamiltonian systems can be derived by
    looking at \emph{weak-form} of equations}%

  Multiplying conservation law form by a smooth test function $w(x,v)$
  and integrating over an arbitrary volume element $K$ gives the
  weak-form
  \begin{align*}
    \int_K w\pfrac{f}{t}d\Omega 
    + \int_{\partial K}w \gvec{\alpha}\cdot\mvec{n}f dS
    - \int_K \nabla w \cdot \gvec{\alpha} f d\Omega
    = 0.
  \end{align*}
  Picking $w=1$ leads to (with periodic boundary conditions)
  \emph{particle conservation}
  \begin{align*}
    \frac{d}{dt} \int_K f d\Omega = 0.
  \end{align*}

\end{frame}
% ----------------------------------------------------------------

% ----------------------------------------------------------------
\begin{frame}{Energy conservation is derived using Hamiltonian itself
    as test function}%

  Substituting the Hamiltonian for the test function and using the
  identity $\nabla H \cdot \gvec{\alpha} = 0$ leads to
  \begin{align*}
    \int_K H \pfrac{f}{t}d\Omega = 0.
  \end{align*}
  For the incompressible Euler equation this becomes
  \begin{align*}
    \pfraca{t}\int_K \frac{1}{2} |\nabla\phi|^2  d\Omega = 0.
  \end{align*}
  For the Vlasov-Poisson system this becomes
  \begin{align*}
    \pfraca{t}\int \mathcal{E} +
    \frac{\epsilon_0}{2}\left(\pfrac{\phi}{x}\right)^2 dx = 0
  \end{align*}
  where $\mathcal{E}(x,t) \equiv \frac{1}{2}\int_{-\infty}^{\infty}
  mv^2f dv$ is the fluid energy.

\end{frame}
% ----------------------------------------------------------------

% ----------------------------------------------------------------
\begin{frame}{Generalized entropy (enstrophy) conservation can be
    derived using the solution as test function}%

  The solution itself can be used as a test function. This gives
  \begin{align*}
    \int_K f\pfrac{f}{t}d\Omega 
    + \int_{\partial K}f \gvec{\alpha}\cdot\mvec{n}f dS
    - \int_K \nabla f \cdot \gvec{\alpha} f d\Omega
    = 0.
  \end{align*}
  As $\nabla f \cdot \gvec{\alpha} f = \nabla\cdot (\gvec{\alpha}
  f^2/2)$ the last term reduces to a surface integral, leading to
  \begin{align*}
    \pfraca{t}\int_K \frac{1}{2}f^2 d\Omega = 0.
  \end{align*}

\end{frame}
% ----------------------------------------------------------------

% ----------------------------------------------------------------
\begin{frame}{Valsov-Poisson system also admits momentum
    conservation}%
  
  For the Vlasov-Poisson system we can select the coordinate $v$ as
  the test function. This leads to
  \begin{align*}
    \int_K v\pfrac{f}{t}d\Omega 
    + \int_{\partial K}v \gvec{\alpha}\cdot\mvec{n}f dS
    - \int_K \nabla v \cdot \gvec{\alpha} f d\Omega
    = 0.
  \end{align*}
  As $\nabla v \cdot \gvec{\alpha} = \{v,H\} = \dot{v}f$ the last term
  becomes
  \begin{align*}
    \int_K\dot{v} f d\Omega = \int \frac{|e|}{m} \pfrac{\phi}{x} n\thinspace dx.
  \end{align*}
  Using the Poisson equation to eliminate $n(x,t)$, integrating by
  parts and applying boundary condition leads to the momentum
  conservation law
  \begin{align*}
    \frac{d}{dt}\int_K vf d\Omega = 0.
  \end{align*} 
\end{frame}
% ----------------------------------------------------------------

% 16 -------------------------------------------------------------
\begin{frame}{A discontinuous finite element scheme is used to
    discretize Hamiltonian equation}

  To discretize the equations introduce a triangulation $K_\nu$ of the
  domain $K$. Pick a finite-dimensional function space
  \begin{align*}
    \mathcal{V}^k_m(K) \equiv \{w: w|_{K_\nu} \in P^k(K_\nu) \cap
    C^m\}
  \end{align*}
  where $P^k(K_\nu)$ is the space of polynomials of order at most $k$
  on the element $K_\nu$. Then the discrete problem is stated as: find
  $f_h\in \mathcal{V}^k_{-1}$ such that for all smooth $w$ we have
  \begin{align*}
    \int_{K_\nu} w \pfrac{f_h}{t}\thinspace d\Omega 
    +
    \int_{\partial K_\nu}w^- \mvec{n}\cdot\gvec{\alpha}_h\hat{f}_h\thinspace dS
    -
    \int_{K_\nu} \nabla w\cdot\gvec{\alpha}_h f_h\thinspace d\Omega = 0.
  \end{align*}
  Here $\hat{f}_h = \hat{f}(f^+_h,f^-_h)$ is the consistent
  \emph{numerical} flux on $\partial K_\nu$.
\end{frame}
% ----------------------------------------------------------------

% 16 -------------------------------------------------------------
\begin{frame}{A continuous finite element scheme is used to discretize
    Poisson equation}

\end{frame}
% ----------------------------------------------------------------

% 16 -------------------------------------------------------------
\begin{frame}{Energy conservation table goes here}

\end{frame}
% ----------------------------------------------------------------

% 16 -------------------------------------------------------------
\begin{frame}{Momentum (non)conservation table goes here}

\end{frame}
% ----------------------------------------------------------------

% 16 -------------------------------------------------------------
\begin{frame}{Incompressible Euler example: Merging of two vortices (``vortex waltz'')}

  \begin{figure}
    \setkeys{Gin}{width=0.5\linewidth,keepaspectratio}
    \incfig{two-vortices-128x128_00000.png}
    \incfig{two-vortices-128x128_00003.png}
    \caption{Drop in energy is $1.82\times 10^{-4}\ \%$. This is
      greater than machine precision due to diffusion added from TVD
      Runge-Kutta time-stepping.}
  \end{figure}

\end{frame}
% 16 -------------------------------------------------------------

% 16 -------------------------------------------------------------
\begin{frame}{Incompressible Euler example: Merging of two vortices (``vortex waltz'')}

  \begin{figure}
    \setkeys{Gin}{width=0.5\linewidth,keepaspectratio}
    \incfig{two-vortices-128x128_00006.png}
    \incfig{two-vortices-128x128_00010.png}
    \caption{Drop in energy is $1.82\times 10^{-4}\ \%$. This is
      greater than machine precision due to diffusion added from TVD
      Runge-Kutta time-stepping.}
  \end{figure}

\end{frame}
% 16 -------------------------------------------------------------

% 16 -------------------------------------------------------------
\begin{frame}{Incompressible Euler example: Double shear problem}

  \begin{figure}
    \setkeys{Gin}{width=0.5\linewidth,keepaspectratio}
    \incfig{double-shear_00000.png}
    \incfig{double-shear_00003.png}
    \caption{This problem is harder than ``vortex waltz'' as aliasing
      errors can cause scheme to go unstable.}
  \end{figure}

\end{frame}
% 16 -------------------------------------------------------------

% 16 -------------------------------------------------------------
\begin{frame}{Incompressible Euler example: Double shear problem}

  \begin{figure}
    \setkeys{Gin}{width=0.5\linewidth,keepaspectratio}
    \incfig{double-shear_00006.png}
    \incfig{double-shear_00010.png}
    \caption{This problem is harder than ``vortex waltz'' as aliasing
      errors can cause scheme to go unstable.}
  \end{figure}

\end{frame}
% 16 -------------------------------------------------------------

% 16 -------------------------------------------------------------
\begin{frame}{Vlasov-Poisson example: Linear Landau Damping}
\end{frame}
% 16 -------------------------------------------------------------

% 16 -------------------------------------------------------------
\begin{frame}{Vlasov-Poisson example: Nonlinear Landau Damping}
\end{frame}
% 16 -------------------------------------------------------------

\end{document}

% ----------------------------------------------------------------
\begin{frame}{}%

\end{frame}
% ----------------------------------------------------------------
