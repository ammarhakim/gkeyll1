\documentclass[pdf]{beamer}
\usepackage{biblatex}
\usepackage{amsmath}
\usepackage{amsfonts}

\DeclareMathAlphabet{\mathpzc}{OT1}{pzc}{m}{it}

\newcommand{\eqr}[1]{Eq.\thinspace(#1)}
\newcommand{\pfrac}[2]{\frac{\partial #1}{\partial #2}}
\newcommand{\pfracc}[2]{\frac{\partial^2 #1}{\partial #2^2}}
\newcommand{\pfraca}[1]{\frac{\partial}{\partial #1}}
\newcommand{\pfracb}[2]{\partial #1/\partial #2}
\newcommand{\pfracbb}[2]{\partial^2 #1/\partial #2^2}
\newcommand{\spfrac}[2]{{\partial_{#1}} {#2}}
\newcommand{\mvec}[1]{\mathbf{#1}}
\newcommand{\gvec}[1]{\boldsymbol{#1}}
\newcommand{\script}[1]{\mathpzc{#1}}
\newcommand{\eep}{\mvec{e}_\phi}
\newcommand{\eer}{\mvec{e}_r}
\newcommand{\eez}{\mvec{e}_z}
\newcommand{\iprod}[2]{\langle{#1}\rangle_{#2}}

\newtheorem{thm}{Theorem}
\newtheorem{lem}{Lemma}

\theoremstyle{definition}
\newtheorem{dfn}{Definition}
%\newtheorem{thm}{Theorem}
\newtheorem{proposed}{Proposed Answer}
\DeclareMathOperator{\spn}{span}

%% autoscaled figures
\newcommand{\incfig}{\centering\includegraphics}
\setkeys{Gin}{width=0.7\linewidth,keepaspectratio}

\usepackage{beamerthemesplit}
% \usepackage{palatino} % use palatino as the default font
\setbeamercovered{transparent}
%\usetheme{Berkeley}
\usetheme{lined}
% \usetheme{Frankfurt}
% \usetheme{Copenhagen}
\usecolortheme{dolphin}

% Setup TikZ
\usepackage{tikz}
\usetikzlibrary{arrows}
\tikzstyle{block}=[draw opacity=0.7,line width=1.4cm]

\title[Continuum Discontinuous Galerkin Algorithms]{High-Order,
  Conservative Discontinuous Galerkin Algorithms for (Gyro) Kinetic
  Simulations of Edge Plasma}%
\author{A. H. Hakim \and G. W. Hammett}%

\institute[http://www.ammar-hakim.org] % (optional, aber oft n�tig)
{
  Princeton Plasma Physics Laboratory, Princeton, NJ\\
  ammar@princeton.edu\\
  \url{http://www.ammar-hakim.org}
}

\date[APS/DPP 2012]{American Physical Society, Division of Plasma
  Physics, 29$^\mathrm{th}$~October- 2$^\mathrm{nd}$~November 2012}

\bibliography{gke}
%\footfullcite{jones00}
\begin{document}

\begin{frame}
  \titlepage
\end{frame}

% ----------------------------------------------------------------
\begin{frame}{Long term goal: Accurate and stable continuum schemes
    for full-F edge gyrokinetics in 3D geometries}

  Question: Can one develop accurate and stable schemes that conserve
  invariants, maintain positivity and use as few grid points as
  possible?

  \begin{proposed}
    Explore high-order hybrid discontinuous/continuous Galerkin
    finite-element schemes, enhanced with flux-reconstruction and a
    proper choice of velocity space basis functions.
  \end{proposed}

\end{frame}
% ----------------------------------------------------------------

% ----------------------------------------------------------------
\begin{frame}{Several fluid and kinetic problems are described by a
    Hamiltonian}%

  \begin{align*}
    \pfrac{f}{t} = \{H,f\}
  \end{align*}
  where $H(z^1,z^2)$ is the Hamiltonian and canonical Poisson bracket
  is
  \begin{align*}
    \{g,h\} \equiv \pfrac{g}{z^1}\pfrac{h}{z^2} -
    \pfrac{g}{z^2}\pfrac{h}{z^1}.
  \end{align*}
  Defining phase-space velocity vector $\gvec{\alpha} = (\dot{z}^1,
  \dot{z}^2)$, with $\dot{z}^i = \{z^i,H\}$ leads to \emph{phase-space
    conservation form}
  \begin{align*}
    \pfrac{f}{t} + \nabla\cdot\left(\gvec{\alpha}f\right) = 0.
  \end{align*}
  Additionally $\nabla\cdot\gvec{\alpha} = 0$ (Liouville theorem).
\end{frame}
% ----------------------------------------------------------------

% ----------------------------------------------------------------
\begin{frame}{Example: Incompressible Euler equations in two
    dimensions serves as a model for $E\times B$ nonlinearities in
    gyrokinetics}%

  A basic model problem is the \emph{incompressible} 2D Euler
  equations written in the stream-function ($\phi$) vorticity ($f$)
  formulation. Here the Hamiltonian is simply $H(x,y) = \phi(x,y)$.

  \begin{align*}
    \pfrac{f}{t} + \nabla\cdot(\mvec{u}f) = 0
  \end{align*}
  where $u_x = \{x,H\} = \partial\phi/\partial y$ and $u_y = \{y,H\} =
  -\partial\phi/\partial x$. The potential is determined from
  \begin{align*}
    \nabla^2 \phi = -f.
  \end{align*}
\end{frame}
% ----------------------------------------------------------------

% ----------------------------------------------------------------
\begin{frame}{Example: Vlasov equation for electrostatic plasmas}%
  The Vlasov-Poisson system has the Hamiltonian
  \begin{align*}
    H(x,p) = \frac{1}{2m}p^2 + q\phi(x)
  \end{align*}
  where $q$ is species charge and $m$ is species mass and $p=mv$ is
  momentum. With this $\dot{x} = v$ and $\dot{v} =
  -q\partial\phi/\partial x$ leading to
  \begin{align*}
    \pfrac{f}{t} + v\pfrac{f}{x} -
    \frac{q}{m}\pfrac{\phi}{x}\pfrac{f}{v} = 0
  \end{align*}
\end{frame}
% ----------------------------------------------------------------

% ----------------------------------------------------------------
\begin{frame}{For Vlasov equation potential can be determined in two
    different ways}%

  For electron plasma waves use full Poisson equation
  \begin{align*}
    \frac{\partial^2 \phi}{\partial x^2} = -\frac{\rho_c}{\epsilon_0}
  \end{align*}
  where $\rho_c = |e| (n_{io}(x) - n(x,t))$ is total charge
  density. For ion-acoustic waves use quasi-neutrality
  \begin{align*}
    n_{i}(x) = n_{eo}\left(1 + \frac{|e|\phi}{T_e}\right)
  \end{align*}
  where $n_{eo}$ is the constant electron initial density and $T_e$ is
  the fixed electron temperature. This determines potential without
  the need to solve a Poisson equation and is a model of parallel
  dynamics in gyrokinetics.

\end{frame}
% ----------------------------------------------------------------

% ----------------------------------------------------------------
\begin{frame}{Gyrokinetic equation can also be derived from
    gyro-center Hamiltonian}

  In the Hamiltonian gyrokinetic theory\footfullcite{Brizard:2007fs}
  the gyrokinetic equation is derived from the gyrocentre Hamiltonian
  in gyro-center coordinates $(\mvec{R}, v_{\parallel}, \mu, \alpha)$
  \begin{align*}
    H = \frac{1}{2}m_i v_{\parallel}^2 + \mu B + e_i \langle\phi \rangle_\alpha
  \end{align*}
  where $v_{\parallel}$ is the parallel velocity, $\mu$ is the
  magnetic moment, $\alpha$ is gyro-angle and $\phi$ is the
  electrostatic potential. Poisson bracket is no longer canonical, but
  gyrokinetic Vlasov equation can still be written as a conservation
  equation in phase-space.
\end{frame}
% ----------------------------------------------------------------

% ----------------------------------------------------------------
\begin{frame}{Invariants for Hamiltonian systems can be derived by
    looking at \emph{weak-form} of equations}%

  Multiplying conservation law form by a smooth test function $w(x,v)$
  and integrating over an arbitrary volume element $K$ gives the
  weak-form
  \begin{align*}
    \int_K w\pfrac{f}{t}d\Omega 
    + \int_{\partial K}w^- \gvec{\alpha}\cdot\mvec{n}f dS
    - \int_K \nabla w \cdot \gvec{\alpha} f d\Omega
    = 0.
  \end{align*}
  Picking $w=1$ leads to (with periodic boundary conditions)
  \emph{particle conservation}
  \begin{align*}
    \frac{d}{dt} \int_K f d\Omega = 0.
  \end{align*}

\end{frame}
% ----------------------------------------------------------------

% ----------------------------------------------------------------
\begin{frame}{Energy conservation is derived using Hamiltonian itself
    as test function}%

  Substituting the Hamiltonian for the test function and using the
  identity $\nabla H \cdot \gvec{\alpha} = 0$ leads to
  \begin{align*}
    \int_K H \pfrac{f}{t}d\Omega = 0.
  \end{align*}
  For the incompressible Euler equation this becomes
  \begin{align*}
    \pfraca{t}\int_K \frac{1}{2} |\nabla\phi|^2  d\Omega = 0.
  \end{align*}
  For the Vlasov-Poisson system this becomes
  \begin{align*}
    \pfraca{t}\int \mathcal{E} +
    \frac{\epsilon_0}{2}\left(\pfrac{\phi}{x}\right)^2 dx = 0
  \end{align*}
  where $\mathcal{E}(x,t) \equiv \frac{1}{2}\int_{-\infty}^{\infty}
  mv^2f dv$ is the fluid energy.

\end{frame}
% ----------------------------------------------------------------

% ----------------------------------------------------------------
\begin{frame}{Generalized entropy (enstrophy) conservation can be
    derived using the solution as test function}%

  The solution itself can be used as a test function. This gives
  \begin{align*}
    \int_K f\pfrac{f}{t}d\Omega 
    + \int_{\partial K}f^- \gvec{\alpha}\cdot\mvec{n}f dS
    - \int_K \nabla f \cdot \gvec{\alpha} f d\Omega
    = 0.
  \end{align*}
  As $\nabla f \cdot \gvec{\alpha} f = \nabla\cdot (\gvec{\alpha}
  f^2/2)$ the last term reduces to a surface integral, leading to
  \begin{align*}
    \pfraca{t}\int_K \frac{1}{2}f^2 d\Omega = 0.
  \end{align*}

\end{frame}
% ----------------------------------------------------------------

% ----------------------------------------------------------------
\begin{frame}{Valsov-Poisson system also admits momentum
    conservation}%
  
  For the Vlasov-Poisson system we can select the coordinate $v$ as
  the test function. This leads to
  \begin{align*}
    \int_K v\pfrac{f}{t}d\Omega 
    + \int_{\partial K}v \gvec{\alpha}\cdot\mvec{n}f dS
    - \int_K \nabla v \cdot \gvec{\alpha} f d\Omega
    = 0.
  \end{align*}
  As $\nabla v \cdot \gvec{\alpha} = \{v,H\} = \dot{v}f$ the last term
  becomes
  \begin{align*}
    \int_K\dot{v} f d\Omega = \int \frac{|e|}{m} \pfrac{\phi}{x} n\thinspace dx.
  \end{align*}
  Using the Poisson equation to eliminate $n(x,t)$, integrating by
  parts and applying boundary condition leads to the momentum
  conservation law
  \begin{align*}
    \frac{d}{dt}\int_K vf d\Omega = 0.
  \end{align*} 
\end{frame}
% ----------------------------------------------------------------

% 16 -------------------------------------------------------------
\begin{frame}{A discontinuous finite element scheme is used to
    discretize Hamiltonian equation}

  To discretize the equations introduce a triangulation $K_\nu$ of the
  domain $K$. Pick a finite-dimensional function space
  \begin{align*}
    \mathcal{V}^k_m(K) \equiv \{w: w|_{K_\nu} \in P^k(K_\nu) \cap
    C^m\}
  \end{align*}
  where $P^k(K_\nu)$ is the space of polynomials of order at most $k$
  on the element $K_\nu$. Then the discrete problem is stated as: find
  $f_h\in \mathcal{V}^k_{-1}$ such that for all smooth $w$ we have
  \begin{align*}
    \int_{K_\nu} w \pfrac{f_h}{t}\thinspace d\Omega 
    +
    \int_{\partial K_\nu}w^- \mvec{n}\cdot\gvec{\alpha}_h\hat{f}_h\thinspace dS
    -
    \int_{K_\nu} \nabla w\cdot\gvec{\alpha}_h f_h\thinspace d\Omega = 0.
  \end{align*}
  Here $\hat{f}_h = \hat{f}(f^+_h,f^-_h)$ is the consistent
  \emph{numerical} flux on $\partial K_\nu$.
\end{frame}
% ----------------------------------------------------------------

% 16 -------------------------------------------------------------
\begin{frame}{A continuous finite element scheme is used to discretize
    Poisson equation}

  To discretize the Poisson equation the problem is stated as: find
  $\phi_h \in \mathcal{V}^r_0$ such that for all smooth $\psi$ we have
  \begin{align*}
    \int_K \psi \nabla^2 \phi_h d\Omega = \int_K \psi s d\Omega
  \end{align*}
  where $s$ represents the sources. For ion-acoustic waves the number
  density and potential are related by a \emph{projection} operator:
  find $\phi_h \in \mathcal{V}^k_0$ given a $n_{ih} \in
  \mathcal{V}_{-1}^k$ such that for all smooth $w$
  \begin{align*}
    \int w n_{ih}\thinspace dx = 
    n_{eo}\int w \left(1 + \frac{|e|\phi_h}{T_e}\right) dx
  \end{align*}
  This leads to a \emph{global} solve for the potential. For
  relaxation to the case in which potential is allowed to be
  \emph{discontinuous} and hence a local determination of the
  potential, see poster by G. Hammett.
\end{frame}
% ----------------------------------------------------------------

% ----------------------------------------------------------------
\begin{frame}{Only recently conditions for conservation of discrete
    energy were discovered}%

  Liu and Shu\footfullcite{liu-shu-2000} have shown that discrete
  energy is conserved for 2D incompressible flow if
  \begin{align*}
    \phi_h \in \mathcal{V}_1^k \subseteq f_h \in \mathcal{V}^k_{-1}
  \end{align*}
  We can generalize this to systems describe by a Hamiltonian as
  \begin{thm}
    If the basis functions for the Hamiltonian are a continuous subset
    of the basis functions used for the distribution, then the
    discrete energy will be conserved.
  \end{thm}
  This means that characteristics are discontinuous: $\alpha^i_h \in
  \mathcal{V}_{-1}^{k-1}$.
\end{frame}
% ----------------------------------------------------------------

% ----------------------------------------------------------------
\begin{frame}{Momentum is \emph{not} conserved with this scheme or
    simple variations of it}%

  For electrostatic problems the condition for conservation of
  discrete momentum reduces to a vanishing average force, i.e. we must
  have
  \begin{align*}
    \int n_h E_h dx = 0
  \end{align*}
  However, one can show that as $E_h$ is discontinuous, the present
  scheme \emph{does not} satisfy this condition, and hence momentum is
  not conserved.

  One can imagine that projecting $E_h \in \mathcal{V}_{-1}^{k-1}$ to
  a smoother space $\mathcal{V}_0^{k-1}$ to make it continuous would
  help. However, even with a projection momentum is not
  conserved. Solving the Poisson equation with higher order continuity
  ($\phi_h \in \mathcal{V}_1^r$) also does not help as then the energy
  conservation condition is violated.
\end{frame}
% ----------------------------------------------------------------

% ----------------------------------------------------------------
\begin{frame}{Prototype code named Gkeyll has been developed}%

  \begin{itemize}
  \item Gkeyll is written in C++ and is inspired by framework efforts
    like Facets, VORPAL (Tech-X Corporation) and WarpX
    (U. Washington). Uses structured grids with arbitrary
    dimension/order nodal basis functions.
  \item Linear solvers from
    Petsc\footnote{http://www.mcs.anl.gov/petsc/} are used for
    inverting stiffness matrices.
  \item Games programming language Lua\footnote{http://www.lua.org},
    used in games like World of Warcraft (10 million users), is used
    as an embedded scripting language to drive simulations.
  \item MPI is used for parallelization via the {\tt txbase} library
    developed at Tech-X Corporation.
  \item Package management and builds are automated via {\tt scimake}
    and {\tt bilder}, both developed at Tech-X Corporation.
  \end{itemize}
  
\end{frame}
% ----------------------------------------------------------------

% ----------------------------------------------------------------
\begin{frame}{A simulation journal is
    maintained\footnote{http://www.ammar-hakim.org/sj} to enable
    reproducible research}%

  \begin{itemize}
  \item Each algorithm is carefully tested with known
    analytical or numerical results.
  \item Tests are extensively documented and results, notes as
    well as Lua programs are put online.
  \item Allows sharing of results as well as enables
    reproducibility as scripts, figures and notes are all
    available globally via the internet.
  \end{itemize}

\end{frame}
% ----------------------------------------------------------------

\end{document}

% ----------------------------------------------------------------
\begin{frame}{}%

\end{frame}
% ----------------------------------------------------------------



  \begin{columns}
    \begin{column}{0.55\textwidth}
    \end{column}
    \begin{column}{0.5\textwidth}
      \begin{figure}
        \setkeys{Gin}{width=0.9\linewidth,keepaspectratio}
        \incfig{s149-vlasov-fp_distf_00015.png}
        \caption{\small{Distribution function at $t=3$ for flow in a
            potential well. The black lines show contours of constant
            particle energy. A separatrix forms along the
            trapped-passing boundary. Simulation run with a DG2 scheme
            on a 64�128 grid.}}
      \end{figure}
    \end{column}
  \end{columns}


