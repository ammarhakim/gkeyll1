\documentclass[pdf]{beamer}
\usepackage{biblatex}
\usepackage{amsmath}
\usepackage{amsfonts}

\DeclareMathAlphabet{\mathpzc}{OT1}{pzc}{m}{it}

\newcommand{\eqr}[1]{Eq.\thinspace(#1)}
\newcommand{\pfrac}[2]{\frac{\partial #1}{\partial #2}}
\newcommand{\pfracc}[2]{\frac{\partial^2 #1}{\partial #2^2}}
\newcommand{\pfraca}[1]{\frac{\partial}{\partial #1}}
\newcommand{\pfracb}[2]{\partial #1/\partial #2}
\newcommand{\pfracbb}[2]{\partial^2 #1/\partial #2^2}
\newcommand{\spfrac}[2]{{\partial_{#1}} {#2}}
\newcommand{\mvec}[1]{\mathbf{#1}}
\newcommand{\gvec}[1]{\boldsymbol{#1}}
\newcommand{\script}[1]{\mathpzc{#1}}
\newcommand{\eep}{\mvec{e}_\phi}
\newcommand{\eer}{\mvec{e}_r}
\newcommand{\eez}{\mvec{e}_z}
\newcommand{\iprod}[2]{\langle{#1}\rangle_{#2}}

\newtheorem{thm}{Theorem}
\newtheorem{lem}{Lemma}

\theoremstyle{definition}
\newtheorem{dfn}{Definition}
\newtheorem{proposed}{Proposed Answer}
\DeclareMathOperator{\spn}{span}

%% autoscaled figures
\newcommand{\incfig}{\centering\includegraphics}
\setkeys{Gin}{width=0.7\linewidth,keepaspectratio}

\usepackage{beamerthemesplit}
% \usepackage{palatino} % use palatino as the default font
\setbeamercovered{transparent}
%\usetheme{Berkeley}
\usetheme{lined}
% \usetheme{Frankfurt}
% \usetheme{Copenhagen}
\usecolortheme{dolphin}

% Setup TikZ
\usepackage{tikz}
\usetikzlibrary{arrows}
\tikzstyle{block}=[draw opacity=0.7,line width=1.4cm]

\title[Continuum Discontinuous Galerkin Algorithms]{High-Order,
  Conservative Discontinuous Galerkin Algorithms for (Gyro) Kinetic
  Simulations of Edge Plasma}%
\author{A. H. Hakim \and G. W. Hammett}%

\institute[http://www.ammar-hakim.org] % (optional, aber oft n�tig)
{
  Princeton Plasma Physics Laboratory, Princeton, NJ\\
  ammar@princeton.edu\\
  \url{http://www.ammar-hakim.org}
}

\date[APS/DPP 2012]{American Physical Society, Division of Plasma
  Physics, 29$^\mathrm{th}$~October- 2$^\mathrm{nd}$~November 2012}

\bibliography{gke}
%\footfullcite{jones00}
\begin{document}

\begin{frame}
  \titlepage
\end{frame}

% ----------------------------------------------------------------
\begin{frame}{Long term goal: Accurate and stable continuum schemes
    for full-F edge gyrokinetics in 3D geometries}

  Question: Can one develop accurate and stable schemes that conserve
  invariants, maintain positivity and use as few grid points as
  possible?

  \begin{proposed}
    Explore high-order hybrid discontinuous/continuous Galerkin
    finite-element schemes, enhanced with flux-reconstruction and a
    proper choice of velocity space basis functions.
  \end{proposed}

\end{frame}
% ----------------------------------------------------------------

% ----------------------------------------------------------------
\begin{frame}{Several fluid and kinetic problems are described by a
    Hamiltonian}%

  \begin{align*}
    \pfrac{f}{t} = \{H,f\}
  \end{align*}
  where $H(z^1,z^2)$ is the Hamiltonian and canonical Poisson bracket
  is
  \begin{align*}
    \{g,h\} \equiv \pfrac{g}{z^1}\pfrac{h}{z^2} -
    \pfrac{g}{z^2}\pfrac{h}{z^1}.
  \end{align*}
  Defining phase-space velocity vector $\gvec{\alpha} = (\dot{z}^1,
  \dot{z}^2)$, with $\dot{z}^i = \{z^i,H\}$ leads to \emph{phase-space
    conservation form}
  \begin{align*}
    \pfrac{f}{t} + \nabla\cdot\left(\gvec{\alpha}f\right) = 0.
  \end{align*}
  Additionally $\nabla\cdot\gvec{\alpha} = 0$ (Liouville theorem).
\end{frame}
% ----------------------------------------------------------------

% ----------------------------------------------------------------
\begin{frame}{Example: Incompressible Euler equations in two
    dimensions}%

  A basic model problem is the \emph{incompressible} 2D Euler
  equations written in the stream-function ($\phi$) vorticity ($f$)
  formulation. Here the Hamiltonian is simply $H(x,y) = \phi(x,y)$.

  \begin{align*}
    \pfrac{f}{t} + \nabla\cdot(\mvec{u}f) = 0
  \end{align*}
  where $u_x = \{x,H\} = \partial\phi/\partial y$ and $u_y = \{y,H\} =
  -\partial\phi/\partial x$. The potential is determined from
  \begin{align*}
    \nabla^2 \phi = -f.
  \end{align*}
\end{frame}
% ----------------------------------------------------------------

% ----------------------------------------------------------------
\begin{frame}{Example: Vlasov equation for electrostatic plasmas}%
  The Vlasov-Poisson system has the Hamiltonian
  \begin{align*}
    H(x,p) = \frac{1}{2m}p^2 + q\phi(x)
  \end{align*}
  where $e$ is species charge and $m$ is electron mass and $p=mv$ is
  momentum. With this $\dot{x} = v$ and $\dot{v} =
  -q\partial\phi/\partial x$ leading to
  \begin{align*}
    \pfrac{f}{t} + v\pfrac{f}{x} -
    \frac{q}{m}\pfrac{\phi}{x}\pfrac{f}{v} = 0
  \end{align*}
\end{frame}
% ----------------------------------------------------------------

% ----------------------------------------------------------------
\begin{frame}{For Vlasov equation potential can be determined in two
    different ways}%

  Ensure quasi-neutrality using either
  \begin{align*}
    \frac{\partial^2 \phi}{\partial x^2} = -\frac{\rho_c}{\epsilon_0}
  \end{align*}
  where $\rho_c = |e| (n_{io}(x) - n(x,t))$ is total charge
  density. This is a model for \emph{perpendicular dynamics} in
  gyrokinetics. For \emph{parallel dynamics} use
  \begin{align*}
    n_{i}(x) = n_{eo}\left(1 + \frac{|e|\phi}{T_e}\right)
  \end{align*}
  where $n_{eo}$ is the constant electron initial density and $T_e$ is
  the fixed electron temperature. This determines potential without
  the need to solve a Poisson equation.

\end{frame}
% ----------------------------------------------------------------

% ----------------------------------------------------------------
\begin{frame}{Gyrokinetic equation can also be derived from
    gyro-center Hamiltonian}

  In the Hamiltonian gyrokinetic theory\footfullcite{Brizard:2007fs}
  the gyrokinetic equation is derived from the gyrocentre Hamiltonian
  in gyro-center coordinates $(\mvec{R}, v_{\parallel}, \mu, \alpha)$
  \begin{align*}
    H = \frac{1}{2}m_i v_{\parallel}^2 + \mu B + e_i \langle\phi \rangle_\alpha
  \end{align*}
  where $v_{\parallel}$ is the parallel velocity, $\mu$ is the
  magnetic moment, $\alpha$ is gyro-angle and $\phi$ is the
  electrostatic potential. Poisson bracket is no longer canonical, but
  gyrokinetic Vlasov equation can still be written as a conservation
  equation in phase-space.
\end{frame}
% ----------------------------------------------------------------

\end{document}

% ----------------------------------------------------------------
\begin{frame}{}%

\end{frame}
% ----------------------------------------------------------------

 The potential
  is determined from
  \begin{align*}
    \frac{\partial^2 \phi}{\partial x^2} = -\frac{\varrho_c}{\epsilon_0}
  \end{align*}
  and where $\varrho_c$ is total charge density of plasma.