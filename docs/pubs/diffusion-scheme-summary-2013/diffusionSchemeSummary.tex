\documentclass[serif,12pt]{beamer}
\usepackage{amssymb}
\usepackage[backend=bibtex8]{biblatex}

\usepackage{mathtools}
\usepackage{pxfonts}
% Movie includes
\usepackage{movie15}
\usepackage{hyperref}

\mode<presentation>
{ \usetheme{Madrid}
  \usecolortheme{seagull} }
\usepackage{graphicx}

%remove institute from footer
\makeatletter
\setbeamertemplate{footline}
{
  \leavevmode%
  \hbox{%
  \begin{beamercolorbox}[wd=.5\paperwidth,ht=2.25ex,dp=1ex,center]{author in head/foot}%
    \usebeamerfont{author in head/foot}\insertshorttitle%~~\beamer@ifempty{\insertshortinstitute}{}{(\insertshortinstitute)}
  \end{beamercolorbox}%
  \begin{beamercolorbox}[wd=.5\paperwidth,ht=2.25ex,dp=1ex,right]{date in head/foot}%
    \usebeamerfont{date in head/foot}\insertshortdate{}\hspace*{2em}
    \insertframenumber{} / \inserttotalframenumber\hspace*{2ex} 
  \end{beamercolorbox}}%
  \vskip0pt%
}
\makeatother

%remove navigation symbols
\setbeamertemplate{navigation symbols}{}

\title{Summary of DG Diffusion Schemes}

\addbibresource{SeminarTalk.bib}
%\bibliography{SeminarTalk.bib}

\begin{document}

\begin{frame}
  \titlepage
\end{frame}

\begin{frame}{Outline}
  \tableofcontents
  % You might wish to add the option [pausesections]
\end{frame}

\begin{frame}
\frametitle{Introduction}
	\begin{itemize}
		\item Looked at the following equation:
			\begin{align*}
				g = \frac{\partial f}{\partial t}=& \frac{\partial^2 f}{\partial x^2}
			\end{align*}
		\item Explicit update formulas derived for approximating solution using piecewise linear basis functions:
		\begin{align*}
			u_h(x,t) = u_0 + \frac{x-x_j}{\Delta x / 2} u_1
		\end{align*}
		\item Von Neumann analysis to find eigenvalues
	\end{itemize}
\end{frame}

\begin{frame}
\frametitle{Notation}
\begin{align*}
\lbrack u \rbrack = & u^+ - u^-\\
\overline{u} = & \frac{u^+ + u^-}{2}\\
u^{\pm} = & \lim_{\epsilon \to 0^\pm} u(x+\epsilon,t)
\end{align*}
\end{frame}

\section{Antisymmetric LDG}
\begin{frame}
\frametitle{Asymmetric Local DG (AS LDG)}
\begin{align*}
	\frac{\partial w}{\partial x} + f =& 0,\; \frac{\partial g}{\partial x} + w = 0
\end{align*}
\begin{itemize}
	\item Two choices of fluxes:
	\begin{itemize}
		\item $\hat{f} = f^+$, $\hat{w} = w^-$\\
		\begin{center}
		$\frac{\partial}{\partial t}\left(\begin{array}{cc}u_0\\u_1
		\end{array}\right) = \frac{1}{\Delta x^2}\left(
		\begin{array}{cc}
		4T^{-1} -8+4 T & 2T^{-1}+2-4 T \\
		 -12 T^{-1} +6+6 T & -6 T^{-1} -24-6 T
		\end{array}\right)\left(\begin{array}{c}
		u_0 \\
		 u_1 
		\end{array}
		\right)$
		\end{center}
		\item $\hat{f} = f^-$, $\hat{w} = w^+$\\
		\begin{center}
		$\frac{1}{\Delta x^2}\left(
		\begin{array}{cc}
		4T^{-1} -8+4 T & 4T^{-1}-2-2 T \\
		 -6 T^{-1} -6+12 T & -6 T^{-1} -24-6 T
		\end{array}
		\right)$
		\end{center}
	\end{itemize}
\end{itemize}
\end{frame}

\begin{frame}
\frametitle{AS LDG Eigenvalues}
	\begin{itemize}
		\item Both choices of fluxes result in the same eigenvalues for the Von Neumann analysis
		\item Defining $x = k\Delta x$,
		\begin{align*}
			\lambda_1 = & \frac{1}{\Delta x^2}\left(-16 - 2 \cos x - \sqrt{186 + 136 \cos x + 2 \cos 2x}\right)\\
			\lambda_2 = & \frac{1}{\Delta x^2}\left(-16 - 2 \cos x + \sqrt{186 + 136 \cos x + 2 \cos 2x}\right)
		\end{align*}
		\item $k << 1$ limit:
		\begin{align*}
			\lambda_1 = & -\frac{36}{\Delta x^2} + 3 k^2 - \frac{k^4\Delta x^2}{6} + \frac{k^6\Delta x^4}{270} + O[k^7\Delta x^5]\\
			\lambda_2 = & -k^2 + \frac{k^6\Delta x^4}{540} + O[k^7\Delta x^5]
		\end{align*}
	\end{itemize}
\end{frame}

\section{Symmetric LDG}
\begin{frame}
\frametitle{Symmetric Local DG (S LDG)}
	\begin{itemize}
		\item By averaging the results from the two AS LDG schemes, we get a symmetric scheme for LDG\\
		\begin{center}
		$\frac{1}{\Delta x^2}\left(
		\begin{array}{cc}
		4T^{-1} -8+4 T & 3T^{-1}-3 T \\
		 -9 T^{-1} +9 T & -6 T^{-1} -24-6 T
		\end{array}
		\right)$
		\end{center}
		\item Defining $x = k\Delta x$,
		\begin{align*}
			\lambda_1 = & \frac{1}{\Delta x^2}\left(-16 - 2 \cos x - 2 \sqrt{42 + 40 \cos x - \cos 2x}\right)\\
			\lambda_2 = & \frac{1}{\Delta x^2}\left(-16 - 2 \cos x - 2 \sqrt{42 + 40 \cos x - \cos 2x}\right)
		\end{align*}
		\item $k << 1$ limit:
		\begin{align*}
			\lambda_1 = & -\frac{36}{\Delta x^2} + 3 k^2 - \frac{k^4\Delta x^2}{12} + O[k^6\Delta x^4]\\
			\lambda_2 = & -k^2 - \frac{k^4\Delta x^2}{12} +  O[k^6\Delta x^4]
		\end{align*}
	\end{itemize}
\end{frame}

\section{Direct DG}
\begin{frame}
\frametitle{Direct DG (DDG)}
\begin{itemize}
	\item We looked at two versions of Direct DG, the standard asymmetric version with interface corrections (DDG) and a symmetric version (SDDG)
	\item DDG solves the following weak form:
	\begin{align*}
		\int_{I_j} u_t v \mathrm{d}x -\widehat{(u_x)}v\bigr|_{x_{j-\frac{1}{2}}}^{x_{j+\frac{1}{2}}}+\int_{I_j} u_x v_x \mathrm{d}x+ \frac{1}{2}\lbrack u\rbrack(v_x)_{j+\frac{1}{2}}^- + \frac{1}{2}\lbrack u\rbrack(v_x)_{j-\frac{1}{2}}^+= & 0
	\end{align*}
	\item The flux is defined as:
	\begin{align*}
		\widehat{u_x} = & \beta_0 \frac{\lbrack u \rbrack}{\Delta x} + \overline{u_x} + \beta_1 \Delta x \lbrack u_{xx} \rbrack
	\end{align*}
	\item For $k=1$, took $\beta_0 = 2$ and $\beta_1 = 0$
\end{itemize}
\end{frame}

\begin{frame}
\frametitle{DDG Scheme}
\begin{itemize}
	\item Obtained the following update formulas:\\
	\begin{center}
		$\frac{1}{\Delta x^2}\left(
		\begin{array}{cc}
		2T^{-1} -4+2 T & T^{-1}- T \\
		 -3 T^{-1} +3 T & -12
		\end{array}
		\right)$
	\end{center}
	\item Defining $x = k\Delta x$,
		\begin{align*}
			\lambda_1 = & \frac{1}{\Delta x^2}\left(-8 + 2 \cos x - 2 \sqrt{6 + 4 \cos x - \cos 2x}\right)\\
			\lambda_2 = & \frac{1}{\Delta x^2}\left(-8 + 2 \cos x + 2 \sqrt{6 + 4 \cos x - \cos 2x}\right)
		\end{align*}
		\item $k << 1$ limit:
		\begin{align*}
			\lambda_1 = & -\frac{12}{\Delta x^2} - k^2 + \frac{k^4\Delta x^2}{4} + O[k^6\Delta x^4]\\
			\lambda_2 = & -k^2 - \frac{k^4\Delta x^2}{12} + \frac{k^6 \Delta x^4}{40} + O[k^7\Delta x^5]
		\end{align*}
\end{itemize}
\end{frame}

\section{Symmetric Direct DG}
\begin{frame}
\frametitle{Symmetric DDG (SDDG)}
\begin{itemize}
	\item SDDG solves the following weak form:
	\begin{align*}
		\int_{I_j} u_t v \mathrm{d}x -\widehat{(u_x)}v\bigr|_{x_{j-\frac{1}{2}}}^{x_{j+\frac{1}{2}}}+\int_{I_j} u_x v_x \mathrm{d}x+ (\lbrack u\rbrack\widehat{v_x})_{j+\frac{1}{2}}^- + (\lbrack u\rbrack \widehat{v_x} )_{j-\frac{1}{2}} = & 0
	\end{align*}
	\item The flux is defined as:
	\begin{align*}
		\widehat{u_x} = & \beta_0 \frac{\lbrack u \rbrack}{\Delta x} + \overline{u_x} + \beta_1 \Delta x \lbrack u_{xx} \rbrack\\
		\widehat{v_x} = & \beta_0 \frac{\lbrack v \rbrack}{\Delta x} + \overline{v_x} + \beta_1 \Delta x \lbrack v_{xx} \rbrack
	\end{align*}
	\item $v$ is nonzero only inside the cell $I_j$, so only half of the terms contribute to the computation of $\widehat{v_x}$
\end{itemize}
\end{frame}

\begin{frame}
\frametitle{Symmetric DDG Results}
	\begin{itemize}
		\item Used same values of $\beta_0=2$ and $\beta_1=0$ from DDG
		\item Get same update equations for piecewise linear case as the symmetric LDG method\\
		\begin{center}
		$\frac{1}{\Delta x^2}\left(
		\begin{array}{cc}
		4T^{-1} -8+4 T & 3T^{-1}-3 T \\
		 -9 T^{-1} +9 T & -6 T^{-1} -24-6 T
		\end{array}
		\right)$
		\end{center}
		\item The two methods might be different when higher-order basis functions are used ($k>1$)
	\end{itemize}
\end{frame}

\section{LDG and DDG Schemes Compared}
\begin{frame}
\frametitle{Comparison of LDG and DDG Schemes}
	\begin{figure}
   		\includegraphics[width=0.85\textwidth]{compareDiffusionSchemes.pdf}
	\end{figure}
\end{frame}

\end{document}

