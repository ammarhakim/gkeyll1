\documentclass{siamltex}
\usepackage{amsmath}
\usepackage{graphicx}% Include figure files

%% Autoscaled figures
\newcommand{\incfig}{\centering\includegraphics}
\setkeys{Gin}{width=0.9\linewidth,keepaspectratio}

%% Commonly used macros
\newcommand{\eqr}[1]{Eq.\thinspace(#1)}
\newcommand{\pfrac}[2]{\frac{\partial #1}{\partial #2}}
\newcommand{\pfracc}[2]{\frac{\partial^2 #1}{\partial #2^2}}
\newcommand{\pfraca}[1]{\frac{\partial}{\partial #1}}
\newcommand{\pfracb}[2]{\partial #1/\partial #2}
\newcommand{\mvec}[1]{\mathbf{#1}}
\newcommand{\gvec}[1]{\boldsymbol{#1}}
\newcommand{\script}[1]{\mathpzc{#1}}

\title{Consistency of discontinuous Galerkin discretizations of
  diffusion operators}

\author{A.~H. Hakim\footnotemark[2] \and G.~W. Hammett\footnotemark[2]
  \and E.~Shi\footnotemark[2]\ \footnotemark[3]}

\footnotetext[2]{Princeton Plasma Physics Laboratory, Princeton, NJ
  08543-0451.}%
\footnotetext[3]{Program in Plasma Physics, Department of Astrophysics,
  Princeton University, NJ}

\begin{document}

\maketitle

\begin{abstract}
  Commony used DG schemes for diffusion are inconsistent.
  % Here we compare two classes of discontinuous Galerkin algorithms for
  % diffusion problems, the Local Discontinuous Galerkin (Local DG)
  % approach and the Recover Discontinuous Galerkin approach. The Local
  % DG approach has been widely used, but there are asymmetries in its
  % solution which do not seem to have been noted before, and it appears
  % to be very sensitive to small changes in the level of discontinuity
  % in the solution at the cell interfaces. The Recover DG algorithm is
  % based on the idea of evaluating diffusive fluxes at cell interfaces
  % by recovering a smooth solution that interpolates between two
  % adjacent cells. This approach appears to work well for the tests
  % done here. For hyperdiffusion, a Recovery DG algorithm is more
  % efficient because it has a narrower stencil than Local DG. These
  % comparisons also indicate the importance of projecting the initial
  % conditions onto the proper discrete representation (attempting to
  % force continuity can degrade accuracy). It should be noted that
  % multiple eigenfunctions in DG algorithms have a physical
  % interpretation as representing different wavelength modes, some of
  % which are shorter than the usual Nyquist limit wavelength of 2
  % cells.
\end{abstract}

\begin{keywords} 
  Discontinuous Galerkin, diffusion equation, consistency, interface
  recovery, local discontinuous Galerkin, symmetric discontinuous
  Galerkin.
\end{keywords}

\section{Introduction}

The discontinuous Galerkin (DG) method is a class of high-order
schemes for the numerical solution of partial differential equations
(PDEs). Developed originally for solving the linear, two-dimensional,
steady-state neutron transport equation\cite{reed-hill-1973}, these
schemes were extended to time-dependent, nonlinear, hyperbolic
PDEs\cite{Cockburn:2001vr}, and, subsequently, are now a leading
choice to achieve high-order solutions on complex geometries.

DG schemes are a generalization of finite-volume schemes: in addition
to evolving the cell averages, additional moments of the solution are
also evolved. Explicitly evolving moments eliminates the
``reconstruct'' and ``average'' steps in the standard finite-volume
Reconstruct-Evolve-Average (RAE) procedure\cite{leveque_book_2002} as
a high-order representation is already available in each cell. This
representation is expressed using basis functions local to a cell. In
this sense, the DG representation appears like a Finite-Element (FE)
method. However, unlike FE methods in which some degree of continuity
of the solution across cell boundaries is assumed, no such assumption
is made in DG schemes. In fact, similar to finite-volume schemes, the
jump across cell interfaces is used in an approximate or exact Riemann
solver to couple neighboring cells together and, with appropriate
choices, provide upwinding.

Although the jumps across cell interfaces serve as an important aspect
of DG schemes for hyperbolic equations, the presence of these
discontinuities can be an issue when discretizing second or higher
order operators using the DG framework. In particular, estimates of
the solution derivatives are needed at cell interfaces. As the
solution itself is discontinuous at cell interfaces, it is not clear
how to compute such derivative estimates.

At present there are three broad techniques for computing the solution
derivatives. The first is to introduce auxiliary variables to rewrite
the high-order PDE as a system of first-order PDEs, and use the DG
framework on the resulting larger system. This leads naturally to
estimates of solution derivatives at cell interfaces by particular
choices of numerical fluxes for the introduced auxiliary variables.
The second technique is to derive a weak-form of the PDE and introduce
special numerical fluxes, combined with ``penalty'' terms that
penalize the solution for being discontinuous across interfaces. These
two techniques are not completely independent, and, in some ways,
result from an attempt to apply ideas from FE methods to the DG
discretization. The third technique is to reconstruct a continuous
representation of the solution in the two cells sharing an interface
and use that instead to compute the needed derivatives. This approach
is closer in spirit to finite-volume methods in which solution
gradients at cell interfaces, for example, needed in Navier-Stokes
solvers, are reconstructed using a least-square approach.

\bibliography{gke}
\bibliographystyle{siam}

\end{document}

A particular class of DG schemes for second (and higher) order
operators is obtained by rewriting the original system as a larger
system of first-order equations. In particular, consider the heat
conduction equation
\begin{align}
  \pfrac{f}{t} = \alpha \pfracc{f}{x}
\end{align}
where $f(x,t)$ is a scalar quantity and $\alpha$ is a diffusion
coefficient.
