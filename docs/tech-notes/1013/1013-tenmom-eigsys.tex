\documentclass[11pt, reqno]{amsart}
%% AMS packages and font files
\usepackage{amsmath}
\usepackage{amsfonts}
\usepackage{amsthm}
\usepackage[dvips]{graphicx}
\usepackage[usenames,dvipsnames]{color}
\usepackage{setspace}
\usepackage{fancyhdr}
% \pagestyle{fancyplain}

\DeclareMathAlphabet{\mathpzc}{OT1}{pzc}{m}{it}

%% Set page size properly
% \oddsidemargin  0.0in
% \evensidemargin 0.0in
% \textwidth      6.5in
% \textheight     9.0in
% \leftmargin     1.0in
% \rightmargin    1.0in

%% Autoscaled figures
\newcommand{\incfig}{\centering\includegraphics}
\setkeys{Gin}{width=0.9\linewidth,keepaspectratio}

%% Commonly used macros
\newcommand{\eqr}[1]{Eq.\thinspace(#1)}
\newcommand{\pfrac}[2]{\frac{\partial #1}{\partial #2}}
\newcommand{\pfracc}[2]{\frac{\partial^2 #1}{\partial #2^2}}
\newcommand{\pfraca}[1]{\frac{\partial}{\partial #1}}
\newcommand{\pfracb}[2]{\partial #1/\partial #2}
\newcommand{\pfracbb}[2]{\partial^2 #1/\partial #2^2}
\newcommand{\spfrac}[2]{{\partial_{#1}} {#2}}
\newcommand{\mvec}[1]{\mathbf{#1}}
\newcommand{\gvec}[1]{\boldsymbol{#1}}
\newcommand{\script}[1]{\mathpzc{#1}}

\newtheorem{thm}{Theorem}
\newtheorem{lem}{Lemma}

\theoremstyle{definition}
\newtheorem{dfn}{Definition}

\title[Tenmoment Eigensystem]{The eigensystem of the ten-moment
  equations}%
\author{Ammar H. Hakim}%
\date{}

\begin{document}
% header text
\lhead{Tech-Note 1013}
\maketitle

In this document I list the eigensystem of the ten-moment
equations. These equations are derived by taking moments of the
Boltzmann equation and truncating the resulting infinite series of
equations by assuming the heat flux tensor vanishes. In
non-conservative form these equations are
\begin{align}
  \partial_t{n} + n \partial_j{u_j} + u_j \partial_j{n} &= 0\\
  \partial_t{u_i}
  + \frac{1}{mn}\partial_j{P_{ij}}
  + u_j \partial_j{u_i} &=
  \frac{q}{m}\left(E_i + \epsilon_{kmi}u_kB_m\right) \\
  \partial_i{P_{ij}} + P_{ij}\partial_k{u_k}
  + \partial_k{u_{[i}}P_{j]k}
  + u_k\partial_k{P_{ij}}
  &= \frac{q}{m}B_m \epsilon_{km[i}P_{jk]}
\end{align}
In these equations square brackets around indices represent the
minimal sum over permutations of free indices within the bracket
needed to yield completely symmetric tensors. Note that there is one
such system of equations for \emph{each} species in the plasma. Here,
$q$ is the species charge, $m$ is the species mass, $n$ is the number
density, $u_j$ is the velocity, $P_{ij}$ the pressure tensor and
$\mathbf{E}$ and $\mathbf{B}$ are the electric and magnetic field
respectively. Also $\partial_t \equiv \partial /\partial t$ and
$\partial_i \equiv \partial /\partial x_i$.

For use in numerical schemes the eigensystem of flux Jacobian written
in conservation form is needed. However, it is easier to work with the
primitive form of the equations shown above. The eigensystems of the
quasilinear matrix and flux Jacobian are related and once we determine
one of them the other can be computed by a matrix multiplication.

\end{document}