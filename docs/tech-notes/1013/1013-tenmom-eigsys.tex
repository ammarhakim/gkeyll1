\documentclass[11pt, reqno]{amsart}
%% AMS packages and font files
\usepackage{amsmath}
\usepackage{amsfonts}
\usepackage{amsthm}
\usepackage[dvips]{graphicx}
\usepackage[usenames,dvipsnames]{color}
\usepackage{setspace}
\usepackage{fancyhdr}
% \pagestyle{fancyplain}

\DeclareMathAlphabet{\mathpzc}{OT1}{pzc}{m}{it}

%% Set page size properly
% \oddsidemargin  0.0in
% \evensidemargin 0.0in
% \textwidth      6.5in
% \textheight     9.0in
% \leftmargin     1.0in
% \rightmargin    1.0in

%% Autoscaled figures
\newcommand{\incfig}{\centering\includegraphics}
\setkeys{Gin}{width=0.9\linewidth,keepaspectratio}

%% Commonly used macros
\newcommand{\eqr}[1]{Eq.\thinspace(#1)}
\newcommand{\pfrac}[2]{\frac{\partial #1}{\partial #2}}
\newcommand{\pfracc}[2]{\frac{\partial^2 #1}{\partial #2^2}}
\newcommand{\pfraca}[1]{\frac{\partial}{\partial #1}}
\newcommand{\pfracb}[2]{\partial #1/\partial #2}
\newcommand{\pfracbb}[2]{\partial^2 #1/\partial #2^2}
\newcommand{\spfrac}[2]{{\partial_{#1}} {#2}}
\newcommand{\mvec}[1]{\mathbf{#1}}
\newcommand{\gvec}[1]{\boldsymbol{#1}}
\newcommand{\script}[1]{\mathpzc{#1}}

\newtheorem{thm}{Theorem}
\newtheorem{lem}{Lemma}

\theoremstyle{definition}
\newtheorem{dfn}{Definition}

\title[Tenmoment Eigensystem]{The eigensystem of the ten-moment
  equations}%
\author{Ammar H. Hakim}%
\date{}

\begin{document}
% header text
\lhead{Tech-Note 1013}
\maketitle

\section{The eigensystem of the equations written in quasilinear form}

In this document I list the eigensystem of the ten-moment
equations. These equations are derived by taking moments of the
Boltzmann equation and truncating the resulting infinite series of
equations by assuming the heat flux tensor vanishes. In
non-conservative form these equations are
\begin{align}
  \partial_t{n} + n \partial_j{u_j} + u_j \partial_j{n} &= 0 \label{eq:n} \\
  \partial_t{u_i}
  + \frac{1}{mn}\partial_j{P_{ij}}
  + u_j \partial_j{u_i} &=
  \frac{q}{m}\left(E_i + \epsilon_{kmi}u_kB_m\right) \label{eq:ui} \\
  \partial_t{P_{ij}} + P_{ij}\partial_k{u_k}
  + \partial_k{u_{[i}}P_{j]k}
  + u_k\partial_k{P_{ij}}
  &= \frac{q}{m}B_m \epsilon_{km[i}P_{jk]} \label{eq:Pij}
\end{align}
In these equations square brackets around indices represent the
minimal sum over permutations of free indices within the bracket
needed to yield completely symmetric tensors. Note that there is one
such system of equations for \emph{each} species in the plasma. Here,
$q$ is the species charge, $m$ is the species mass, $n$ is the number
density, $u_j$ is the velocity, $P_{ij}$ the pressure tensor and
$\mathbf{E}$ and $\mathbf{B}$ are the electric and magnetic field
respectively. Also $\partial_t \equiv \partial /\partial t$ and
$\partial_i \equiv \partial /\partial x_i$.

To determine the eigensystem of the homogeneous part of this system we
first write, in one-dimension, the left-hand side of
Eqns.\thinspace(\ref{eq:n})-(\ref{eq:Pij}) in the form
\begin{align}
  \partial_t{\mvec{v}} + \mvec{A}\partial_{1}{\mvec{v}} = 0 \label{eq:qlForm}
\end{align}
where $\mvec{v}$ is the vector of primitive variables and $\mvec{A}$
is the quasilinear coefficient matrix\footnote{There is no standard
  name for this matrix. I choose to call it the \emph{quasilinear
    coefficient matrix} instead of the incorrect term ``primitive flux
  Jacobian''.}. For the ten-moment system we have
\begin{align}
  \mvec{v} = 
    \left[
    \begin{matrix}
      \rho,
      u_1,
      u_2,
      u_3,
      P_{11},
      P_{12},
      P_{13},
      P_{22},
      P_{23},
      P_{33}
    \end{matrix}
  \right]^T
\end{align}
where $\rho \equiv mn$ and 
\begin{align}
  \mvec{A} = 
    \left[
    \begin{matrix}
      u_1  & \rho   & 0      & 0     & 0     & 0     & 0      & 0    & 0    & 0 \\
      0    & u_1    & 0      & 0     & 1/\rho & 0     & 0     & 0    & 0    & 0 \\
      0    & 0      & u_1    & 0     & 0     & 1/\rho & 0     & 0    & 0    & 0 \\
      0    & 0      & 0      & u_1   & 0     & 0     & 1/\rho & 0    & 0    & 0 \\
      0    & 3P_{11} & 0      & 0     & u_1   & 0     & 0      & 0    & 0    & 0 \\
      0    & 2P_{12} & P_{11} & 0     & 0    & u_1    & 0      & 0    & 0    & 0 \\
      0    & 2P_{13} & 0      & P_{11} & 0    & 0      & u_1    & 0    & 0    & 0 \\
      0    & P_{22}  & 2P_{12} & 0     & 0    & 0      & 0     & u_1   & 0    & 0 \\
      0    & P_{23}  & P_{13}  & P_{12} & 0    & 0      & 0     & 0     & u_1  & 0 \\
      0    & P_{33}  & 0      & 2P_{13} & 0   & 0      & 0     & 0     & 0    & u_1
    \end{matrix}
  \right]
\end{align}
The eigensystem of this matrix needs to be determined. It is easiest
to use a computer algebra system for this. I prefer the open source
package Maxima for this. The right-eigenvectors returned by Maxima
need to massaged a little bit to bring them into a clean form. The
results are described below.

The eigenvalues of the system are given by
\begin{align}
  \lambda^{1,2} &= u_1-\sqrt{P_{11}/\rho} \\
  \lambda^{3,4} &= u_1+\sqrt{P_{11}/\rho} \\
  \lambda^{5}   &= u_1-\sqrt{3P_{11}/\rho} \\
  \lambda^{6}   &= u_1+\sqrt{3P_{11}/\rho} \\
  \lambda^{7,8,9,10}    &= u_1
\end{align}
To maintain hyperbolicity we must hence have $\rho>0$ and
$P_{11}>0$. In multiple dimensions, in general, the diagonal elements
of the pressure tensor must be positive. When $P_{11}=0$ the system
reduces to the cold fluid equations which is known to be rank
deficient and hence not hyperbolic as usually understood\footnote{For
  hyperbolicity the quasilinear matrix must posses real eigenvalues
  and a complete set of linearly independent right eigenvectors. For
  the cold fluid system we only have a single eigenvalue (the fluid
  velocity) and a single eigenvector. This can lead to generalized
  solutions like delta shocks.}. Also notice that the eigenvalues do
not include the usual fluid sound-speed $c_s=\sqrt{5p/3\rho}$ but
instead have two different propagation speeds $c_1=\sqrt{P_{11}/\rho}$
and $c_2=\sqrt{3P_{11}/\rho}$. This is because the (neutral)
ten-moment system does not go to the correct limit of Euler equations
in the absence of collisions. In fact, it is collisions that drive the
pressure tensor to isotropy. These collision terms should also be
included in the plasma ten-moment system. In this case, however, the
situation is complicated due to the presence of multiple species of
very different masses which leads to inter-species collision terms
that need to be computed carefully. For a two-species plasma, for
example, see the paper by Green\cite{Green1973} in which the relations
for relaxation of momentum and energy are used to derive a simplified
collision integral for use in the Boltzmann equation.

The right eigenvectors (column vectors) are given below.
\begin{align}
  \mvec{r}^{1,3}
  =
  \left[
    \begin{matrix}
      0 \\
      0 \\
      \mp c_1 \\
      0 \\
      0 \\
      P_{11} \\
      0 \\
      2P_{12} \\
      P_{13} \\
      0
    \end{matrix}
  \right]
  \quad
  \mvec{r}^{2,4}
  =
  \left[
    \begin{matrix}
      0 \\
      0 \\
      0 \\
      \mp c_1 \\
      0 \\
      0 \\
      P_{11} \\
      0 \\
      P_{12} \\
      2P_{13}
    \end{matrix}
  \right]
\end{align}
and
\begin{align}
  \mvec{r}^{5,6}
  =
  \left[
    \begin{matrix}
      \rho P_{11} \\
      \mp c_2 P_{11} \\
      \mp c_2 P_{12} \\
      \mp c_2 P_{13} \\
      3 P_{11}^2 \\
      3 P_{11}P_{12} \\
      3 P_{11}P_{13} \\
      P_{11}P_{22} + 2 P_{12}^2 \\
      P_{11}P_{23} + 2P_{12}P_{13} \\
      P_{11}P_{33} + 2P_{13}^2
    \end{matrix}
  \right]
\end{align}
and
\begin{align}
  \mvec{r}^{7}
  =
  \left[
    \begin{matrix}
      1 \\
      0 \\
      0 \\
      0 \\
      0 \\
      0 \\
      0 \\
      0 \\
      0 \\
      0
    \end{matrix}
  \right]
  \quad
  \mvec{r}^{8}
  =
  \left[
    \begin{matrix}
      0 \\
      0 \\
      0 \\
      0 \\
      0 \\
      0 \\
      0 \\
      1 \\
      0 \\
      0
    \end{matrix}
  \right]
  \quad
  \mvec{r}^{9}
  =
  \left[
    \begin{matrix}
      0 \\
      0 \\
      0 \\
      0 \\
      0 \\
      0 \\
      0 \\
      0 \\
      1 \\
      0
    \end{matrix}
  \right]
  \quad
  \mvec{r}^{10}
  =
  \left[
    \begin{matrix}
      0 \\
      0 \\
      0 \\
      0 \\
      0 \\
      0 \\
      0 \\
      0 \\
      0 \\
      1
    \end{matrix}
  \right]
\end{align}

We can now compute the left eigenvectors (row vectors) by inverting
the matrix with right eigenvectors stored as columns. This ensures the
normalization $\mvec{l}^p \mvec{r}^k = \delta^{pk}$, where the
$\mvec{l}^p$ are the left eigenvectors. On performing the inversion we
have
\begin{align}
  \mvec{l}^{1,3} &= 
  \left[
    \begin{matrix}
      0 & \pm\dfrac{P_{12}}{2c_1P_{11}} & \mp\dfrac{1}{2c_1} & 
      0 & -\dfrac{P_{12}}{2P_{11}^2} & \dfrac{1}{2P_{11}} & 0 & 0 & 0 & 0
    \end{matrix}
  \right] \\
  \mvec{l}^{2,4} &= 
  \left[
    \begin{matrix}
      0 & \pm\dfrac{P_{13}}{2c_1P_{11}} & 0 & \mp\dfrac{1}{2c_1}
      & -\dfrac{P_{13}}{2P_{11}^2} & 0 & \dfrac{1}{2P_{11}} & 0 & 0 & 0
    \end{matrix}
  \right]
\end{align}
and
\begin{align}
  \mvec{l}^{5,6} = 
  \left[
    \begin{matrix}
      0 & \mp\dfrac{1}{2c_2P_{11}} & 0 & 0 & \dfrac{1}{6P_{11}^2}
      & 0 & 0 & 0 & 0 & 0
    \end{matrix}
    \right]
\end{align}
and
\begin{align}
  \mvec{l}^{7} &= 
  \left[
    \begin{matrix}
      1 & 0 & 0 & 0 & -\dfrac{1}{3c_1^2} & 0 & 0 & 0 & 0 & 0
    \end{matrix}
    \right] \\
  \mvec{l}^{8} &= 
  \left[
    \begin{matrix}
      0 & 0 & 0 & 0 & \dfrac{4P_{12}^2-P_{11}P_{22}}{3P_{11}^2} 
      & -\dfrac{2P_{12}}{P_{11}} & 0 & 1 & 0 & 0
    \end{matrix}
    \right] \\
  \mvec{l}^{9} &= 
  \left[
    \begin{matrix}
      0 & 0 & 0 & 0 & \dfrac{4P_{12}P_{13}-P_{11}P_{23}}{3P_{11}^2} 
      & -\dfrac{P_{13}}{P_{11}} & -\dfrac{P_{12}}{P_{11}} & 0 & 1 & 0
    \end{matrix}
    \right] \\
  \mvec{l}^{10} &= 
  \left[
    \begin{matrix}
      0 & 0 & 0 & 0 & \dfrac{4P_{13}^2-P_{11}P_{33}}{3P_{11}^2} & 0
      & -\dfrac{2P_{13}}{P_{11}} & 0 & 0 & 1
    \end{matrix}
    \right]
\end{align}

\section{The eigensystem of the equations written in conservative form}

In the wave-propagation scheme the quasilinear equations can be
updated. However, the resulting solution will not be
conservative. This actually might not be a problem for the ten-moment
system as the system (as written) can not be put into a homogeneous
conservation law form anyway. However, most often for numerical
simulations the eigensystem of the conservation form of the homogeneous
system is needed. This eigensystem is related to the eigensystem of
the quasilinear form derived above. To see this consider a
conservation law
\begin{align}
  \partial_t \mvec{q} + \partial_1 \mvec{f} = 0
\end{align}
where $\mvec{f} = \mvec{f}(\mvec{q})$ is a flux function. Now consider
an invertible transformation $\mvec{q} = \varphi(\mvec{v})$. This
transforms the conservation law to
\begin{align}
  \partial_t \mvec{v} 
  + (\varphi')^{-1}\ D\mvec{f}\ \varphi' \partial_1 \mvec{v} = 0
\end{align}
where $\varphi'$ is the Jacobian matrix of the transformation and $D\mvec{f}
\equiv \partial \mvec{f}/\partial \mvec{q}$ is the flux
Jacobian. Comparing this to \eqr{\ref{eq:qlForm}} we see that the
quasilinear matrix is related to the flux Jacobian by
\begin{align}
  \mvec{A} = (\varphi')^{-1}\ D\mvec{f}\ \varphi'
\end{align}
This clearly shows that the eigenvalues of the flux Jacobian are the
same as those of the quasilinear matrix while the right and left
eigenvectors can be computed using $\varphi' \mvec{r}^p$ and
$\mvec{l}^p(\varphi')^{-1} $ respectively.

For the ten-moment system as written in
Eqns.\thinspace(\ref{eq:n})-(\ref{eq:Pij}) the required transformation
is
\begin{align}
  \mvec{q} = \varphi(\mvec{v})
  =
  \left[
    \begin{matrix}
      \rho \\
      \rho u_1 \\
      \rho u_2 \\
      \rho u_3 \\
      \rho u_1u_1 + P_{11} \\
      \rho u_1u_2 + P_{12} \\
      \rho u_1u_3 + P_{13} \\
      \rho u_2u_2 + P_{22} \\
      \rho u_2u_3 + P_{23} \\
      \rho u_3u_3 + P_{33}
    \end{matrix}
  \right]
\end{align}
For this transformation we have
\begin{align}
  \varphi'(\mvec{v}) = 
    \left[
    \begin{matrix}
      1         & 0          & 0         & 0         & 0 & 0 & 0 & 0 & 0 & 0 \\
      u_1       & \rho       & 0         & 0         & 0 & 0 & 0 & 0 & 0 & 0 \\
      u_2       & 0          & \rho      & 0         & 0 & 0 & 0 & 0 & 0 & 0 \\
      u_3       & 0          & 0         & \rho      & 0 & 0 & 0 & 0 & 0 & 0 \\
      u_1u_1    & 2\rho u_1  & 0         & 0         & 1 & 0 & 0 & 0 & 0 & 0 \\
      u_1u_2    & \rho u_2   & \rho u_1  & 0         & 0 & 1 & 0 & 0 & 0 & 0 \\
      u_1u_3    & \rho u_3   & 0         & \rho u_1  & 0 & 0 & 1 & 0 & 0 & 0 \\
      u_2u_2    & 0          & 2\rho u_2 & 0         & 0 & 0 & 0 & 1 & 0 & 0 \\
      u_2u_3    & 0          & \rho u_3  & \rho u_2  & 0 & 0 & 0 & 0 & 1 & 0\\
      u_3u_3    & 0          & 0         & 2\rho u_3 & 0 & 0 & 0 & 0 & 0 & 1
    \end{matrix}
  \right]
\end{align}
The inverse of the transformation Jacobian is
\begin{align}
  (\varphi')^{-1} = 
    \left[
    \begin{matrix}
      1         & 0          & 0         & 0      & 0 & 0 & 0 & 0 & 0 & 0 \\
      -u_1/\rho & 1/\rho     & 0         & 0      & 0 & 0 & 0 & 0 & 0 & 0 \\
      -u_2/\rho & 0          & 1/\rho    & 0      & 0 & 0 & 0 & 0 & 0 & 0 \\
      -u_3/\rho & 0          & 0         & 1/\rho & 0 & 0 & 0 & 0 & 0 & 0 \\
      u_1u_1    & -2u_1      & 0         & 0      & 1 & 0 & 0 & 0 & 0 & 0 \\
      u_1u_2    & -u_2       & -u_1      & 0      & 0 & 1 & 0 & 0 & 0 & 0 \\
      u_1u_3    & -u_3       & 0         & -u_1   & 0 & 0 & 1 & 0 & 0 & 0 \\
      u_2u_2    & 0          & -2 u_2    & 0      & 0 & 0 & 0 & 1 & 0 & 0 \\
      u_2u_3    & 0          & -u_3      & -u_2   & 0 & 0 & 0 & 0 & 1 & 0\\
      u_3u_3    & 0          & 0         & -2u_3  & 0 & 0 & 0 & 0 & 0 & 1
    \end{matrix}
  \right]
\end{align}

\bibliography{../common/lucee}
\bibliographystyle{plain}

\end{document}
