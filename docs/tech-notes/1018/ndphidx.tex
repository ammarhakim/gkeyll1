\documentclass[11pt, reqno]{amsart}
\usepackage{hyperref}
%% AMS packages and font files
\usepackage{amsmath}
\usepackage{amsfonts}
\usepackage{amsthm}
\usepackage[dvips]{graphicx}
\usepackage[usenames,dvipsnames]{color}
\usepackage{setspace}
\usepackage{fancyhdr}
% \pagestyle{fancyplain}

\DeclareMathAlphabet{\mathpzc}{OT1}{pzc}{m}{it}

%% Set page size properly
% \oddsidemargin  0.0in
% \evensidemargin 0.0in
% \textwidth      6.5in
% \textheight     9.0in
% \leftmargin     1.0in
% \rightmargin    1.0in

%% Set page size properly
\oddsidemargin  0.0in
\evensidemargin 0.0in
\textwidth      6.5in
\textheight     9.0in
\headheight     1.0in
%\headsep        0.0in
\leftmargin      1.0in
\rightmargin      1.0in
\topmargin      -.75in


%% Autoscaled figures
\newcommand{\incfig}{\centering\includegraphics}
\setkeys{Gin}{width=0.9\linewidth,keepaspectratio}

%% Commonly used macros
\newcommand{\eqr}[1]{Eq.\thinspace(#1)}
\newcommand{\pfrac}[2]{\frac{\partial #1}{\partial #2}}
\newcommand{\pfracc}[2]{\frac{\partial^2 #1}{\partial #2^2}}
\newcommand{\pfraca}[1]{\frac{\partial}{\partial #1}}
\newcommand{\pfracb}[2]{\partial #1/\partial #2}
\newcommand{\pfracbb}[2]{\partial^2 #1/\partial #2^2}
\newcommand{\spfrac}[2]{{\partial_{#1}} {#2}}
\newcommand{\mvec}[1]{\mathbf{#1}}
\newcommand{\gvec}[1]{\boldsymbol{#1}}
\newcommand{\script}[1]{\mathpzc{#1}}
\newcommand{\eep}{\mvec{e}_\phi}
\newcommand{\eer}{\mvec{e}_r}
\newcommand{\eez}{\mvec{e}_z}
\newcommand{\iprod}[2]{\langle{#1}\rangle_{#2}}

\newtheorem{thm}{Theorem}
\newtheorem{lem}{Lemma}

\theoremstyle{definition}
\newtheorem{dfn}{Definition}

\title{Does average force vanish in our DG scheme?}%
\author{Ammar H. Hakim}%
\date{}

\begin{document}
% header text
\lhead{Tech-Note 1018}
\maketitle

The Poisson equation is
\begin{align}
  \frac{\partial^2 \phi}{\partial x^2} = n
\end{align}
In the continuous case we have
\begin{align}
  \int n\frac{\partial \phi}{\partial x} dx 
  = 
  \int \frac{\partial^2 \phi}{\partial x^2} \frac{\partial \phi}{\partial x} dx 
  =
  \int \frac{\partial }{\partial x}\left(
  \frac{1}{2} \left(\frac{\partial \phi}{\partial x}\right)^2
  \right) dx
  =
  0
  \label{eq:average-force-id}
\end{align}

For the discrete case we have
\begin{align}
  \int n\frac{\partial \phi}{\partial x} dx 
  =
  \sum_j \int_{I_j} n\frac{\partial \phi}{\partial x} dx 
  + \frac{1}{2}\sum_j (\phi_{j-1/2}^+ - \phi_{j-1/2}^-)(n_{j-1/2}^+ + n_{j-1/2}^-)
\end{align}
The second term comes about from the integration across each cell
interface. However, as $\phi$ is continuous, this term vanishes and we
only need to look at the first term.

The weak-form of the Poisson equation for a single cell $I_j$ is
obtained by multiplying by a test function $w(x)$ to give
\begin{align}
  w(x_{j+1/2}^-)\phi'_{j+1/2} - w(x_{j-1/2}^+)\phi'_{j-1/2} -
  \int_{I_j} w'\phi' dx
  =
  \int_{I_j} wn dx
\end{align}
where primes denote derivatives with respect to $x$. Use $\phi'$ as a
test function\footnote{As $\phi$ can be used as a test function in a
  single cell, so can its derivative as they lie in the same space
  spanned by the finite-element space use to discretize the Poisson
  equation.}. This gives
\begin{align}
  \phi'_{{j+1/2}^-}\phi'_{j+1/2} - \phi'_{{j-1/2}^+}\phi'_{j-1/2} -
  \int_{I_j} \phi''\phi' dx
  =
  \int_{I_j} \phi'n dx
\end{align}
The integral term on the LHS is a perfect derivative which when summed
over all cells will vanish as the edge contributions from shared cells
will cancel. Hence, we have the final expression
\begin{align}
  \sum_j \left(\phi'_{{j-1/2}^+} -
  \phi'_{{j-1/2}^-}\right)\phi'_{j-1/2}
  =
  \sum_j \int_{I_j}n\phi' dx.
\end{align}
This shows that unless the first derivatives of the potential are
continuous, the discrete form of the identity
\eqr{\ref{eq:average-force-id}} is not satisfied. 

This makes sense in hindsight as in proving the identity in the
continuous case there is an implicit assumption that the gradient of
$\phi'$ can be computed, which means that a much higher smoothness is
assumed than is available in the DG scheme.

\end{document}
