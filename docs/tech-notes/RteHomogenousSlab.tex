\documentclass[14]{beamer}

\usepackage{amsmath}
\usepackage{amsfonts}

\newcommand{\eqr}[1]{Eq.\thinspace(#1)}
\newcommand{\pfrac}[2]{\frac{\partial #1}{\partial #2}}
\newcommand{\pfraca}[1]{\frac{\partial}{\partial #1}}
\newcommand{\pfracb}[2]{\partial #1/\partial #2}
\newcommand{\mvec}[1]{\mathbf{#1}}
\newcommand{\gvec}[1]{\boldsymbol{#1}}
\newcommand{\script}[1]{\mathpzc{#1}}

%% autoscaled figures
\newcommand{\incfig}{\centering\includegraphics}
\setkeys{Gin}{width=0.7\linewidth,keepaspectratio}

\usepackage{beamerthemesplit}

% Setup TikZ
\usepackage{tikz}
\usetikzlibrary{arrows}
\tikzstyle{block}=[draw opacity=0.7,line width=1.4cm]

\title[RTE in homogeneous slab]{%
  Algorithm for solution of the radiation transport equation in homogeneous slab}

\author{Ammar H. Hakim}

\date[Lucee]
{Lucee: Algorithm Details}

\begin{document}

\begin{frame}
  \titlepage
\end{frame}

% ----------------------------------------------------------------
\begin{frame}{Problem statement: basic equations}
  \begin{align}
    \mu\pfrac{L(\tau,\mu,\phi)}{\tau} + L(\tau,\mu,\phi)
    =
    \frac{\varpi}{4\pi}
    \int_{-1}^1 \int_0^{2\pi}
    p(\cos\Theta) L(\tau,\mu,\phi) d\mu d\phi
  \end{align}
  where
  \begin{itemize}
  \item $L(\tau,\mu,\phi)$ is the radiance in units of Watt m$^{-2}$
    sr$^{-1}$ nm$^{-1}$,
  \item $\tau$ is the optical depth,
  \item $\varpi$ is the albedo of single scattering,
  \item $\mu$ is the cosine of the polar angle measured with the
    positive $Z$-axis and $\phi$ is the azimuthal angle,
  \item $p(\cos\Theta) = \sum_{l=0}^L\beta_lP_l(\cos\Theta)$ is the
    phase function, where $\Theta$ is the scattering angle and
    $\beta_0=1$.
  \end{itemize}
\end{frame}
% ----------------------------------------------------------------

% ----------------------------------------------------------------
\begin{frame}{Problem statement: boundary conditions}
  \begin{align}
    L(0, \mu) &= \pi F \delta(\mu-\mu_0) \delta(\phi-\phi_0) \\
    L(\tau_0, -\mu) &= 0
  \end{align}
  for $\mu\in [0,1]$, and where
  \begin{itemize}
  \item $\tau_0$ is the optical depth of the slab,
  \item $\mu_0$ and $\phi_0$ are the cosine of the polar angle and
    azimuthal angle of incident beam,
  \item $\mu_0\pi F$ is the total downward irradiance incident on the
    slab.
  \end{itemize}
\end{frame}
% ----------------------------------------------------------------

% ----------------------------------------------------------------
\begin{frame}{The reduced problem}
  Remove the singular component of the radiance using
  \begin{align}
    L(\tau,\mu,\phi) = L_*(\tau,\mu,\phi) 
    + \pi F \delta(\mu-\mu_0) \delta(\phi-\phi_0) e^{-\tau/\mu_0}
  \end{align}
  Use addition theorem of spherical harmonics to write
  \begin{align}
    p(cos\Theta) = \sum_{m=0}^L(2-\delta_{0,m})
    \sum_{l=m}^L\beta_l P_l^m(\mu) P_l^m(\mu')
    \cos[m(\phi'-\phi)]
  \end{align}
  where
  \begin{align}
    P_l^m(\mu) = \left[
      \frac{(l-m)!}{(l+m)!}
      \right]^{1/2}
      (1-\mu^2)^{1/2}\frac{d^m}{d\mu^m}
      P_l(\mu)
  \end{align}
  are the normalized Legendre functions.
\end{frame}
% ----------------------------------------------------------------

% ----------------------------------------------------------------
\begin{frame}{The reduced problem}
  The diffuse field can be written as
  \begin{align}
    L_*(\tau,\mu,\phi) = \frac{1}{2} \sum_{m=0}^L
    (2-\delta_{0,m})L^m(\tau,\mu) cos[m(\phi-\phi_0)]
  \end{align}
  where
  \begin{align}
    \mu\pfrac{L^m(\tau,\mu)}{\tau} + &L^m(\tau,\mu)
    = \\
    &\frac{\varpi}{2}
    \sum_{l=m}^L \beta_l P_l^m(\mu)
    \int_{-1}^1
    P_l^m(\mu') L(\tau,\mu') d\mu'
    + Q^m(\tau,\mu)
  \end{align}
  and 
  \begin{align}
    Q^m(\tau,\mu) = \frac{\varpi}{2}e^{-\tau/\mu_0}
    \sum_{l=m}^L \beta_l P^m_l(\mu_0) P_l^m(\mu)
  \end{align}
\end{frame}
% ----------------------------------------------------------------

% ----------------------------------------------------------------
\begin{frame}{The reduced boundary conditions}
  The boundary conditions now reduce to
  \begin{align}
    L^m(0, \mu) &= 0 \\
    L^m(\tau_0, -\mu) &= 0
  \end{align}
  The aim of the algorithm is to compute the azimulthal components of
  the radiance $L^m(\tau,\mu)$ and the \emph{irradiances} defined by
  \begin{align}
    E_\pm^l(\tau) \equiv \int_0^1 \int_0^{2\pi}
    P_l(\mu) L(\tau,\pm \mu, \phi) d\mu d\phi
  \end{align}
\end{frame}
% ----------------------------------------------------------------

% ----------------------------------------------------------------
\begin{frame}{Initialization of algorithm}
  \begin{itemize}
  \item First, pick a set of $N$ quadrature points in the domain $\mu
    \in[0,1]$. Denote these as $\mu_i$ and the corresponding weights as
    $w_i$, for $i=1,\ldots,N$.
  \item The $N$ quadrature points in the domain $\mu \in[0,-1]$ are
    $-\mu_i$ and the corresponding weights as $w_i$, for
    $i=1,\ldots,N$.
  \item The algorithm solves the radiance at the discrete quadrature
    points in each hemisphere, i.e. $L^m(\tau,\pm \mu_i)$ are
    computed.
  \item Gaussian quadrature is usually used. 
  \item It is may also be valuable to use a quadrature scheme that has
    $\mu = \pm 1$. This directly gives the \emph{remote-sensing
      reflectance} $L(0,-1)/E_+^1(0)$ without any need for
    interpolation.
  \end{itemize}
\end{frame}
% ----------------------------------------------------------------

\end{document}

% ----------------------------------------------------------------
\begin{frame}
\end{frame}
% ----------------------------------------------------------------