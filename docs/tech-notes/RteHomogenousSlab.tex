\documentclass[11pt]{article}
%% AMS packages and font files
\usepackage{amsmath}
\usepackage{amsfonts}
\usepackage[dvips]{graphicx}
\usepackage[usenames,dvipsnames]{color}
\usepackage{setspace}
\usepackage{fancyhdr}
% \pagestyle{fancyplain}

\DeclareMathAlphabet{\mathpzc}{OT1}{pzc}{m}{it}

%% Set page size properly
\oddsidemargin  0.0in
\evensidemargin 0.0in
\textwidth      6.5in
\textheight     9.0in
\headheight     1.0in
\headsep        0.0in
\leftmargin      1.0in
\rightmargin      1.0in
\topmargin      -.75in

%% Autoscaled figures
\newcommand{\incfig}{\centering\includegraphics}
\setkeys{Gin}{width=0.9\linewidth,keepaspectratio}

%% Commonly used macros
\newcommand{\eqr}[1]{Eq.\thinspace(#1)}
\newcommand{\pfrac}[2]{\frac{\partial #1}{\partial #2}}
\newcommand{\pfraca}[1]{\frac{\partial}{\partial #1}}
\newcommand{\pfracb}[2]{\partial #1/\partial #2}
\newcommand{\mvec}[1]{\mathbf{#1}}
\newcommand{\gvec}[1]{\boldsymbol{#1}}
\newcommand{\script}[1]{\mathpzc{#1}}

\title{Algorithms for solution of the radiation transport equation in%
  a slab}
\author{Ammar H. Hakim}
\date{}

\begin{document}
% header text
\lhead{Tech-Note 1002}
\maketitle

\section{Homogeneous slab algorithm}


\subsection{Basic equations}
\begin{align}
  \mu\pfrac{L(\tau,\mu,\phi)}{\tau} + L(\tau,\mu,\phi)
  =
  \frac{\varpi}{4\pi}
  \int_{-1}^1 \int_0^{2\pi}
  p(\cos\Theta) L(\tau,\mu,\phi) d\mu d\phi
\end{align}
where
\begin{itemize}
\item $L(\tau,\mu,\phi)$ is the radiance in units of Watt m$^{-2}$
  sr$^{-1}$ nm$^{-1}$,
\item $\tau$ is the optical depth,
\item $\varpi$ is the albedo of single scattering,
\item $\mu$ is the cosine of the polar angle measured with the
  positive $Z$-axis and $\phi$ is the azimuthal angle,
\item $p(\cos\Theta) = \sum_{l=0}^L\beta_lP_l(\cos\Theta)$ is the
  phase function, where $\Theta$ is the scattering angle and
  $\beta_0=1$.
\end{itemize}




\subsection{Boundary conditions}
\begin{align}
  L(0, \mu) &= \pi F \delta(\mu-\mu_0) \delta(\phi-\phi_0) \\
  L(\tau_0, -\mu) &= 0
\end{align}
for $\mu\in [0,1]$, and where
\begin{itemize}
\item $\tau_0$ is the optical depth of the slab,
\item $\mu_0$ and $\phi_0$ are the cosine of the polar angle and
  azimuthal angle of incident beam,
\item $\mu_0\pi F$ is the total downward irradiance incident on the
  slab.
\end{itemize}




\subsection{The reduced problem}
Remove the singular component of the radiance using
\begin{align}
  L(\tau,\mu,\phi) = L_*(\tau,\mu,\phi) 
  + \pi F \delta(\mu-\mu_0) \delta(\phi-\phi_0) e^{-\tau/\mu_0}
\end{align}
Use addition theorem of spherical harmonics to write
\begin{align}
  p(cos\Theta) = \sum_{m=0}^L(2-\delta_{0,m})
  \sum_{l=m}^L\beta_l P_l^m(\mu) P_l^m(\mu')
  \cos[m(\phi'-\phi)]
\end{align}
where
\begin{align}
  P_l^m(\mu) = \left[
    \frac{(l-m)!}{(l+m)!}
  \right]^{1/2}
  (1-\mu^2)^{1/2}\frac{d^m}{d\mu^m}
  P_l(\mu)
\end{align}
are the normalized Legendre functions.

The diffuse field can be written as
\begin{align}
  L_*(\tau,\mu,\phi) = \frac{1}{2} \sum_{m=0}^L
  (2-\delta_{0,m})L^m(\tau,\mu) \cos[m(\phi-\phi_0)]
\end{align}
where
\begin{align}
  \mu\pfrac{L^m(\tau,\mu)}{\tau} + L^m(\tau,\mu)
  = 
  \frac{\varpi}{2}
  \sum_{l=m}^L \beta_l P_l^m(\mu)
  \int_{-1}^1
  P_l^m(\mu') L(\tau,\mu') d\mu'
  + Q^m(\tau,\mu)
\end{align}
and 
\begin{align}
  Q^m(\tau,\mu) = \frac{\varpi F}{2}e^{-\tau/\mu_0}
  \sum_{l=m}^L \beta_l P^m_l(\mu_0) P_l^m(\mu)
\end{align}




\subsection{The reduced boundary conditions}
The boundary conditions now reduce to
\begin{align}
  L^m(0, \mu) &= 0 \\
  L^m(\tau_0, -\mu) &= 0
\end{align}
The aim of the algorithm is to compute the azimuthal components of
the radiance $L^m(\tau,\mu)$ and the \emph{irradiances} defined by
\begin{align}
  E_\pm^l(\tau) \equiv \int_0^1 \int_0^{2\pi}
  P_l(\mu) L(\tau,\pm \mu, \phi) d\mu d\phi
\end{align}




\subsection{Initialization of algorithm}
\begin{itemize}
\item First, pick a set of $N$ quadrature points in the domain $\mu
  \in[0,1]$. Denote these as $\mu_i$ and the corresponding weights
  as $w_i$, for $i=1,\ldots,N$. Gaussian\footnote{It is may also be
    valuable to use a quadrature scheme that has includes $\pm
    1$. This directly gives the \emph{remote-sensing reflectance}
    $L(0,-1)/E_+^1(0)$ without any need for interpolation.}
  quadrature is used\footnote{See {\tt RteHomogeneousSlab::buildAlgorithms()}}.
\item The $N$ quadrature points in the domain $\mu \in[0,-1]$ are
  $-\mu_i$ and the corresponding weights as $w_i$, for
  $i=1,\ldots,N$.
\item The algorithm solves the radiance at the discrete quadrature
  points in each hemisphere, i.e. $L^m(\tau,\pm \mu_i)$ are
  computed.
\end{itemize}





\subsection{Compute $Q^m(\pm \mu_i)$ for the inhomogeneous source}
Compute $\mvec{Q}_\pm$ where\footnote{See {\tt
    RteHomogeneousSlab::qbeam()}}
\begin{align}
  \mvec{Q}_\pm = [Q^m(\pm\mu_1), \ldots, Q^m(\pm\mu_N)]^T
\end{align}
\begin{align}
  Q^m(\pm \mu_i) = \frac{\varpi F}{2}
  \sum_{l=m}^L \beta_l P^m_l(\mu_0) P_l^m(\pm \mu_i)
\end{align}
Note that $P_l^m(-\mu_i) = (-1)^{l-m}P_l^m(\mu_i)$.




\subsection{Compute the eigensystem of the RTE}
Compute the matrices $\mvec{E}$ and $\mvec{F}$ as
follows\footnote{See {\tt RteHomogeneousSlab::calc\_FE()}}
\begin{align}
  \mvec{E} &= \left(
    1-\frac{\varpi}{2}\sum_{l=m}^L
    \beta_l
    \left[1+(-1)^{l-m}\right] 
    \gvec{\Pi}(l,m) \gvec{\Pi}^T(l,m) \mvec{W}
  \right)
  \mvec{M}^{-1}
\end{align}
\begin{align}
  \mvec{F} &= \left(
    1-\frac{\varpi}{2}\sum_{l=m}^L
    \beta_l
    \left[1-(-1)^{l-m}\right] 
    \gvec{\Pi}(l,m) \gvec{\Pi}^T(l,m) \mvec{W}
  \right)
  \mvec{M}^{-1}
\end{align}
where $\mvec{M} = \textrm{diag}(\mu_1,\ldots,\mu_N)$, $\mvec{W} =
\textrm{diag}(w_i,\ldots,w_N)$ and $\gvec{\Pi}(l,m) =
[P_l^m(\mu_1),\ldots,P_l^m(\mu_N)]^T$.

Then compute the matrix product $\mvec{F}\mvec{E}$ and determine its
eigenvalues and eigenvectors. Denote these by $\lambda_j$ and
$\mvec{X}(\lambda_j)$.  Then, the eigenvalues of the RTE are $\pm
\nu_j$, where $\nu_j = 1/\sqrt\lambda_j$. The eigenvectors
are\footnote{See {\tt RteHomogeneousSlab::calc\_RTE\_eigensystem()}}
\begin{align}
  \gvec{\Phi}_\pm (\nu_j)
  = \frac{1}{2}\mvec{M}^{-1}
  (\mvec{I} \pm \nu_j\mvec{E})\mvec{X}(\lambda_j)
\end{align}
Note the identity $\gvec{\Phi}_+(-\nu_j) =
\gvec{\Phi}_-(\nu_j)$.




\subsection{Compute eigensystem normalization coefficients}
These are given by\footnote{See RteHomogeneousSlab::get\_norms()}
\begin{align}
  N(\nu_j) = \sum_{\alpha=1}^N
  w_\alpha \mu_\alpha
  \left[
    \phi^2(\nu_j,\mu_\alpha) - \phi^2(\nu_j,-\mu_\alpha)
  \right]
\end{align}
Here $\phi(\nu_j,\mu_\alpha) = \gvec{\Phi}_{+\alpha}(\nu_j)$ and
$\phi(\nu_j,-\mu_\alpha) = \gvec{\Phi}_{-\alpha}(\nu_j)$.





\subsection{Homogeneous solution can now be determined up to constant
  coefficients}
Once we have the eigensystem, we can compute the homogeneous solution
\begin{align}
  \mvec{L}^h_+(\tau) = \sum_{j=1}^N
  \left[
    A_j \gvec{\Phi}_+(\nu_j) e^{-\tau/\nu_j} +
    B_j \gvec{\Phi}_-(\nu_j) e^{-(\tau_0-\tau)/\nu_j}
  \right]
\end{align}
and
\begin{align}
  \mvec{L}^h_-(\tau) = \sum_{j=1}^N
  \left[
    A_j \gvec{\Phi}_-(\nu_j) e^{-\tau/\nu_j} +
    B_j \gvec{\Phi}_+(\nu_j) e^{-(\tau_0-\tau)/\nu_j}
  \right]
\end{align}
Here $\mvec{L}_\pm(\tau) = [L(\tau,\pm\mu_1),\ldots,
L(\tau,\pm\mu_N)]^T$. The coefficients $A_j$ and $B_j$ are unknown.




\subsection{Particular solution can now be determined}
The particular solutions can be written as\footnote{See
  RteHomogeneousSlab::particular\_solution()}
\begin{align}
  \mvec{L}^p_+(\tau) = \sum_{j=1}^N
  \left[
    \script{A}_j(\tau) \gvec{\Phi}_+(\nu_j) +
    \script{B}_j(\tau) \gvec{\Phi}_-(\nu_j)
  \right]
\end{align}
and
\begin{align}
  \mvec{L}^p_-(\tau) = \sum_{j=1}^N
  \left[
    \script{A}_j(\tau) \gvec{\Phi}_-(\nu_j) +
    \script{B}_j(\tau) \gvec{\Phi}_+(\nu_j)
  \right]
\end{align}

The functions appearing in the particular solutions can be computed as
\begin{align}
  \script{A}_j(\tau) = &\frac{1}{N(\nu_j)}
  \int_0^\tau dx
  \sum_{\alpha=1}^N w_\alpha e^{-(\tau-x)/\nu_j}
  \left[
    Q(x,\mu_\alpha)\phi(\nu_j,\mu_\alpha) +
    Q(x,-\mu_\alpha)\phi(\nu_j,-\mu_\alpha)
  \right]
\end{align}
and
\begin{align}
  \script{B}_j(\tau) = &\frac{1}{N(\nu_j)}
  \int_\tau^{\tau_0} dx
  \sum_{\alpha=1}^N w_\alpha e^{-(x-\tau)/\nu_j}
  \left[
    Q(x,\mu_\alpha)\phi(\nu_j,-\mu_\alpha) +
    Q(x,-\mu_\alpha)\phi(\nu_j,\mu_\alpha)
  \right]
\end{align}


\subsection{For beam illumination $\script{A}$ and $\script{B}$ can
  be explicitly computed}
For uniform illumination we can compute explicitly\footnote{See
  RteHomogeneousSlab::scriptAB()}
\begin{align}
  \script{A}_j(\tau) = \frac{\mu_0 \nu_j}{N(\nu_j)}
  C(\tau ; \nu_j, \mu_0)
  \left[
    \gvec{\Phi}_+^T \mvec{W} \mvec{Q}_+ +
    \gvec{\Phi}_-^T \mvec{W} \mvec{Q}_-
  \right]
\end{align}
and
\begin{align}
  \script{B}_j(\tau) = \frac{\mu_0 \nu_j}{N(\nu_j)}
  e^{-\tau/\mu_0}S(\tau_0-\tau ; \nu_j, \mu_0)
  \left[
    \gvec{\Phi}_-^T \mvec{W} \mvec{Q}_+ +
    \gvec{\Phi}_+^T \mvec{W} \mvec{Q}_-
  \right]
\end{align}
Here
\begin{align}
  S(\tau;x,y) = \frac{1-e^{-\tau/x}e^{-\tau/y}}{x+y} \quad\textrm{and}\quad
  C(\tau;x,y) = \frac{e^{-\tau/x}-e^{-\tau/y}}{x-y}
\end{align}




\subsection{Use boundary conditions to determine linear system for
  $A_j$ and $B_j$}
Using the boundary conditions we get
\begin{align}
  \sum_{j=1}^N
  \left[
    A_j \gvec{\Phi}_+(\nu_j) +
    B_j \gvec{\Phi}_-(\nu_j) e^{-\tau_0/\nu_j}
  \right]
  &= -\mvec{L}_+^p(0) \\
  \sum_{j=1}^N
  \left[
    A_j \gvec{\Phi}_-(\nu_j) e^{-\tau_0/\nu_j} +
    B_j \gvec{\Phi}_+(\nu_j)
  \right]
  &= -\mvec{L}_-^p(\tau_0)
\end{align}
This is a linear system that can be solved to determine the
coefficients $A_i$ and $B_i$\footnote{See {\tt
    RteHomogeneousSlab::calc\_AB\_coeffs()}}.





\subsection{For half-space problems algorithm needs modification}
For half-space problems $L(\tau,\mu,\phi) \rightarrow 0$ as $\tau
\rightarrow \infty$. Hence, we must set $B_j = 0$. Further, the
functions $\script{B}_j(\tau)$ must now be computed with the
function $S^\infty(x,y) = 1/(x+y)$ instead. Note, there is no
dependence on $\tau$ in $S^\infty$.

Finally, the linear system to solve now reduces to a $N\times N$
system for the $N$ unknowns $A_j$
\begin{align}
  \sum_{j=1}^N
  A_j \gvec{\Phi}_+(\nu_j) = -\mvec{L}_+^p(0)
\end{align}
which can now be inverted to determine $A_j$\footnote{See
  {\tt RteHomogeneousSlab::calc\_A\_coeffsInf()}}




\subsection{Computing irradiances}
The irradiances, defined previously, can be written as
\begin{align}
  E_+^l(\tau) &= \pi \int_0^1
  P_l(\mu) L^0(\tau,\mu) d\mu
  + \pi F e^{-\tau/\mu_0} P_l(\mu_0) \\
  E_-^l(\tau) &= \pi \int_0^1
  P_l(\mu) L^0(\tau,-\mu) d\mu
\end{align}
Note that to compute the irradiances only the $m=0$ solution is
needed. The irradiances can be computed as\footnote{See {\tt
    RteHomogeneousSlab::calc\_irradiances()}}
\begin{align}
  E_+^l(\tau) &= \pi \sum_{k=1}^N
  \left[
    A_k e^{-\tau/\nu_k} + \script{A}_k(\tau)
  \right] R_+^l(\nu_k)
  + \pi \sum_{k=1}^N
  \left[
    B_k e^{-(\tau_0-\tau)/\nu_k} + \script{B}_k(\tau)
  \right] R_-^l(\nu_k)
  + \pi P_l(\mu_0) e^{-\tau/\nu_0} \\
  E_-^l(\tau) &= \pi \sum_{k=1}^N
  \left[
    A_k e^{-\tau/\nu_k} + \script{A}_k(\tau)
  \right] R_-^l(\nu_k)
  + \pi \sum_{k=1}^N
  \left[
    B_k e^{-(\tau_0-\tau)/\nu_k} + \script{B}_k(\tau)
  \right] R_+^l(\nu_k)
\end{align}
where $R_\pm^l(\nu_j) \equiv \mvec{\Pi}(l,0) \mvec{W}
\gvec{\Phi}_\pm(\nu_j)$.




\subsection{Angular interpolation}
Introduce a set of $M$ dummy nodes $\mu_i$, $i=N+1,\ldots,N+M$ with
zero weights. Then, the eigenvectors of the RTE become the extended
length $N+M$ vector $[\gvec{\Phi}_\pm(\nu),
\hat{\gvec{\Phi}}_\pm(\nu)]^T$. We can show that for $i=1,\ldots,N$
and $j=N+1,\ldots,N+M$, we get\footnote{See {\tt
    RteHomogeneousSlab::calc\_extended\_eigensystem}}
\begin{align}
  \hat{\phi}(\nu_i,\mu_j) &= \frac{\nu_i}{\nu_i-\mu_j} \frac{\varpi}{2}
  \sum_{l=m}^L \beta_l P_l^m(\mu_j) \sum_{k=1}^N
  w_k P_l^m(\mu_k)
  \left[
    \phi(\nu_i,\mu_k) + (-1)^{l-m}\phi(\nu_i,-\mu_j)
  \right] \\
  \hat{\phi}(\nu_i,-\mu_j) &= \frac{\nu_i}{\nu_i+\mu_j} \frac{\varpi}{2}
  \sum_{l=m}^L \beta_l P_l^m(\mu_j) \sum_{k=1}^N
  w_k P_l^m(\mu_k)
  \left[
    (-1)^{l-m} \phi(\nu_i,\mu_k) + \phi(\nu_i,-\mu_j)
  \right]
\end{align}
The extra $M$ eigenvalues are simply the dummy nodes themselves,
i.e. $\nu_j = \mu_j$ for $i=N+1,\ldots,N+M$. We can show
\begin{align}
  \phi(\nu_i,\mu_j) &= \phi(\nu_i,-\mu_j) = 0 \\
  \hat{\phi}(\nu_i,\mu_j) &= \hat{\phi}(-\nu_i,-\mu_j) = \delta_{ij}.
\end{align}
Hence, homogeneous solution at dummy nodes is
\begin{align}
  L^h(\tau,\mu_i) &= \sum_{j=1}^N
  \left[
    A_j \hat{\phi}(\nu_j, \mu_i) e^{-\tau/\nu_j} +
    B_j \hat{\phi}(-\nu_j, \mu_i) e^{-(\tau_0-\tau)/\nu_j}
  \right]
  + A_i e^{-\tau/\nu_i} \\
  L^h(\tau,-\mu_i) &= \sum_{j=1}^N
  \left[
    A_j \hat{\phi}(\nu_j, -\mu_i) e^{-\tau/\nu_j} +
    B_j \hat{\phi}(-\nu_j, -\mu_i) e^{-(\tau_0-\tau)/\nu_j}
  \right]
  + B_i e^{-(\tau_0-\tau)/\nu_i}
\end{align}
for $i=N+1,\ldots,N+M$.

\appendix

\section{A note on particular solutions of the RTE}

To compute particular solutions we need to first find a Green's
function for the infinite medium RTE. Denote the Green's function by
$G(\tau,\pm\mu_i; x,\zeta_\alpha)$, where $\alpha = \pm 1, \ldots, \pm
N$ and $\zeta_{-i} = -\mu_i$. Then, we can show that the Green's
function must satisfy the jump conditions
\begin{align}
  &\lim_{\epsilon \rightarrow 0} \mu_i
  \left[
    G(x+\epsilon, \mu_i; x, \zeta_\alpha)
    -
    G(x-\epsilon, \mu_i; x, \zeta_\alpha)
  \right] = \delta_{i,\alpha} \\
  &\lim_{\epsilon \rightarrow 0} \mu_i
  \left[
    G(x+\epsilon, -\mu_i; x, \zeta_\alpha)
    -
    G(x-\epsilon, -\mu_i; x, \zeta_\alpha)
  \right] = -\delta_{-i,\alpha}
\end{align}
for $i=1,\ldots,N$. At regular nodes we can assume a solution of the
form
\begin{align}
  G(\tau,\mu_i;x,\zeta_\alpha) 
  &= \sum_{j=1}^N A_{j,\alpha} \phi(\nu_j,\mu_i) e^{-(\tau-x)/\nu_j} 
  \quad \tau>x
  \\
  &= \sum_{j=1}^N B_{j,\alpha} \phi(-\nu_j,\mu_i) e^{-(x-\tau)/\nu_j}
  \quad \tau<x
\end{align}
and
\begin{align}
  G(\tau,-\mu_i;x,\zeta_\alpha) 
  &= \sum_{j=1}^N A_{j,\alpha} \phi(\nu_j,-\mu_i) e^{-(\tau-x)/\nu_j} 
  \quad \tau>x
  \\
  &= \sum_{j=1}^N B_{j,\alpha} \phi(-\nu_j,-\mu_i) e^{-(x-\tau)/\nu_j}
  \quad \tau<x
\end{align}
Substituting these in the jump condition and using the orthogonality
of the eignefunctions we can show that we have
\begin{align}
  A_{k,\alpha} &= \frac{1}{N(\nu_k)} w_\alpha \phi(\nu_k,\zeta_\alpha) \\
  B_{k,\alpha} &= \frac{1}{N(\nu_k)} w_\alpha \phi(-\nu_k,\zeta_\alpha).
\end{align}

\end{document}
