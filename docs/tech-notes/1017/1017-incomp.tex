\documentclass[11pt, reqno]{amsart}
\usepackage{hyperref}
%% AMS packages and font files
\usepackage{amsmath}
\usepackage{amsfonts}
\usepackage{amsthm}
\usepackage[dvips]{graphicx}
\usepackage[usenames,dvipsnames]{color}
\usepackage{setspace}
\usepackage{fancyhdr}
% \pagestyle{fancyplain}

\DeclareMathAlphabet{\mathpzc}{OT1}{pzc}{m}{it}

%% Set page size properly
% \oddsidemargin  0.0in
% \evensidemargin 0.0in
% \textwidth      6.5in
% \textheight     9.0in
% \leftmargin     1.0in
% \rightmargin    1.0in

%% Autoscaled figures
\newcommand{\incfig}{\centering\includegraphics}
\setkeys{Gin}{width=0.9\linewidth,keepaspectratio}

%% Commonly used macros
\newcommand{\eqr}[1]{Eq.\thinspace(#1)}
\newcommand{\pfrac}[2]{\frac{\partial #1}{\partial #2}}
\newcommand{\pfracc}[2]{\frac{\partial^2 #1}{\partial #2^2}}
\newcommand{\pfraca}[1]{\frac{\partial}{\partial #1}}
\newcommand{\pfracb}[2]{\partial #1/\partial #2}
\newcommand{\pfracbb}[2]{\partial^2 #1/\partial #2^2}
\newcommand{\spfrac}[2]{{\partial_{#1}} {#2}}
\newcommand{\mvec}[1]{\mathbf{#1}}
\newcommand{\gvec}[1]{\boldsymbol{#1}}
\newcommand{\script}[1]{\mathpzc{#1}}
\newcommand{\eep}{\mvec{e}_\phi}
\newcommand{\eer}{\mvec{e}_r}
\newcommand{\eez}{\mvec{e}_z}
\newcommand{\iprod}[2]{\langle{#1}\rangle_{#2}}

\newtheorem{thm}{Theorem}
\newtheorem{lem}{Lemma}

\theoremstyle{definition}
\newtheorem{dfn}{Definition}

\title[Incompressible Flow]{Solving two-dimensional incompressible
  Euler equations}%
\author{Ammar H. Hakim}%
\date{}

\begin{document}
% header text
\lhead{Tech-Note 1017}
\maketitle

\section{The Basic Scheme}

We wish to solve the incompressible Euler equations written in
vorticity-streamfunction form
\begin{align}
  \pfrac{\chi}{t} + \nabla\cdot(\mvec{u}\chi) = 0 \label{eq:vort}
\end{align}
where $\chi$ is the fluid vorticity and $\mvec{u} =
\nabla\psi\times\eez$ is the fluid velocity. Here $\psi$ is the
streamfunction determined from a Poisson equation
\begin{align}
  \nabla^2 \psi = -\chi. \label{eq:stream}
\end{align}
As the flow is incompressible ($\nabla\cdot\mvec{u}=0$) we can rewrite
\eqr{\ref{eq:vort}} in the form
\begin{align}
  \pfrac{\chi}{t} + \{\chi,\psi\} = 0 \label{eq:vortPoisson}
\end{align}
where $\{\chi,\psi\}$ is the Poisson bracket operator defined by
\begin{align}
 \{\chi,\psi\} = \pfrac{\chi}{x}\pfrac{\psi}{y} - \pfrac{\chi}{y}\pfrac{\psi}{x}.
\end{align}

To discretize this equation we will continuous finite-elements (CG) to
solve the Poisson equation and discontinuous Galerkin finite-elements
(DG) to solve the vorticity equations. Let the domain be $\Omega$ and
divide it into cells denoted by $\script{C}_i$, $i=1,\ldots,N$, where
$N$ is the number of cells. Of course, the cells should be
non-overlapping, cover the domain and there should be no hanging
nodes.

We pick two sets of basis functions. For the CG method we denote the
basis functions as $\varphi_j$. We assume these are continuous and
defined over the complete domain $\Omega$. For the DG method we denote
the basis functions as $v_j$ and assume these are defined only in each
cell $\script{C}_i$. Hence, the solution for the vorticity is allowed
to be discontinuous across the cell edges, while the continuity for the
streamfunction is enforced\footnote{All of these definitions can be
  made more precise at the cost of turning the text into
  incomprehensible gobbledygook.}. The restriction of the global basis
function on a cell $\script{C}_i$ will be denoted by
$\varphi_j^{(i)}$.

We first define two sets of inner-products. Let $\langle f\rangle
_V$ be the integral
\begin{align}
  \langle f\rangle_V \equiv \int_V f d\mvec{x}
\end{align}
Here $d\mvec{x}$ is the volume element and $V$ can be either $\Omega$
or $\script{C}_i$. The second inner-product, $\langle
f\rangle_{\partial V}$ is the closed integral over the surface of the
complete domain or a single cell:
\begin{align}
  \langle f\rangle_{\partial V} \equiv \oint_{\partial V} f
  d\mvec{s}
\end{align}
where $d\mvec{s}$ is the surface element. 

We can derive the discrete weak-form of the vorticity equation by
multiplying \eqr{\ref{eq:vort}} by $v_j$ and integrating over a
\emph{cell} to get
\begin{align}
  \iprod{v_j\pfrac{\chi}{t}}{\script{C}_i} 
  +
  \iprod{v_j\mvec{u}\cdot\mvec{n} \widehat{\chi}}{\partial \script{C}_i}
  -
  \iprod{\nabla v_j\cdot \mvec{u}\chi}{\script{C}_i}
  = 0. \label{eq:weak_vort}
\end{align}
where $\mvec{n}$ is the outward normal to the surface element
$d\mvec{s}$. Here we have integrated by parts and defined
$\widehat{\chi}$ to be the vorticity along the edge of a cell,
computed using a suitable averaging or upwinding procedure. Similarly,
we can derive the discrete weak-form of the Poisson equation by
multiplying \eqr{\ref{eq:stream}} by $\varphi_j$ and integrating over
the \emph{complete domain} to get
\begin{align}
  \iprod{\varphi_j \nabla\psi \cdot \mvec{n}}{\partial \Omega}
  -
  \iprod{\nabla \varphi_j\cdot \nabla\psi}{\Omega}
  =
  -
  \iprod{\varphi_j\chi}{\Omega} \label{eq:weak_stream}
\end{align}

In \eqr{\ref{eq:weak_vort}} there appears a term like
$\mvec{u}\cdot\mvec{n}$ which needs to be evaluated on the edge of
each cell. To see that this term is continuous across the cell surface
we write it as
\begin{align}
  \mvec{u}\cdot\mvec{n}
  = (\nabla\psi\times \eez)\cdot\mvec{n}
  = \nabla\psi\cdot(\eez\times\mvec{n})
  = \nabla\psi\cdot\gvec{\tau}
\end{align}
where $\gvec{\tau} \equiv \eez\times\mvec{n}$ is a tangent vector
along the cell edge. As $\psi$ is continuous this means that its
gradient along the edge ($\nabla\psi\cdot\gvec{\tau}$) is also
continuous. However, note that $\nabla{\psi}$ itself need not be
continuous and, in general, will not be unless special basis functions
are selected. Also note that
\begin{align}
  \mvec{u}\cdot\gvec{\tau}
  = (\nabla\psi\times \eez)\cdot\gvec{\tau}
  = \nabla\psi\cdot(\eez\times\gvec{\tau})
  = -\nabla\psi\cdot\mvec{n}
\end{align}
Hence, the first term on the left side of \eqr{\ref{eq:weak_stream}}
is essentially the fluid circulation projected on the continuous basis
functions.

This completes the formal definition of the numerical scheme. Of
course, significant work is still needed to bring these equations into
a form that can be actually implemented.

\section{Conservation of Energy}

The discrete scheme, with the proper choice of basis functions,
conserves energy. To see this, consider the case of periodic boundary
conditions in which case the the first term on the left side of
\eqr{\ref{eq:weak_stream}} vanishes. Take the time-derivative of this
equation to get
\begin{align}
  \iprod{\nabla \varphi_j\cdot \nabla\pfrac{\psi}{t}}{\Omega}
  =
  \iprod{\varphi_j\pfrac{\chi}{t}}{\Omega}
\end{align}
Now, assume that the streamfunction is expanded as $\psi =
\sum_j\psi_j\varphi_j$. Then, multiply the above equation by $\psi_j$
and sum over all $j$ to get
\begin{align}
  \frac{dE}{dt}
  =
  \sum_j \psi_j \iprod{\varphi_j\pfrac{\chi}{t}}{\Omega}
\end{align}
where we have defined\footnote{All these manipulations are possible
  as the inner-product is a linear operator.}
\begin{align}
  \sum_j \psi_j \iprod{\nabla \varphi_j\cdot
    \nabla\pfrac{\psi}{t}}{\Omega}
  =
  \iprod{\nabla \psi\cdot \nabla\pfrac{\psi}{t}}{\Omega}  
  =
  \frac{d}{dt} \iprod{\frac{1}{2} |\nabla \psi|^2}{\Omega}
  \equiv
  \frac{dE}{dt}
\end{align}

Now, if in each cell $\script{C}_i$ we have
\begin{align}
  \textrm{span}(\varphi_1^{(i)},\varphi_2^{(i)},\ldots)
  \subseteq
  \textrm{span}(v_1,v_2,\ldots) \label{eq:span_cond}
\end{align}
where recall that $\varphi_j^{(i)}$ is the restriction of $\varphi_j$
on cell $\script{C}_i$, then we can write \eqr{\ref{eq:weak_vort}}
as
\begin{align}
  \iprod{\varphi_j^{(i)}\pfrac{\chi}{t}}{\script{C}_i} 
  +
  \iprod{\varphi_j^{(i)}\mvec{u}\cdot\mvec{n} \widehat{\chi}}{\partial \script{C}_i}
  -
  \iprod{\nabla \varphi_j^{(i)}\cdot \mvec{u}\chi}{\script{C}_i}
  = 0.
\end{align}
Now multiply this by $\psi_j$, sum over all $j$ and over all cells to
get
\begin{align}
  \sum_j \psi_j \iprod{\varphi_j \pfrac{\chi}{t}}{\Omega}
  =
  - 
  \sum_i \sum_j \psi_j \iprod{\varphi_j^{(i)}\mvec{u}\cdot\mvec{n} \widehat{\chi}}{\partial \script{C}_i}
  +
  \iprod{\nabla \psi \cdot \mvec{u}\chi}{\Omega}
\end{align}
The second term on the right side of this equation vanishes as $\nabla
\psi \cdot \mvec{u} = 0$. As the contribution from edge integrals have
opposite signs for cells sharing a common edge, the first term reduces
to an integral over the domain boundary
$\iprod{\psi\mvec{u}\cdot\mvec{n} \widehat{\chi}}{\partial\Omega}$,
which vanishes as the domain is periodic.

Hence, we have proved that $dE/dt = 0$ which means that energy is
conserved by the numerical scheme. Note that for this proof to work we
must select basis functions for the CG and DG scheme such that
\eqr{\ref{eq:span_cond}} is satisfied. This can be done by using, in
each cell, the \emph{same} basis functions for both CG and DG schemes.

\end{document}

For general boundary conditions (in which case the term on the left
side of \eqr{\ref{eq:weak_stream}} does not vanish) the net change in
energy from the numerical scheme is simply the integrated flux of
energy over the domain boundary.
