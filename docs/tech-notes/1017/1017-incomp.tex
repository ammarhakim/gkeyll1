\documentclass[11pt, reqno]{amsart}
\usepackage{hyperref}
%% AMS packages and font files
\usepackage{amsmath}
\usepackage{amsfonts}
\usepackage{amsthm}
\usepackage[dvips]{graphicx}
\usepackage[usenames,dvipsnames]{color}
\usepackage{setspace}
\usepackage{fancyhdr}
% \pagestyle{fancyplain}

\DeclareMathAlphabet{\mathpzc}{OT1}{pzc}{m}{it}

%% Set page size properly
% \oddsidemargin  0.0in
% \evensidemargin 0.0in
% \textwidth      6.5in
% \textheight     9.0in
% \leftmargin     1.0in
% \rightmargin    1.0in

%% Autoscaled figures
\newcommand{\incfig}{\centering\includegraphics}
\setkeys{Gin}{width=0.9\linewidth,keepaspectratio}

%% Commonly used macros
\newcommand{\eqr}[1]{Eq.\thinspace(#1)}
\newcommand{\pfrac}[2]{\frac{\partial #1}{\partial #2}}
\newcommand{\pfracc}[2]{\frac{\partial^2 #1}{\partial #2^2}}
\newcommand{\pfraca}[1]{\frac{\partial}{\partial #1}}
\newcommand{\pfracb}[2]{\partial #1/\partial #2}
\newcommand{\pfracbb}[2]{\partial^2 #1/\partial #2^2}
\newcommand{\spfrac}[2]{{\partial_{#1}} {#2}}
\newcommand{\mvec}[1]{\mathbf{#1}}
\newcommand{\gvec}[1]{\boldsymbol{#1}}
\newcommand{\script}[1]{\mathpzc{#1}}
\newcommand{\eep}{\mvec{e}_\phi}
\newcommand{\eer}{\mvec{e}_r}
\newcommand{\eez}{\mvec{e}_z}
\newcommand{\iprod}[2]{\langle{#1}\rangle_{#2}}

\newtheorem{thm}{Theorem}
\newtheorem{lem}{Lemma}

\theoremstyle{definition}
\newtheorem{dfn}{Definition}

\title[Incompressible Flow]{Solving two-dimensional incompressible
  Euler equations}%
\author{Ammar H. Hakim}%
\date{}

\begin{document}
% header text
\lhead{Tech-Note 1017}
\maketitle

\section{The Basic Scheme}

We wish to solve the incompressible Euler equations written in
vorticity-streamfunction form
\begin{align}
  \pfrac{\chi}{t} + \nabla\cdot(\mvec{u}\chi) = 0 \label{eq:vort}
\end{align}
where $\chi$ is the fluid vorticity and $\mvec{u} =
\nabla\psi\times\eez$ is the fluid velocity. Here $\psi$ is the
streamfunction determined from a Poisson equation
\begin{align}
  \nabla^2 \psi = -\chi. \label{eq:stream}
\end{align}
As the flow is incompressible ($\nabla\cdot\mvec{u}=0$) we can rewrite
\eqr{\ref{eq:vort}} in the form
\begin{align}
  \pfrac{\chi}{t} + \{\chi,\psi\} = 0 \label{eq:vortPoisson}
\end{align}
where $\{\chi,\psi\}$ is the Poisson bracket operator defined by
\begin{align}
 \{\chi,\psi\} = \pfrac{\chi}{x}\pfrac{\psi}{y} - \pfrac{\chi}{y}\pfrac{\psi}{x}.
\end{align}

To discretize this equation we will continuous finite-elements (CG) to
solve the Poisson equation and discontinuous Galerkin finite-elements
(DG) to solve the vorticity equations. Let the domain be $\Omega$ and
divide it into cells denoted by $\script{C}_i$, $i=1,\ldots,N$, where
$N$ is the number of cells. Of course, the cells should be
non-overlapping, cover the domain and there should be no hanging
nodes.

We pick two sets of basis functions. For the CG method we denote the
basis functions as $\varphi_j$. We assume these are continuous and
defined over the complete domain $\Omega$. For the DG method we denote
the basis functions as $v_j$ and assume these are defined only in each
cell $\script{C}_i$. Hence, the solution for the vorticity is allowed
to be discontinuous across the cell edges, while the continuity for the
streamfunction is enforced\footnote{All of these definitions can be
  made more precise at the cost of turning the text into
  incomprehensible gobbledygook.}. The restriction of the global basis
function on a cell $\script{C}_i$ will be denoted by
$\varphi_j^{(i)}$.

We first define two sets of inner-products. Let $\langle f\rangle
_V$ be the integral
\begin{align}
  \langle f\rangle_V \equiv \int_V f d\mvec{x}
\end{align}
Here $d\mvec{x}$ is the volume element and $V$ can be either $\Omega$
or $\script{C}_i$. The second inner-product, $\langle
f\rangle_{\partial V}$ is the closed integral over the surface of the
complete domain or a single cell:
\begin{align}
  \langle f\rangle_{\partial V} \equiv \oint_{\partial V} f
  d\mvec{s}
\end{align}
where $d\mvec{s}$ is the surface element. 

We can derive the discrete weak-form of the vorticity equation by
multiplying \eqr{\ref{eq:vort}} by $v_j$ and integrating over a
\emph{cell} to get
\begin{align}
  \iprod{v_j\pfrac{\chi}{t}}{\script{C}_i} 
  +
  \iprod{v_j\mvec{u}\cdot\mvec{n} \widehat{\chi}}{\partial \script{C}_i}
  -
  \iprod{\nabla v_j\cdot \mvec{u}\chi}{\script{C}_i}
  = 0. \label{eq:weak_vort}
\end{align}
where $\mvec{n}$ is the outward normal to the surface element
$d\mvec{s}$. Here we have integrated by parts and defined
$\widehat{\chi}$ to be the vorticity along the edge of a cell,
computed using a suitable averaging or upwinding procedure. Similarly,
we can derive the discrete weak-form of the Poisson equation by
multiplying \eqr{\ref{eq:stream}} by $\varphi_j$ and integrating over
the \emph{complete domain} to get
\begin{align}
  \iprod{\varphi_j \nabla\psi \cdot \mvec{n}}{\partial \Omega}
  -
  \iprod{\nabla \varphi_j\cdot \nabla\psi}{\Omega}
  =
  -
  \iprod{\varphi_j\chi}{\Omega} \label{eq:weak_stream}
\end{align}
Note that this equation can also be obtained by summing over all cells
the \emph{local} weak-form of the streamfunction equation
\begin{align}
  \iprod{\varphi_j^{(i)} \nabla\psi \cdot \mvec{n}}{\partial \script{C}_i}
  -
  \iprod{\nabla \varphi_j^{(i)}\cdot \nabla\psi}{\script{C}_i}
  =
  -
  \iprod{\varphi_j^{(i)}\chi}{\script{C}_i} \label{eq:local_weak_stream}
\end{align}
This form of the discrete Poisson equation is actually more useful as
it allows for the systematic construction of the global discrete
equation by the process of finite-element \emph{assembly}. In this
assembly process the contribution from the first term of the left side
of this equation vanishes for all shared edges, leaving us with the
first term on the left side of \eqr{\ref{eq:weak_stream}}.

In \eqr{\ref{eq:weak_vort}} there appears a term like
$\mvec{u}\cdot\mvec{n}$ which needs to be evaluated on the edge of
each cell. To see that this term is continuous across the cell surface
we write it as
\begin{align}
  \mvec{u}\cdot\mvec{n}
  = (\nabla\psi\times \eez)\cdot\mvec{n}
  = \nabla\psi\cdot(\eez\times\mvec{n})
  = \nabla\psi\cdot\gvec{\tau}
\end{align}
where $\gvec{\tau} \equiv \eez\times\mvec{n}$ is a tangent vector
along the cell edge. As $\psi$ is continuous this means that its
gradient along the edge ($\nabla\psi\cdot\gvec{\tau}$) is also
continuous. However, note that $\nabla{\psi}$ itself need not be
continuous and, in general, will not be unless special basis functions
are selected. Also note that
\begin{align}
  \mvec{u}\cdot\gvec{\tau}
  = (\nabla\psi\times \eez)\cdot\gvec{\tau}
  = \nabla\psi\cdot(\eez\times\gvec{\tau})
  = -\nabla\psi\cdot\mvec{n}
\end{align}
Hence, the first term on the left side of \eqr{\ref{eq:weak_stream}}
is essentially the fluid circulation projected on the continuous basis
functions.

This completes the formal definition of the numerical scheme. Of
course, significant work is still needed to bring these equations into
a form that can be actually implemented.

\section{Conservation of Energy}

The discrete scheme, with the proper choice of basis functions,
conserves energy. To see this, consider the case of periodic boundary
conditions in which case the the first term on the left side of
\eqr{\ref{eq:weak_stream}} vanishes. Take the time-derivative of this
equation to get
\begin{align}
  \iprod{\nabla \varphi_j\cdot \nabla\pfrac{\psi}{t}}{\Omega}
  =
  \iprod{\varphi_j\pfrac{\chi}{t}}{\Omega}
\end{align}
Now, assume that the streamfunction is expanded as $\psi =
\sum_j\psi_j\varphi_j$. Then, multiply the above equation by $\psi_j$
and sum over all $j$ to get
\begin{align}
  \frac{dE}{dt}
  =
  \sum_j \psi_j \iprod{\varphi_j\pfrac{\chi}{t}}{\Omega}
\end{align}
where we have defined\footnote{All these manipulations are possible
  as the inner-product is a linear operator.}
\begin{align}
  \sum_j \psi_j \iprod{\nabla \varphi_j\cdot
    \nabla\pfrac{\psi}{t}}{\Omega}
  =
  \iprod{\nabla \psi\cdot \nabla\pfrac{\psi}{t}}{\Omega}  
  =
  \frac{d}{dt} \iprod{\frac{1}{2} |\nabla \psi|^2}{\Omega}
  \equiv
  \frac{dE}{dt}
\end{align}

Now, if in each cell $\script{C}_i$ we have
\begin{align}
  \textrm{span}(\varphi_1^{(i)},\varphi_2^{(i)},\ldots)
  \subseteq
  \textrm{span}(v_1,v_2,\ldots) \label{eq:span_cond}
\end{align}
where recall that $\varphi_j^{(i)}$ is the restriction of $\varphi_j$
on cell $\script{C}_i$, then we can specialize
\eqr{\ref{eq:weak_vort}} as
\begin{align}
  \iprod{\varphi_j^{(i)}\pfrac{\chi}{t}}{\script{C}_i} 
  +
  \iprod{\varphi_j^{(i)}\mvec{u}\cdot\mvec{n} \widehat{\chi}}{\partial \script{C}_i}
  -
  \iprod{\nabla \varphi_j^{(i)}\cdot \mvec{u}\chi}{\script{C}_i}
  = 0.
\end{align}
Now multiply this by $\psi_j$, sum over all $j$ and over all cells to
get
\begin{align}
  \sum_j \psi_j \iprod{\varphi_j \pfrac{\chi}{t}}{\Omega}
  =
  - 
  \sum_i \sum_j \psi_j \iprod{\varphi_j^{(i)}\mvec{u}\cdot\mvec{n} \widehat{\chi}}{\partial \script{C}_i}
  +
  \iprod{\nabla \psi \cdot \mvec{u}\chi}{\Omega}
\end{align}
The second term on the right side of this equation vanishes as $\nabla
\psi \cdot \mvec{u} = 0$. As the contribution from edge integrals have
opposite signs for cells sharing a common edge, the first term reduces
to an integral over the domain boundary
$\iprod{\psi\mvec{u}\cdot\mvec{n} \widehat{\chi}}{\partial\Omega}$,
which vanishes as the domain is periodic.

Hence, we have proved that $dE/dt = 0$ which means that energy is
conserved by the semi-discrete scheme\footnote{The fully discrete
  scheme, for example with Runge-Kutta time-stepping, will still
  dissipate energy as the time-stepping is not reversible. To get a
  fully conservative scheme we need to use a centered
  time-discretization.}. Note that for this proof to work we must
select basis functions for the CG and DG scheme such that
\eqr{\ref{eq:span_cond}} is satisfied. This can be done by using, in
each cell, the \emph{same} basis functions for both CG and DG schemes.

\section{Nodal Basis Functions}

We will use \emph{nodal basis functions}. These are constructed by
picking a set of nodes $(x_i,y_i)$, $i=1,\ldots,K$, in each cell and
defining basis functions $N_j(x,y)$, $j=1,\ldots,K$, such that
\begin{align}
  N_j(x_i,y_i) = \delta_{ji}
\end{align}
Hence, if a function is expanded in a cell as
\begin{align}
  f(x,y) = \sum_{i=1}^K f_{i}N_i(x,y)
\end{align}
then the expansion coefficients are simply the value of the function
at the corresponding node. Once a set of nodal basis functions are
selected they can be used in the discrete weak-form by setting
$v_j=\varphi_j^{(i)}=N_j(x,y)$ for each cell $\script{C}_i$ in the
grid. Note that with this choice of basis functions for both the CG
and DG scheme leads to an energy conserving spatial discretization.

Expanding the vorticity in a cell $\script{C}_i$ as
\begin{align}
  \chi(x,y) = \sum_{k=1}^K \chi_k N_k(x,y)
\end{align}
and using in \eqr{\ref{eq:weak_vort}} we get
\begin{align}
  \sum_{k} \iprod{N_jN_k}{\script{C}_i} \pfrac{\chi_k}{t}
  +
  \iprod{N_j\mvec{u}\cdot\mvec{n} \widehat{\chi}}{\partial \script{C}_i}
  -
  \iprod{\nabla N_j\cdot\mvec{u} \chi}{\script{C}_i}
  = 0. \label{eq:dg_vort}
\end{align}
In the above equations two integrals still need to be performed: the
first over each face of a cell and the second over the cell
volume. For this one needs to use Gaussian quadrature of the
appropriate order to avoid alaising errors from inexact integration.

Expanding the streamfunction in a cell $\script{C}_i$ as
\begin{align}
  \psi^{(i)}(x,y) = \sum_{k=1}^K \psi_k^{(i)} N_k(x,y)
\end{align}
and using in \eqr{\ref{eq:local_weak_stream}} we get\footnote{The
  surface integral term from each cell is ignored here as it vanishes
  on assembly. It contribution over the domain boundary needs be to
  accounted for, however, for non-periodic boundary conditions.}
\begin{align}
  \sum_k
  \iprod{\nabla N_j\cdot \nabla N_k}{\script{C}_i} \psi_k^{(i)}
  =
  \iprod{N_j\chi}{\script{C}_i}
\end{align}
We can rewrite this in a matrix form by introducing the \emph{local
  stiffness}  matrix
\begin{align}
  \mvec{K}^{(i)}_{j,k} \equiv \iprod{\nabla N_j\cdot \nabla N_k}{\script{C}_i},
\end{align}
the column vectors $\gvec{\psi}^{(i)} =
[\psi^{(i)}_1,\ldots,\psi^{(i)}_K]^T$ and $\gvec{S}^{(i)} =
[\iprod{N_1\chi}{\script{C}_i},\ldots,\iprod{N_K\chi}{\script{C}_i}]^T$
to get
\begin{align}
  \mvec{K}^{(i)}\gvec{\psi}^{(i)}
  =
  \mvec{S}^{(i)} \label{eq:local_cg_form}
\end{align}
To assemble the final global stiffness matrix we first introduce a
connectivity matrix in each cell that maps the local node numbers to
the global node numbers\footnote{In basic presentations of FEM this
  connectivity matrix is not used. However, it makes the FE procedure
  more systematic specially in higher (than one) dimensions and when
  the grid is unstructured. Without this matrix it is virtually
  impossible to figure out what the global matrices look like on an
  unstructured grid.}. The connectivity matrix essentially identifies
shared nodes and enforces continuity required in the CG scheme.  In a
cell $\script{C}_i$ this matrix is denoted by $\mvec{C}^{(i)}$ and has
shape $N_g\times K$, where $N_g$ are the global degrees of freedom and
is defined such that
\begin{align}
  (\mvec{C}^{(i)})^T \gvec{\psi} = \gvec{\psi}^{(i)}
\end{align}
where $\gvec{\psi}$ is the column vector (of length $N_g$) of the
stream function on complete grid. Premultiplying
\eqr{\ref{eq:local_cg_form}} by $\mvec{C}^{(i)}$ and using the above
definition and summing over all cells we get
\begin{align}
  \sum_i
  \mvec{C}^{(i)}
  \mvec{K}^{(i)}
  (\mvec{C}^{(i)})^T
  \gvec{\psi}
  =
  \sum_i
  \mvec{C}^{(i)} \mvec{S}^{(i)}.
\end{align}
This completes the definition of the final form of the discretized
spatial operator\footnote{For general boundary conditions there will
  be an additional term corresponding to the first term on the left
  side of \eqr{\ref{eq:weak_stream}}. For periodic BCs the
  connectivity matrix takes care of identifying appropriate nodes on
  the domain boundary.}. Note that the global stiffness matrix,
$\mvec{K}$, is hence
\begin{align}
  \mvec{K}
  =
  \sum_i
  \mvec{C}^{(i)}
  \mvec{K}^{(i)}
  (\mvec{C}^{(i)})^T.
\end{align}
Notice that the connectivity matrix automatically puts the
contribution of each local stiffness matrix in the correct location in
the global matrix. Of course, the connectivity matrix is very sparse
and does not need to be explicitly stored and these multiplication can
be performed trivially during the process of assembly.

\end{document}

For general boundary conditions (in which case the term on the left
side of \eqr{\ref{eq:weak_stream}} does not vanish) the net change in
energy from the numerical scheme is simply the integrated flux of
energy over the domain boundary.
