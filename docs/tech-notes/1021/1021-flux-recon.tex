\documentclass[11pt, reqno]{amsart}
\usepackage{hyperref}
%% AMS packages and font files
\usepackage{amsmath}
\usepackage{amsfonts}
\usepackage{amsthm}
\usepackage[dvips]{graphicx}
\usepackage[usenames,dvipsnames]{color}
\usepackage{setspace}
\usepackage{fancyhdr}
% \pagestyle{fancyplain}

\DeclareMathAlphabet{\mathpzc}{OT1}{pzc}{m}{it}

%% Set page size properly
% \oddsidemargin  0.0in
% \evensidemargin 0.0in
% \textwidth      6.5in
% \textheight     9.0in
% \leftmargin     1.0in
% \rightmargin    1.0in

%% Autoscaled figures
\newcommand{\incfig}{\centering\includegraphics}
\setkeys{Gin}{width=0.9\linewidth,keepaspectratio}

%% Commonly used macros
\newcommand{\eqr}[1]{Eq.\thinspace(#1)}
\newcommand{\pfrac}[2]{\frac{\partial #1}{\partial #2}}
\newcommand{\pfracc}[2]{\frac{\partial^2 #1}{\partial #2^2}}
\newcommand{\pfraca}[1]{\frac{\partial}{\partial #1}}
\newcommand{\pfracb}[2]{\partial #1/\partial #2}
\newcommand{\pfracbb}[2]{\partial^2 #1/\partial #2^2}
\newcommand{\spfrac}[2]{{\partial_{#1}} {#2}}
\newcommand{\mvec}[1]{\mathbf{#1}}
\newcommand{\gvec}[1]{\boldsymbol{#1}}
\newcommand{\script}[1]{\mathpzc{#1}}
\newcommand{\eep}{\mvec{e}_\phi}
\newcommand{\eer}{\mvec{e}_r}
\newcommand{\eez}{\mvec{e}_z}
\newcommand{\iprod}[2]{\langle{#1}\rangle_{#2}}

\newtheorem{thm}{Theorem}
\newtheorem{lem}{Lemma}
\newtheorem{example}{Example}

\theoremstyle{definition}
\newtheorem{dfn}{Definition}

\title[Flux Reconstruction]{A summary of flux-reconstruction schemes
  for advection and diffusion terms}%
\author{Ammar H. Hakim}%
\date{}%

\begin{document}
% header text
\lhead{Tech-Note 1021}%
\maketitle

Flux reconstruction is methodology introduced by H.T Huynh
(H.T. Huynh, AIAA 2007-4079) that can be used to construct a family of
schemes for the solution of advection-diffusion type equations. In
this document I summarize the main results and work out explicit
formulas for certain classes of flux reconstruction schemes.

\section{Polynomial reconstruction}

In a DG scheme the solution in a cell $C_j\equiv
[x_{j-1/2},x_{j+1/2}]$ is represented (usually) as polynomial
expansion in that cell. For an expansion containing $K$ pieces of
independent data (for example, values of the function at nodes or
expansion coefficients) the polynomial is of degree $K-1$. Let
$u_j(x)$ be the DG representation of a function and $\phi_{j,k}(x)$,
$k=1,\ldots,K$ be basis functions in cell $I_j$. Then, we have the
expansion
\begin{align}
  u_j(x) = \sum_{k=1}^K u_{j,k} \phi_{j,k}(x).
\end{align}
The basis functions can be, for example, Lagrange interpolation
polynomials. At a cell interface $x_{j+1/2}$ let $u^-_{j+1/2}$ and
$u^+_{j+1/2}$ be the values of the function just to the left and right
of the interface, respectively
\begin{align}
  u_{j+1/2}^- = u_{j,k}(x_{j+1/2}) \quad\mathrm{and}\quad
  u_{j+1/2}^+ = u_{j+1,k}(x_{j+1/2}).
\end{align}
Note that in general $u^-_{j+1/2}\ne u^+_{j+1/2}$. The flux
reconstruction approach is essentially a method of reconstructing a
new function $\bar{u}_j(x)$ that is continuous at cell
interfaces. 

There are several ways in which such a reconstruction can be
accomplished. In the FR approach the basic idea is to reconstruct a
polynomial of one higher degree (i.e. degree $K$) with the additional
conditions that the reconstructed function $\bar{u}_j(x)$ take on some
common value at each interface. That is, let us assume we have some
means (yet unspecified) of constructing a common value, which we
denote by $u_{j+1/2}^c$, at each interface. Then, we require that
\begin{align}
  \bar{u}_{j}(x_{j+1/2}) = u_{j+1/2}^c, \quad\quad
  \bar{u}_{j}(x_{j-1/2}) = u_{j-1/2}^c.
\end{align}

To develop the reconstruction function consider the polynomials of
degree $K$, $g^{LB}(\eta)$ and $g^{RB}(\eta)$, in the interval $I\in
[-1,1]$, such that
\begin{align}
  g^{LB}(-1) = 1, \qquad
  g^{LB}(1) = 0 \label{eq:glb}
\end{align}
and 
\begin{align}
  g^{RB}(-1) = 0, \qquad
  g^{RB}(1) = 1. \label{eq:grb}
\end{align}
Then, using these polynomials we can write down the required
reconstruction as
\begin{align}\label{eq:fr_recon}
  \boxed{
  \bar{u}_j(x)
  =
  u_j(x)
  +
  [u^c_{j-1/2} - u_j(-1)]g^{LB}(\eta)
  +
  [u^c_{j+1/2} - u_j(1)]g^{RB}(\eta)
}
\end{align}
where $\eta = \eta(x) \equiv 2(x-x_j)/\Delta x_j$ and $x_j$ is the
coordinate of the cell center and $\Delta x_j$ is the cell size. Also,
$u_j(\pm 1) \equiv u_j(x_{j\pm 1/2})$.

In the form \eqr{\ref{eq:fr_recon}} the reconstructed polynomial
appears as a correction to the function $u_j(x)$. The first correction
term ensures that at $x_{j-1/2}$ the reconstruction takes the common
value $u^c(x_{j-1/2})$ and the second correction term ensures that at
$x_{j+1/2}$ the reconstruction takes the common value
$u^c(x_{j+1/2})$. Hence, the polynomials $g^{LB}(\eta)$ and
$g^{RB}(\eta)$ are called the \emph{left} and the \emph{right}
correction functions, respectively. Note that the first
(second) correction term does not affect the right (left) boundary.

\section{Correction functions}

The correction functions are the critical pieces in reconstructing a
continuous polynomial. Let $g(\eta)\equiv g^{LB}(\eta)$. Once this is
determined, then we use reflection to determine the other correction
function, i.e. $g^{RB}(\eta) = g(-\eta)$. To determine $g(\eta)$ we
already have two constraints
\begin{align}\label{eq:gcond}
  g(-1) = 1, \qquad g(1) = 0.
\end{align}
As $g(\eta)$ is of degree $K$ we need $K-1$ additional pieces of
information to determine it. Different correction functions can be
constructed by imposing different constraints. Note that each choice
of the correction function yields a \emph{different} reconstruction
function and hence, when used to solve PDEs, a \emph{different}
scheme.

\subsection{The DG correction polynomial}

Let $\script{V}^k(I)$ be the space of polynomials of degree at most
$k$ on $I$. This space has $k+1$ independent functions. Then, the
first choice of $g(\eta)$ can be constructed by requiring that
$g(\eta)$ is orthogonal to $\script{V}^{K-2}(I)$. Note that this gives
$K-2+1 = K-1$ conditions, hence determining $g(\eta)$. 

Consider the \emph{modified} Radau polynomial defined as
\begin{align}
  R_k(\eta) = \frac{(-1)^k}{2}
  \left( 
    P_k(\eta) - P_{k-1}(\eta)
  \right)
\end{align}
where, $P_l(\eta)$ is the Legendre polynomial of degree $l$. As
$P_l(-1) = (-1)^l$ and $P(1) = 1$ it can be easily verified that this
polynomial satisfies \eqr{\ref{eq:gcond}}. Further, as $P_l(\eta)$ is
orthogonal to $\script{V}^{l-1}(I)$, $R_k(\eta)$ is orthogonal to
$\script{V}^{k-2}$. Hence, we can use this as the correction function,
i.e. we set
\begin{align}
  \boxed{
  g_{DG}(\eta) = R_K(\eta)
}
\end{align}
The subscript ``DG'' indicates that when used for hyperbolic problems
the resulting scheme is exactly equivalent to the standard DG scheme.

\subsection{The Gauss correction polynomials}

The second choice requires that the correction polynomials vanishes at
the $K-1$ Gauss quadrature points in $I$. This yields the needed
additional constraints that determine the polynomials
$g_{Ga}(\eta)$. It can be shown that this polynomial can be written as
\begin{align}
  \boxed{
  g_{Ga}(\eta) = 
  \frac{K}{2K-1}R_K(\eta) 
  +
  \frac{K-1}{2K-1}R_{K-1}(\eta)
}
\end{align}

\subsection{The Lumped Lobatto correction polynomials}

Another choice is to enforce that $g'(\eta)$ vanishes at $K-1$ of the
$K$ Lobatto quadrature points in $I$. This yields the polynomials
\begin{align}
  \boxed{
  g_{Lo}(\eta) = 
  \frac{K-1}{2K-1}R_K(\eta) 
  +
  \frac{K}{2K-1}R_{K-1}(\eta)
}
\end{align}
The reason this is called the \emph{lumped} Lobatto polynomial is that
the corrections to the \emph{derivatives} of $u_j$ are all lumped to
the left boundary. For use in numerical schemes it is the derivatives
that are usually required, and hence this choice can lead to a more
efficient scheme.

Figure\thinspace(\ref{fig:gpoly}) shows the correction polynomials for
the case of $K=2$ and $K=4$.

\begin{figure}\label{fig:gpoly}
  \setkeys{Gin}{width=0.45\linewidth,keepaspectratio}
  \incfig{gK2.png}
  \incfig{gK4.png}
  \caption{Correction polynomials for $K=2$ (left) and $K=4$.}
\end{figure}

\section{Solving a scalar hyperbolic equation}

The reconstruction technique outlined above can be used to solve
hyperbolic equations. For simplicity, consider the scalar hyperbolic
equation
\begin{align}
  \pfrac{q}{t} + \pfrac{f}{x} = 0
\end{align}
where $q(x,t)$ is the conserved variable and $f = f(q)$ is the flux
function. Also, we will use a \emph{nodal} scheme and assume that the
solution is evolved at $K$ points in each cell, i.e.
\begin{align}
  q_j(x) = \sum_{k=1}^K q_{j,k} \phi_{j,k}(x)
\end{align}
where $q_{j,k}$ is the value of the conserved variable at node
$k=1,\ldots,K$ in cell $j$.

To solve this equation in the FR framework, we first need to define a
\emph{numerical flux-function}
\begin{align}
  F_{j+1/2} \equiv F(q^-_{j+1/2},q^+_{j+1/2})
\end{align}
where $q^-_{j+1/2}$ ($q^+_{j+1/2}$) are is the value of the conserved
variable just to the left (right) of interface $x_{j+1/2}$. Any
numerical flux-function traditionally used in finite-volume or DG
schemes can be used. For example, one can use the numerical flux
\begin{align}
  F(q^-_{j+1/2},q^+_{j+1/2})
  =
  \frac{1}{2}\big(f(q^+_{j+1/2}) + f(q^-_{j+1/2})\big)
  -
  \frac{1}{2}|\lambda|
  (q^+_{j+1/2} - q^-_{j+1/2})
\end{align}
where $\lambda = \lambda(q^-_{j+1/2},q^+_{j+1/2})$ is some estimate of
the wave-speed at the interface.

The next step is to construct a polynomial expansion, $f_j(x)$, in
each cell
\begin{align}
  f_j(x) = \sum_{k=1}^K f_{j,k} \phi_{j,k}(x).
\end{align}
where $f_{j,k} = f(q_{j,k})$. Note that, in general, for a nonlinear
flux function $f_j(x) \ne f(q_j)$.

We next reconstruct the flux in each cell such that the reconstructed
flux matches the numerical flux at the interfaces, and is one degree
higher than the degree used to represent the solution. That is, we use
the FR procedure developed above to write
\begin{align}\label{eq:flux_recon}
  \bar{f}_j(x)
  =
  f_j(x)
  +
  [F_{j-1/2} - f_j(-1)]g^{LB}(\eta)
  +
  [F_{j+1/2} - f_j(1)]g^{RB}(\eta)
\end{align}
where now $f_j(\pm 1) = f_j(x_{j\pm 1/2},t)$.

With the reconstructed flux the solution can be updated by solving the
ODE
\begin{align}
  \frac{d q_{j,k}}{dt} = -\frac{d\bar{f}_j(x_k)}{dx}
\end{align}
using a suitable time-stepping scheme.

\section{Computing second derivatives}

To solve equations with second order derivatives, for example
advection-diffusion equations, we need to discretize terms of the form
\begin{align}
  q = \frac{\partial^2 u}{\partial x^2}.
\end{align}
For this, first consider how one can compute the first derivatives
\begin{align}
  g_j(x) = \frac{\partial u_j(x)}{\partial x}
\end{align}
in each cell $j$ given a DG representation $u_j(x)$. As $u_j(x)$ is
discontinuous, one can use the FR approach to reconstruct a continuous
function $\bar{u}_j(x)$, which can then be used to compute the
derivative $g_j(x)$ in each cell. Unlike in the hyperbolic case there
is no obvious way to compute a common value at each interface. One
option is to use a weighted average of the form
\begin{align}\label{eq:ucomm}
  u^c_{j+1/2} = \kappa u_{j+1/2}^-
  +
  (1-\kappa) u_{j+1/2}^+
\end{align}
where $\kappa \in [0,1]$. Using $\kappa=0$ or $\kappa=1$ will lead to
a one-sided average, while using $\kappa=1/2$ will lead to a straight
average. Once the reconstruction is complete, we use
\begin{align}
  g_j(x) = \frac{\partial \bar{u}_j(x)}{\partial x}
\end{align}
to compute the needed gradient.

To determine the second derivative we now need to compute
\begin{align}
  q_j(x) = \frac{\partial g_j(x)}{\partial x}.
\end{align}
In general, $g_j(x)$ is discontinuous. Hence, one could use the FR
procedure again to compute a continuous function, with the common
interface values given by
\begin{align}\label{eq:gcomm}
  g_{j+1/2}^c = (1-\kappa) g_{j+1/2}^-
  +
  \kappa g_{j+1/2}^+.
\end{align}
Note the switch in the weights between Eqns.\thinspace(\ref{eq:ucomm})
and (\ref{eq:gcomm}). However, the problem with this approach is that
the the resulting scheme for the second derivative will lead to, in
general, a 5-point stencil, instead of a compact 3-point stencil.

\end{document}

As usual, the flux function must satisfy the consistency condition
\begin{align}
  F(q,q) = f(q)
\end{align}
and be a non-increasing function of the first parameter and a
non-decreasing function of the second parameter.