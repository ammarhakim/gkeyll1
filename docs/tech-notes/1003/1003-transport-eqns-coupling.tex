\documentclass[11pt]{article}
%% AMS packages and font files
\usepackage{amsmath}
\usepackage{amsfonts}
\usepackage[dvips]{graphicx}
\usepackage[usenames,dvipsnames]{color}
\usepackage{setspace}
\usepackage{fancyhdr}
%\pagestyle{fancyplain}

\DeclareMathAlphabet{\mathpzc}{OT1}{pzc}{m}{it}

%% Set page size properly
\oddsidemargin  0.0in
\evensidemargin 0.0in
\textwidth      6.5in
\textheight     9.0in
\headheight     1.0in
\headsep        0.0in
\leftmargin      1.0in
\rightmargin      1.0in
\topmargin      -.75in

%% Autoscaled figures
\newcommand{\incfig}{\centering\includegraphics}
\setkeys{Gin}{width=0.9\linewidth,keepaspectratio}

%% Commonly used macros
\newcommand{\eqr}[1]{Eq.\thinspace(#1)}
\newcommand{\pfrac}[2]{\frac{\partial #1}{\partial #2}}
\newcommand{\pfraca}[1]{\frac{\partial}{\partial #1}}
\newcommand{\pfracb}[2]{\partial #1/\partial #2}
\newcommand{\mvec}[1]{\mathbf{#1}}
\newcommand{\gvec}[1]{\boldsymbol{#1}}
\newcommand{\script}[1]{\mathpzc{#1}}

\title{Transport equations, finite-volume schemes and coupling
  algorithms}
\author{Ammar H. Hakim}
\date{}

\begin{document}
% header text
\lhead{Tech-Note 1003}
\maketitle

\section{Transport Equations and Finite-Volume Schemes}

Consider the one-dimensional transport equation written in the form
\begin{align}
  \pfrac{Q}{t} + \frac{1}{V'} \pfraca{x} \left( V' \Gamma \right) = S
\end{align}
where $Q = Q(x,t)$ is an unknown quantity, $\Gamma(x) =
\Gamma(Q,\pfracb{Q}{x})$ is a flux function, $S(x,t) = S(Q,x,t)$ is a
source function and $V' = V'(x)$ is a given function. To descritize
this equation consider a finite-volume grid in which a cell is defined
as $C_i \equiv [x_i, x_{i+1}]$, with $i=1,\ldots,N$ and the unknown
quantities are located at cell-centers $Q_i^n = Q(x_{i+1/2},t_n)$,
fluxes are located at cell-edges $\Gamma_i = \Gamma(x_i)$ and sources
are located at cell-centers $S_i = S(x_{i+1/2},t)$. Here, $x_{i+1/2}
\equiv (x_i+x_{i+1})/2$ and $\Delta x_i \equiv (x_{i+1}-x_i)$. Using a
second order central difference for the spatial operator and
first-order implicit difference for the time derivate we get an update
formula
\begin{align}
  Q_i^{n+1} = Q^n_i - \frac{\Delta t}{V'(x_i) \Delta x_i}
  \left (
    V'(x_{i+1/2}) F_{i+1} - V'(x_{i-1/2}) F_i
  \right)
  +
  \Delta t S_i \equiv D(F_{i+1}, F_{i}) \label{eqn:update-form}
\end{align}
where
\begin{align}
  F_i = H(Q_i^{n+1}, Q_{i-1}^{n+1}) \label{eqn:flux-func}
\end{align}
is the numerical flux function defined such that $H(Q,Q) =
\Gamma(Q,\pfracb{Q}{x})$.

A standard implicit method would solve the system of equations arising
from inserting the definition of the numerical flux into the update
formula, resulting a system of non-linear algebraic equations for the
$N$ unknowns $Q_i^{n+1}$. However, in this note we consider the
possibility of instread solving the system of coupled $2N$ algebraic
equations (dropping the superscripts)
\begin{align}
  Q_i - D(F_{i+1}, F_i) &= 0  \label{eqn:q-eqn} \\
  F_i - H(Q_i, Q_{i-1}) &= 0. \label{eqn:f-eqn}
\end{align}
for the $2N$ unknowns $[Q_1, \ldots, Q_N, F_1, \ldots F_N
]^T$. Letting $\mvec{V} \equiv [\mvec{Q}, \mvec{F}]^T$, where
$\mvec{Q} \equiv [Q_1, \ldots, Q_N]^T$ and $\mvec{F} \equiv [F_1,
\ldots, F_N]^T$ we can write the problem schematically as finding the
roots of the non-linear system of algebraic equations
\begin{align}
  \mvec{G}(\mvec{V}) = \mvec{0} \label{eqn:g-eqn}
\end{align}
where $G_i \equiv Q_i - D(F_{i+1}, F_i)$, and $G_{i+N} \equiv F_i -
F(Q_i, Q_{i-1})$ for $i=1,\ldots,N$. In the following also let
$\mvec{D} \equiv [D_1,\ldots,D_N]^T$ and $\gvec{H} \equiv
[H_1,\ldots,H_N]^T$.

To use a Newton method for solving the non-linear system we need to
compute, at least approximately, the inverse of the Jacobian matrix
$\mvec{J} \equiv \mvec{G}'(\mvec{V})$. The Jacobian itself can be
computed using \eqr{\ref{eqn:g-eqn}} and Eqns.\thinspace
(\ref{eqn:q-eqn}) and (\ref{eqn:f-eqn}) as
\begin{align}
  \mvec{J} =
  \left[
    \begin{array}{cc}
      \mvec{I} & -\pfracb{\mvec{D}}{\mvec{F}} \\
      -\pfracb{\mvec{H}}{\mvec{Q}} & \mvec{I}
    \end{array}
  \right]
\end{align}
Here $\mvec{I}$ is a $N\times N$ unit matrix. Also, it is clear from
Eqns.\thinspace (\ref{eqn:update-form}) and (\ref{eqn:flux-func}),
$\mvec{J}_{12} \equiv -\pfracb{\mvec{D}}{\mvec{F}}$ and $
\mvec{J}_{21} \equiv -\pfracb{\mvec{H}}{\mvec{Q}}$ are bi-diagonal
matrices, the former with a super-diagonal and the latter with a
sub-diagonal. Further, due to the assumed functional form of
$\mvec{S}$, the matrix $\mvec{J}_{12}$ is linear, i.e. does not depend
on either $\mvec{Q}$ or $\mvec{F}$.

Let $\mvec{K} = \mvec{J}^{-1}$, and
\begin{align}
  \mvec{K} = 
  \left[
    \begin{array}{cc}
      \mvec{K}_{11} & \mvec{K}_{12} \\
      \mvec{K}_{21} & \mvec{K}_{22}
    \end{array}
  \right]
\end{align}
Then, using $\mvec{J}\mvec{K} = \mvec{I}$, we can show that
\begin{align}
  \mvec{K}_{11} &= (\mvec{I} - \mvec{J}_{12}\mvec{J}_{21})^{-1} \\
  \mvec{K}_{22} &= (\mvec{I} - \mvec{J}_{21}\mvec{J}_{12})^{-1} \\
  \mvec{K}_{12} &= -\mvec{J}_{12} \mvec{K}_{22} \\
  \mvec{K}_{21} &= -\mvec{J}_{21} \mvec{K}_{11}.
\end{align}
As $\mvec{J}_{12}$ and $\mvec{J}_{21}$ are bidiagonal matrices one
with a sub-diagonal and the other with a super-diagonal, the matrices
$\mvec{I} - \mvec{J}_{12}\mvec{J}_{21}$ and $\mvec{I} -
\mvec{J}_{21}\mvec{J}_{12}$ are tri-diagonal. Hence, to compute the
Newton iteration increments in a naive brute-force algorithm requires
two matrix-matrix multiplications of bi-diagonal matrices, two
tri-diagonal inversions and two matrix-vector products with a
bi-diagonal matrix.

\end{document}