%\documentclass[11pt, reqno]{amsart}
\documentclass[reqno]{amsart}
%% AMS packages and font files
\usepackage{amsmath}
\usepackage{amsfonts}
\usepackage{amsthm}
\usepackage[dvips]{graphicx}
\usepackage[usenames,dvipsnames]{color}
\usepackage{setspace}
\usepackage{fancyhdr}
%\pagestyle{fancyplain}

\DeclareMathAlphabet{\mathpzc}{OT1}{pzc}{m}{it}

%% Set page size properly
%\oddsidemargin  0.0in
%\evensidemargin 0.0in
%\textwidth      6.5in
%\textheight     9.0in
%\headheight     1.0in
%\headsep        0.0in
%\leftmargin     1.0in
%\rightmargin    1.0in
%\topmargin      -.75in

%% Autoscaled figures
\newcommand{\incfig}{\centering\includegraphics}
\setkeys{Gin}{width=0.9\linewidth,keepaspectratio}

%% Commonly used macros
\newcommand{\eqr}[1]{Eq.\thinspace(#1)}
\newcommand{\pfrac}[2]{\frac{\partial #1}{\partial #2}}
\newcommand{\pfraca}[1]{\frac{\partial}{\partial #1}}
\newcommand{\pfracb}[2]{\partial #1/\partial #2}
\newcommand{\mvec}[1]{\mathbf{#1}}
\newcommand{\gvec}[1]{\boldsymbol{#1}}
\newcommand{\script}[1]{\mathpzc{#1}}

\newtheorem{thm}{Theorem}
\newtheorem{lem}{Lemma}

\theoremstyle{definition}
\newtheorem{dfn}{Definition}

\title{Transport equations, finite-volume schemes and coupling
  algorithms}%
\author{Ammar H. Hakim}%
\date{}

\begin{document}
% header text
\lhead{Tech-Note 1003}

\begin{abstract}
  In this note I describe one-dimensional plasma transport equations
  and outline schemes to solve them. I also discuss implicit and
  explicit algorithms to couple multi-domain problems in which each
  domain physics is described by a potentially different set of
  equations.
\end{abstract}

\maketitle

\section{Transport Equations and Finite-Volume Schemes}

Transport in the core of a tokamak is approximately described by a
system of one-dimensional equations for the evolution of density, and
energy for each species in the plasma and a net momentum equation. In
addition, the magnetic field geometry is also evolved. For most
applications only ion densities, electron and ion energies are
evolved.

The key challenge in one-dimensional transport modelling is
determination of accurate flux models. Over the years many complex and
sophisticated flux models of varying physical fidelity have been
developed. Even though the physical realism of core transport
calculations has increased over the years, ultimately core transport
remains an approximate description and recourse to more fundamental
descriptions must be made to fully understand and describe the complex
physics in a tokamak.

The system of transport equations can be written in the form
\begin{align}
  \pfrac{\mvec{Q}}{t} 
  + \frac{1}{V'} \pfraca{x} \left( V' \gvec{\Gamma} \right) = \mvec{S}
  \label{eq:transport}
\end{align}
where $\mvec{Q} = \mvec{Q}(x,t)$ is a $m$-dimensional vector of
conserved quantities (density, momentum, temperature), $\gvec{\Gamma}
= \gvec{\Gamma}(\mvec{Q},\pfracb{\mvec{Q}}{x})$ is a flux function,
$\mvec{S} = \mvec{S}(\mvec{Q},x,t)$ is a source function. The
coordinate $x$ is dimensionless and represents normalized distance
from the magnetic axis, $x=0$, to the separatrix, $x=1$. The equation
is solved on the domain $[0,x_e]$, where $x_e \le 1$. The function
$V(x)$ represents the volume of plasma (in m$^3$) enclosed between
$[0,x]$ and encodes the geometry of the tokamak core region. The
function $V'(x)$ is related to the metric coefficient that appears in
the divergence operator written in generalized coordinate. As $x$ is
dimensionless, the flux function has units of $[\mvec{Q}]/s$, where
$[\mvec{Q}]$ are the units of the conserved quantities.

The point $x=0$ represents a symmetry axis in the tokamak which means
we must set $\gvec{\Gamma}=0$ at $x=0$. At $x=x_e$ one can either
specify the flux, $\gvec{\Gamma}(x_e)=\gvec{\Gamma}_e(t)$, or the
values of the conserved variables $\mvec{Q}(x_e,t)=\mvec{Q}_e(t)$,
where $\gvec{\Gamma}_e(t)$ and $\mvec{Q}_e(t)$ are specified
functions.

To discretize the system of equations consider a finite-volume grid in
which a cell is defined as $C_i \equiv [x_i, x_{i+1}]$, with
$i=1,\ldots,N$. Multiplying the transport equation by $V'(x)$ and
integrating over a cell and using finite-differences for the time
derivative we get the update formula
\begin{align}
  \mvec{Q}_i^{n+1} = \mvec{Q}^n_i - \frac{\Delta t}{V'(x_i) \Delta x_i}
  \left (
    V'(x_{i+1/2}) \mvec{F}_{i+1/2}^{n+\theta} - 
    V'(x_{i-1/2}) \mvec{F}_{i-1/2}^{n+\theta}
  \right)
  +
  \Delta t \mvec{S}_i^{n+\theta} \label{eqn:update-form}
\end{align}
where $0\le\theta\le 1$,
$\mvec{S}_i^{n+\theta}=\mvec{S}(\mvec{Q}_i^{n+\theta},x,t)$ and
\begin{align}
  \mvec{F}_{i-1/2}^{n+\theta} = 
  \mvec{H}(
  \mvec{Q}_{i-1}^{n+\theta},
  \mvec{Q}_i^{n+\theta}
  ) \label{eqn:flux-func}
\end{align}
is a numerical flux function that represents the flux at the cell
edge. Here, $x_{i+1/2} \equiv (x_i+x_{i+1})/2$ and $\Delta x_i \equiv
x_{i+1}-x_i$. Using a Taylor series expansion we can show that the
update formula has $O(\Delta x^2)$ spatial accuracy. Different
temporal accuracy can be achieved by choosing $\theta$: for example,
$\theta=0$ or $\theta=1$ will give $O(\Delta t)$ accuracy while
$\theta=1/2$ will give $O(\Delta t^2)$ accuracy. Values of $\theta>0$
lead to an implicit scheme as $Q^{n+\theta}$ is then not known. The
stability properties of the scheme will also depend on $\theta$. To
compute $Q^{n+\theta}$ one can use, for example, a linear combination
of old and new values, i.e. $(1-\theta)Q^n + \theta Q^{n+1}$. A
predictor-corrector method can also be used, however, this will
generally involve extra flux evaluations that can be expensive in some
transport calculations.

The numerical flux function must satisfy the consistency
condition
\begin{align}
  \lim_{\epsilon\rightarrow 0} \mvec{H}
  \left(
    \mvec{Q}(x-\epsilon,t),
    \mvec{Q}(x+\epsilon,t)
  \right)
  =
  \mvec{\Gamma}(\mvec{Q},\pfracb{\mvec{Q}}{x})
\end{align}
which ensures the convergence of the solution as $\Delta x \rightarrow
0$ and $\Delta t \rightarrow 0$.

The boundary condition at $x=0$ can be applied by simply setting
$\mvec{F}_{1/2}=0$. If the flux is specified at $x=x_e$, then this can
be enforced using $\mvec{F}_{N+1/2}=\mvec{F}_e(t)$. For the second
type of boundary conditions (specified conserved variables) we
introduce a ``ghost-cell'' the same size as the last cell $C_N$. Then,
if we set $\mvec{Q}_{N+1} = 2\mvec{Q}_e(t) - \mvec{Q}_N$, we will
enforce $\mvec{Q}(x_e,t) = \mvec{Q}_e(t)$ to $\textrm{O}(\Delta
x^2)$. Using this in the definition of the numerical flux, we get the
flux at the last edge as
\begin{align}
  \mvec{F}_{N+1/2}^{n+\theta} = \mvec{H}(\mvec{Q}_N^{n+\theta},
  2\mvec{Q}_e(t+\theta\Delta t) - \mvec{Q}_N^{n+\theta}).
\end{align}

\section{Coupling schemes}

To explore \emph{coupling schemes} we can imagine solving
\eqr{\ref{eq:transport}} separately on $x<x_c$ and $x>x_c$, where
$0<x_c<x_e$, and then imposing additional conditions at $x=x_c$. These
\emph{coupling conditions} are continuity of the solution and the
fluxes as shown from the following lemma
\begin{lem}
  \label{lem:couple-conditions}
  Let $0<x_c<x_e$ be some point in $[0,x_e]$. Then $\mvec{Q}(x,t)$ is
  a solution to \eqr{\ref{eq:transport}} in $[0,x_e]$ if it a smooth
  solution in $x<x_c$ and $x>x_c$ and satisfies the conditions
  \begin{align}
    \mvec{Q}_- &= \mvec{Q}_+ \\
    \mvec{\Gamma}_- &= \mvec{\Gamma}_+
  \end{align}
  where $\mvec{Q}_\pm \equiv \lim_{\epsilon\rightarrow
    0}\mvec{Q}(x_c\pm\epsilon,t)$ and $\gvec{\Gamma}_\pm \equiv
  \gvec{\Gamma}(\mvec{Q}_\pm,\pfracb{\mvec{Q}_\pm}{x})$.
\end{lem}

\end{document}

\section{Nonlinear diffusion with temperature-gradient dependent
  diffusivity}

A scalar nonlinear equation relevant to turbulent transport in plasmas
is the nonlinear heat diffusion equation in which diffusivity depends
on the temperature gradient. In this case the conserved variable is
the thermal energy 
\begin{align}
  Q = \frac{3}{2} n(x)T(x,t)
\end{align}
where $n(x)$ is the number density of the plasma (which is spatially
dependent but fixed in time) and $T(x,t)$ is the plasma
temperature. In this case the flux is defined by
\begin{align}
  \Gamma = -\chi n T'
\end{align}
here primes indicate derivatives with respect to $x$ and where $\chi$
is the diffusivity given by
\begin{align}
  \chi &= k(|T'| - T'_c)^\alpha + \chi_0 &\quad |T'| > T'_c \\
  &= \chi_0 &\quad |T'| \leq T'_c,
\end{align}
where $\alpha$, $k$ and $\chi_0$ are constants and $T_c$ is some
critical gradient.


