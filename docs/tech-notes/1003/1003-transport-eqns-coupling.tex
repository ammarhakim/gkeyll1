\documentclass[11pt]{amsart}
%% AMS packages and font files
\usepackage{amsmath}
\usepackage{amsfonts}
\usepackage[dvips]{graphicx}
\usepackage[usenames,dvipsnames]{color}
\usepackage{setspace}
\usepackage{fancyhdr}
%\pagestyle{fancyplain}

\DeclareMathAlphabet{\mathpzc}{OT1}{pzc}{m}{it}

%% Set page size properly
\oddsidemargin  0.0in
\evensidemargin 0.0in
\textwidth      6.5in
\textheight     9.0in
\headheight     1.0in
\headsep        0.0in
\leftmargin     1.0in
\rightmargin    1.0in
\topmargin      -.75in

%% Autoscaled figures
\newcommand{\incfig}{\centering\includegraphics}
\setkeys{Gin}{width=0.9\linewidth,keepaspectratio}

%% Commonly used macros
\newcommand{\eqr}[1]{Eq.\thinspace(#1)}
\newcommand{\pfrac}[2]{\frac{\partial #1}{\partial #2}}
\newcommand{\pfraca}[1]{\frac{\partial}{\partial #1}}
\newcommand{\pfracb}[2]{\partial #1/\partial #2}
\newcommand{\mvec}[1]{\mathbf{#1}}
\newcommand{\gvec}[1]{\boldsymbol{#1}}
\newcommand{\script}[1]{\mathpzc{#1}}

\title{Transport equations, finite-volume schemes and coupling
  algorithms}%
\author{Ammar H. Hakim}%
\date{}

\begin{document}
% header text
\lhead{Tech-Note 1003}
\maketitle

\section{Transport Equations and Finite-Volume Schemes}

Transport in the core of a tokamak is approximately described by a
system of one-dimensional equations for the evolution of density,
momentum and energy for each species in the plasma. In addition, the
magnetic field geometry is also evolved. The key challenge in
one-dimensional transport modelling is determination of accurate flux
models. Over the years many complex and sophisticated flux models, of
varying physical fidelity, have been developed. Even though the
physical realism of core transport calculations has increased over the
years, ultimately core transport remains an approximate description
and recourse to more fundamental descriptions must be made to fully
understand and describe the complex physics in a tokomak.

The system of transport equations can be written in the form
\begin{align}
  \pfrac{\mvec{Q}}{t} 
  + \frac{1}{V'} \pfraca{x} \left( V' \gvec{\Gamma} \right) = \mvec{S}
\end{align}
where $\mvec{Q} = \mvec{Q}(x,t)$ is a $m$-dimensional vector of
conserved quantities (density, momentum, temperature),
$\gvec{\Gamma}(x) = \gvec{\Gamma}(\mvec{Q},\pfracb{\mvec{Q}}{x})$ is a
flux function, $\mvec{S}(x,t) = \mvec{S}(\mvec{Q},x,t)$ is a source
function. The coordinate $x$ is dimensionless and represents
normalized distance from the magnetic axis, $x=0$, to the separatrix,
$x=1$. The function $V(x)$ represents the volume of plasma (in m$^3$)
enclosed between $[0,x]$ and encodes the geometry of the tokamak core
region. For realistic tokamak simulations the geometry changes
continuously and $V(x)$ is a function of time. For most applications
only plasma density, electron and ion energies are evolved.

To discretize the system of equations consider a finite-volume grid in
which a cell is defined as $C_i \equiv [x_i, x_{i+1}]$, with
$i=1,\ldots,N$. Integrating the transport equation over a
cell and using a first-order difference for the time derivative we get
the update formula
\begin{align}
  \mvec{Q}_i^{n+1} = \mvec{Q}^n_i - \frac{\Delta t}{V'(x_i) \Delta x_i}
  \left (
    V'(x_{i+1/2}) \mvec{F}_{i+1/2} - 
    V'(x_{i-1/2}) \mvec{F}_{i-1/2}
  \right)
  +
  \Delta t \mvec{S}_i \label{eqn:update-form}
\end{align}
where
\begin{align}
  \mvec{F}_{i-1/2} = 
  \mvec{H}(
  \mvec{Q}_{i-1},
  \mvec{Q}_i
  ) \label{eqn:flux-func}
\end{align}
is a numerical flux function that represents the flux at the cell
edge. Here, $x_{i+1/2} \equiv (x_i+x_{i+1})/2$ and $\Delta x_i \equiv
(x_{i+1}-x_i)$. The time index on the edge fluxes and sources are
deliberately omitted: choosing to evaluate these at various points in
the interval $[t_n,t_{n+1}]$ will give scheme with different
implicitness. The numerical flux function must satisfy
\begin{align}
  \lim_{\epsilon\rightarrow 0} \mvec{H}
  \left(
    \mvec{Q}(x-\epsilon,t),
    \mvec{Q}(x+\epsilon,t)
  \right)
  =
  \mvec{\Gamma}(\mvec{Q},\pfracb{\mvec{Q}}{x})
\end{align}
which ensures the consistency of the numerical scheme.

\end{document}

Let conserved quantities be located at cell-centers $\mvec{Q}_i^n =
\mvec{Q}(x_{i+1/2},t_n)$, fluxes located at cell-edges
$\gvec{\Gamma}_{i-1/2} = \gvec{\Gamma}(x_{i-1/2})$ and sources located
at cell-centers $\mvec{S}_i = \mvec{S}(\mvec{Q}_i, x_i,t_{n})$.
