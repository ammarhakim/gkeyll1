\documentclass[11pt, reqno]{amsart}
%% AMS packages and font files
\usepackage{amsmath}
\usepackage{amsfonts}
\usepackage{amsthm}
\usepackage[dvips]{graphicx}
\usepackage[usenames,dvipsnames]{color}
\usepackage{setspace}
\usepackage{fancyhdr}
% \pagestyle{fancyplain}

\DeclareMathAlphabet{\mathpzc}{OT1}{pzc}{m}{it}

%% Set page size properly
% \oddsidemargin  0.0in
% \evensidemargin 0.0in
% \textwidth      6.5in
% \textheight     9.0in
% \leftmargin     1.0in
% \rightmargin    1.0in

%% Autoscaled figures
\newcommand{\incfig}{\centering\includegraphics}
\setkeys{Gin}{width=0.9\linewidth,keepaspectratio}

%% Commonly used macros
\newcommand{\eqr}[1]{Eq.\thinspace(#1)}
\newcommand{\pfrac}[2]{\frac{\partial #1}{\partial #2}}
\newcommand{\pfracc}[2]{\frac{\partial^2 #1}{\partial #2^2}}
\newcommand{\pfraca}[1]{\frac{\partial}{\partial #1}}
\newcommand{\pfracb}[2]{\partial #1/\partial #2}
\newcommand{\pfracbb}[2]{\partial^2 #1/\partial #2^2}
\newcommand{\spfrac}[2]{{\partial_{#1}} {#2}}
\newcommand{\mvec}[1]{\mathbf{#1}}
\newcommand{\gvec}[1]{\boldsymbol{#1}}
\newcommand{\script}[1]{\mathpzc{#1}}

\newtheorem{thm}{Theorem}
\newtheorem{lem}{Lemma}

\theoremstyle{definition}
\newtheorem{dfn}{Definition}

\title[Maxwell Eigensystem]{The eigensystem of the Maxwell equations
  with extension to perfectly hyperbolic Maxwell equations.}%
\author{Ammar H. Hakim}%
\date{}

\begin{document}
% header text
\lhead{Tech-Note 1012}
\maketitle

\section{Eigensystem of Maxwell equations}

In this document I list the eigensystem of the Maxwell
equations. Maxwell's equations consist of the curl equations
\begin{align}
  \frac{\partial \mvec{B}}{\partial t} + \nabla\times\mvec{E} &= 0 \\
  \epsilon_0\mu_0\frac{\partial \mvec{E}}{\partial t} -
  \nabla\times\mvec{B} &= -\mu_0\mvec{J}
\end{align}
along with the divergence relations
\begin{align}
  \nabla\cdot\mvec{E} &= \frac{\varrho_c}{\epsilon_0} \label{eq:divE} \\
  \nabla\cdot\mvec{B} &= 0. \label{eq:divB}
\end{align}
Here, $\mvec{E}$ is the electric field, $\mvec{B}$ is the magnetic
flux density, $\epsilon_0$, $\mu_0$ are permittivity and permeability
of free space, and $\mvec{J}$ and $\varrho$ are specified currents and
charges respectively. The speed of light is determined from
$c=1/(\mu_0\epsilon_0)^{1/2}$.

These are linear equations and hence the eigensytem is independent of
the value of the electromagnetic fields. In 1D Maxwell equations can
be written as, ignoring sources,
\begin{align}
  \pfraca{t}
  \left[
    \begin{matrix}
      E_x \\
      E_y \\
      E_z \\
      B_x \\
      B_y \\
      B_z
    \end{matrix}
  \right]
  +
  \pfraca{x}
  \left[
    \begin{matrix}
      0 \\
      c^2B_z \\
      -c^2B_y \\
      0 \\
      -E_z \\
      E_y
    \end{matrix}
  \right]
  =
  0.
\end{align}
The eigenvalues of this system are $\{0,0,c,c,-c,-c\}$. The right
eigenvectors of the flux Jacobian are given by the columns of the
matrix
\begin{align}
  R
  =
  \left[
    \begin{matrix}
      0 & 1 & 0 & 0 & 0 & 0 \\
      0 & 0 & c & 0 & -c & 0 \\
      0 & 0 & 0 & -c & 0 & c \\
      1 & 0 & 0 & 0 & 0 & 0 \\
      0 & 0 & 0 & 1 & 0 & 1 \\
      0 & 0 & 1 & 0 & 1 & 0
    \end{matrix}
  \right].
  \label{eq:rev}
\end{align}
The left eigenvectors are the rows of the matrix
\begin{align}
  L
  =
  \left[
    \begin{matrix}
      0 & 0 & 0 & 1 & 0 & 0 \\
      1 & 0 & 0 & 0 & 0 & 0 \\
      0 & \frac{1}{2c} & 0 & 0 & 0 & \frac{1}{2} \\
      0 & 0 & -\frac{1}{2c} & 0 & \frac{1}{2} & 0 \\
      0 & -\frac{1}{2c} & 0 & 0 & 0 & \frac{1}{2} \\
      0 & 0 & \frac{1}{2c} & 0 & \frac{1}{2} & 0
    \end{matrix}
  \right].
  \label{eq:lev}
\end{align}

\section{Eigensystem of Perfectly Hyperbolic Maxwell equations}

The perfectly hyperbolic Maxwell equations are a modification of the
Maxwell equations that take into account the divergence relations. The
modified equations explicitly ``clean'' divergence errors and are a
hyperbolic generalization of the Hodge project method commonly used in
electromagnetism to correct for charge conservation
errors\cite{munz_2000, munz_2000b, munz_2000c}.

These equations are written as
\eqr{\ref{eq:divE}} and \eqr{\ref{eq:divB}}.
\begin{align}
  \frac{\partial \mvec{B}}{\partial t} + \nabla\times\mvec{E} +
  \gamma \nabla\psi
  &= 0 \\
  \epsilon_0\mu_0\frac{\partial \mvec{E}}{\partial t} -
  \nabla\times\mvec{B} +     \chi \nabla \phi
  &= -\mu_0\mvec{J} \\
  \frac{1}{\chi}\pfrac{\phi}{t} + \nabla\cdot\mvec{E} 
  &= \frac{\varrho_c}{\epsilon_0} \\
  \frac{\epsilon_0\mu_0}{\gamma}\pfrac{\psi}{t} + \nabla\cdot\mvec{B} 
  &= 0.
\end{align}
Here, $\psi$ and $\psi$ are correction potentials for the electric and
magnetic field respectively and $\chi$ and $\gamma$ are dimensionless
factors that control the speed at which the errors are propagated.

In 1D these equations can be written as, ignoring sources,
\begin{align}
  \pfraca{t}
  \left[
    \begin{matrix}
      E_x \\
      E_y \\
      E_z \\
      B_x \\
      B_y \\
      B_z \\
      \phi \\
      \psi
    \end{matrix}
  \right]
  +
  \pfraca{x}
  \left[
    \begin{matrix}
      \chi c^2 \phi \\
      c^2B_z \\
      -c^2B_y \\
      \gamma \psi \\
      -E_z \\
      E_y \\
      \chi E_x \\
      \gamma c^2B_x
    \end{matrix}
  \right]
  =
  0.
\end{align}
The eigenvalues of this system are $\{-c\gamma, c\gamma, -c\chi,
c\chi, c, c, -c, -c\}$. The right eigenvectors of the flux Jacobian
are given by the columns of the matrix
\begin{align}
  R
  =
  \left[
    \begin{matrix}
      0  & 0 & 1 & 1 & 0 &  0 &  0 & 0 \\
      0  & 0 & 0 & 0 & c &  0 & -c & 0 \\
      0  & 0 & 0 & 0 & 0 & -c &  0 & c \\
      1  & 1 & 0 & 0 & 0 &  0 &  0 & 0 \\
      0  & 0 & 0 & 0 & 0 &  1 &  0 & 1 \\
      0  & 0 & 0 & 0 & 1 &  0 &  1 & 0 \\
      0  & 0 & -\frac{1}{c} & \frac{1}{c} & 0 &  0 &  0 & 0 \\
     -c  & c & 0 & 0 & 0 &  0 &  0 & 0
    \end{matrix}
  \right].
  \label{eq:ph_rev}
\end{align}
The left eigenvectors are the rows of the matrix
\begin{align}
  L
  =
  \left[
    \begin{matrix}
      0 & 0 & 0 & \frac{1}{2} & 0 & 0 & 0 & -\frac{1}{2c} \\
      0 & 0 & 0 & \frac{1}{2} & 0 & 0 & 0 & \frac{1}{2c} \\
      \frac{1}{2} & 0 & 0 & 0 & 0 & 0 & -\frac{c}{2} & 0 \\
      \frac{1}{2} & 0 & 0 & 0 & 0 & 0 & \frac{c}{2} & 0 \\
      0 & \frac{1}{2c} & 0 & 0 & 0 & \frac{1}{2} & 0 & 0 \\
      0 & 0 & -\frac{1}{2c} & 0 & \frac{1}{2} & 0 & 0 & 0 \\
      0 & -\frac{1}{2c} & 0 & 0 & 0 & \frac{1}{2} & 0 & 0 \\
      0 & 0 & \frac{1}{2c} & 0 & \frac{1}{2} & 0 & 0 & 0
    \end{matrix}
  \right].
  \label{eq:ph_lev}
\end{align}

\bibliography{../common/lucee}
\bibliographystyle{plain}

\end{document}