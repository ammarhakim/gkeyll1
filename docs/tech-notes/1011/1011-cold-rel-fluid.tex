\documentclass[11pt, reqno]{amsart}
%% AMS packages and font files
\usepackage{amsmath}
\usepackage{amsfonts}
\usepackage[dvips]{graphicx}
\usepackage[usenames,dvipsnames]{color}
\usepackage{setspace}
\usepackage{fancyhdr}
% \pagestyle{fancyplain}

\DeclareMathAlphabet{\mathpzc}{OT1}{pzc}{m}{it}

%% Set page size properly
\oddsidemargin  0.0in
\evensidemargin 0.0in
\textwidth      6.5in
\textheight     9.0in
\headheight     1.0in
%\headsep        0.0in
\leftmargin      1.0in
\rightmargin      1.0in
\topmargin      -.75in

%% Autoscaled figures
\newcommand{\incfig}{\centering\includegraphics}
\setkeys{Gin}{width=0.9\linewidth,keepaspectratio}

%% Commonly used macros
\newcommand{\eqr}[1]{Eq.\thinspace(#1)}
\newcommand{\pfrac}[2]{\frac{\partial #1}{\partial #2}}
\newcommand{\pfracc}[2]{\frac{\partial^2 #1}{\partial #2^2}}
\newcommand{\pfraca}[1]{\frac{\partial}{\partial #1}}
\newcommand{\pfracb}[2]{\partial #1/\partial #2}
\newcommand{\pfracbb}[2]{\partial^2 #1/\partial #2^2}
\newcommand{\spfrac}[2]{{\partial_{#1}} {#2}}
\newcommand{\mvec}[1]{\mathbf{#1}}
\newcommand{\gvec}[1]{\boldsymbol{#1}}
\newcommand{\script}[1]{\mathpzc{#1}}

\newtheorem{thm}{Theorem}
\newtheorem{lem}{Lemma}

\theoremstyle{definition}
\newtheorem{dfn}{Definition}


\title{The charged cold relativistic fluid equations}%
\author{Ammar H. Hakim}%
\date{}

\begin{document}
% header text
\lhead{Tech-Note 1011}

\maketitle

In this note I list the charged cold relativistic fluid (CRF)
equations and describe a solver for use in simulations of
ultra-intense laser propagation in an under-dense plasma. The cold
fluid equations describe the evolution of the electrons while the ion
density is held fixed. The electromagnetic fields are evolved using
Maxwell equations of electromagnetism.

\section{Cold neutral relativistic fluid equations}

In this section the mathematical structure of the cold neutral
relativistic fluid equations is examined. The coupling to the
electromagnetic fields is through electric and magnetic fields and is
ignored. The conservative form of CRF are
\begin{align}
  \pfrac{n}{t} + \nabla\cdot(n\mvec{u}) &= 0 \\
  \pfraca{t}(n\mvec{p}) + \nabla\cdot (n\mvec{p}\mvec{u}) &= 0
\end{align}
where $n(\mvec{x},t)$ is the fluid number density,
$\mvec{u}(\mvec{x},t)$ is the velocity and $\mvec{p}(\mvec{x},t)$ is
the momentum. The momentum and velocity are related by the
relativistic identity
\begin{align}
  \mvec{p} = m\gamma \mvec{v}
\end{align}
where $m$ is the species mass and where
\begin{align}
  \gamma &= \left(1+\frac{|\mvec{p}|^2}{m^2c^2}\right)^{1/2} \\
  &= \left(1-|\mvec{v}|^2/c^2\right)^{-1/2}.
\end{align}

\end{document}