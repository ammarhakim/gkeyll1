\documentclass[11pt, reqno]{amsart}
%% AMS packages and font files
\usepackage{amsmath}
\usepackage{amsfonts}
\usepackage[dvips]{graphicx}
\usepackage[usenames,dvipsnames]{color}
\usepackage{setspace}
\usepackage{fancyhdr}
% \pagestyle{fancyplain}

\DeclareMathAlphabet{\mathpzc}{OT1}{pzc}{m}{it}

%% Set page size properly
\oddsidemargin  0.0in
\evensidemargin 0.0in
\textwidth      6.5in
\textheight     9.0in
\headheight     1.0in
%\headsep        0.0in
\leftmargin      1.0in
\rightmargin      1.0in
\topmargin      -.75in

%% Autoscaled figures
\newcommand{\incfig}{\centering\includegraphics}
\setkeys{Gin}{width=0.9\linewidth,keepaspectratio}

%% Commonly used macros
\newcommand{\eqr}[1]{Eq.\thinspace(#1)}
\newcommand{\pfrac}[2]{\frac{\partial #1}{\partial #2}}
\newcommand{\pfracc}[2]{\frac{\partial^2 #1}{\partial #2^2}}
\newcommand{\pfraca}[1]{\frac{\partial}{\partial #1}}
\newcommand{\pfracb}[2]{\partial #1/\partial #2}
\newcommand{\pfracbb}[2]{\partial^2 #1/\partial #2^2}
\newcommand{\spfrac}[2]{{\partial_{#1}} {#2}}
\newcommand{\mvec}[1]{\mathbf{#1}}
\newcommand{\gvec}[1]{\boldsymbol{#1}}
\newcommand{\script}[1]{\mathpzc{#1}}

\newtheorem{thm}{Theorem}
\newtheorem{lem}{Lemma}

\theoremstyle{definition}
\newtheorem{dfn}{Definition}


\title{The MUSCL-Hancock scheme for solution of hyperbolic equations}%
\author{Ammar H. Hakim}%
\date{}

\begin{document}
% header text
\lhead{Tech-Note 1010}

\maketitle

In this document I outline the MUSCL-Hancock scheme for the solution
of 1D hyperbolic partial differential equations. This scheme is a
predictor-corrector scheme and is second order accurate in both space
and time. We start from the system of hyperbolic equations
\begin{align}
  \pfrac{Q}{t} + \pfrac{F}{x} = 0 \label{eq:cons_form}
\end{align}
where $Q(x,t)$ is a vector of $m$ conserved quantities and $F=F(Q)$
are fluxes. In the following we denote the flux Jacobian as
$A\equiv\pfracb{F}{Q}$ and the eigenvalues as $\lambda^p$ and the
right- and left-eigenvectors as $r^p$ (column vector) and $l^p$ (row
vector), for $p=1,\ldots,m$, respectively.

We will also assume (though this is not required) that the system can
be put into the non-conservative (``primitive'') quasi-linear form
\begin{align}
  \pfrac{V}{t} + A_p\pfrac{V}{x} = 0 \label{eq:prim_form}
\end{align}
where $V(x,t)$ are a vector of $m$ primitive quantities and $A_p(V)$
is a $m\times m$ matrix. Note that any invertible transform
$Q=\varphi(V)$ will transform \eqr{\ref{eq:cons_form}} into
\eqr{\ref{eq:prim_form}}.

\section{The basic algorithm}

The essential idea of the MUSCL-Hancock scheme is to use cell averages
to predict the values of the conserved (or primitive) quantities at
cell edges at $t+\Delta t/2$ and then use these predicted values to
update the solution to $t+\Delta t$. The steps in the algorithm are as
follows.
\begin{enumerate}

\item Given cell averages reconstruct a (possibly limited) linear
  representation of the variables inside each cell. This can be done
  for either the conserved variables or the primitive
  variables. Hence, in each cell we represent the solution as
  \begin{align}
    W(x,t) = W_i + \frac{x-x_i}{\Delta x}\delta W_i \label{eq:lin_recon}
  \end{align}
  for $x_{i-1/2}<x<x_{i+1/2}$ and where $x_i \equiv
  (x_{i+1/2}+x_{i-1/2})/2$, $\Delta x \equiv x_{i+1/2}-x_{i-1/2}$ and
  $\delta W_i$ are the reconstructed \emph{slopes}. In
  \eqr{\ref{eq:lin_recon}} $W(x,t)$ stands for the variables we are
  reconstructing (either primitive or conserved variables). To
  determine the slopes we can use an averaging procedure
  \begin{align}
    \delta W_i = \mathrm{ave}(W_i-W_{i-1}, W_{i+1}-W_i) \label{eqn:slope_recon}
  \end{align}
  where $\mathrm{ave}(a, b)$ is a suitable ``averaging'' function,
  applied to each component of the vector. Note that using the
  standard average $\mathrm{ave}(a, b) = (a+b)/2$ leads to a
  central-difference computed slope, while $\mathrm{ave}(a, b) = 0$
  leads to a zero slope or a first-order representation in each
  cell. Other forms of the average function can be used to avoid
  spurious oscillations around discontinuities and prevent the
  formation of unphysical states. See the next section for more
  details on the reconstruction and averaging steps.
  
\item Use the slopes to predict the solution at half time-step,
  $\Delta t/2$. If the primitive variable slopes have been determined
  then use the update formula
  \begin{align}
    \tilde{V}_j = V_j -\frac{\Delta t}{2 \Delta x} A_p(V_i) \delta V_i
  \end{align}
  If the conserved variable slopes have been determined then use the
  update formula
  \begin{align}
    \tilde{Q}_j = Q_j -\frac{\Delta t}{2 \Delta x} A(Q_i) \delta Q_i
  \end{align}
  In these formulas $\tilde{V}_j$ and $\tilde{Q}_j$ denote the
  predicted values in cell $\script{C}_i$.

\item Use the predicted solution to compute the predicted values at
  cell edges. As the solution is assumed to be linear, the edge values
  are
  \begin{align}
    W_{i-1/2}^+ &= \tilde{W}_i - \delta W_i/2 \\
    W_{i+1/2}^- &= \tilde{W}_i + \delta W_i/2
  \end{align}
  Note that we are using the predicted solution at $t+\Delta t/2$ but
  the slopes at $t$ to compute the edge values. This gives the edge
  values at $t+\Delta t/2$ to $O(\Delta t^2)$.

\item Use the edge values in a Reimann solver (a numerical flux) to
  update the conserved variables to time $t$
  \begin{align}
    Q^{n+1}_i = Q_i^n - \frac{\Delta t}{\Delta x}(F_{i+1/2}-F_{i+1/2}) \label{eqn:corrector_step}
  \end{align}
  where $F_{i\pm 1/2}$ are the numerical fluxes computed from the
  predicted edge values:
  \begin{align}
    F_{i-1/2} \equiv F(W_{i-1/2}^-, W_{i-1/2}^+).
  \end{align}
  See the last section for details on numerical fluxes that can be
  used in \eqr{\ref{eqn:corrector_step}}.

\end{enumerate}

\section{Reconstruction and limiting}

It is simplest to reconstruct each of the conserved variables or the
primitive variables directly. This procedure is called
\emph{component} reconstruction and limiting. However, a better
approach that results in smoother solutions is to limit the
\emph{characteristic} variables instead. In this case the limiting is
done after projecting the differences on left eigenvectors of the flux
Jacobian. Let $L(Q)$ be the matrix of left eigenvectors arranged as
rows and let $R(Q)$ be the matrix of right eigenvectors arranged as
columns. Note that $L=R^{-1}$. Then the reconstruction becomes,
instead of \eqr{\ref{eqn:slope_recon}},
\begin{align}
  \delta W_i = R(Q_i)\ \mathrm{ave}(\Delta^i_{i-1}, \Delta^i_i)
\end{align}
where $\Delta^j_i = L(Q_j)(W_{i+1}-W_i)$. If the averaging function is
non-linear then even for a linear system of the equations the
characteristic limiting and component limiting do not coincide.

There are several possible averaging function one can use (besides the
zero and simple-averages). For example, the following choices are all
designed to avoid unphysical oscillations around discontinuities
\begin{itemize}
  \item Minmod limiting
    \begin{align}
      \mathrm{ave}(a,b) = 
      \begin{cases}
        \mathrm{minmod}((a+b)/2, 2a, 2b)& \text{if $ab>0$} \\
        0& \text{if $ab\le 0$}
      \end{cases}
    \end{align}

  \item Supebee limiting
    \begin{align}
      \mathrm{ave}(a,b) = 
      \begin{cases}
        \mathrm{minmod}\left(
          \mathrm{maxmod}(a,b), \mathrm{minmod}(2a,2b)
          \right)& \text{if $ab>0$} \\
        0& \text{if $ab\le 0$}
      \end{cases}
    \end{align}

  \item Epsilon limiting
    \begin{align}
      \mathrm{ave}(a,b) = \frac{(b^2+\epsilon^2)a + (a^2+\epsilon^2)b}{a^2+b^2+2\epsilon^2}
    \end{align}
    where $\epsilon^2 \sim \Delta x^3$ is a parameter.
\end{itemize}
In the above expressions the $\mathrm{mimod}(a_0,a_1,\ldots)$ function
is defined as
\begin{align}
  \mathrm{minmod}(a_0,a_1,\ldots) =
  \begin{cases}
    \min(a_0,a_1,\ldots)& \text{if $a_i>0$, for all $i=0,1,\ldots$} \\
    \max(a_0,a_1,\ldots)& \text{if $a_i<0$, for all $i=0,1,\ldots$} \\
    0& \text{otherwise}
  \end{cases}
\end{align}

None of the above reconstructions (except the zero-average) ensures
that invariant domains are preserved. Another way to put it is that
unless something special is done the scheme may not be positivity
preserving. For example, while solving the Euler equations the
predicted edge values of density and pressure may become negative,
leading to unphysical states. A simple but crude way to fix this is to
set slopes of \emph{all} quantities in a cell to zero if any of the
values at either cell edge becomes negative. More nuanced methods can
also be develop by self-consistently\footnote{What this means is that
  if the slopes of density and pressure are adjusted, the complete
  predicted solution (and hence the edge values) must be recomputed
  with the new slopes.} adjusting the slopes just enough to ensure
invariant domains are preserved.

\section{Numerical fluxes}

A wide variety of numerical fluxes can be used to compute the edge
fluxes needed in \eqr{\ref{eqn:corrector_step}}. It is important to
use a numerical flux that preserves positivity. This combined with a
positivity preserving reconstruction will ensure, under a suitable CFL
condition, the positivity of the complete scheme.

The simplest numerical flux to use is the local Lax flux (also called
the Rusanov flux). This is given by
\begin{align}
  F(Q^-,Q^+) = \frac{F(Q^-) + F(Q^+)}{2} - c\frac{Q^+- Q^-}{2}
\end{align}
Here $c>0$ is a parameter given by
\begin{align}
  c = \sup_{Q=Q^-,Q^+} \sup_p | \lambda^p |. \label{eqn:lax_c}
\end{align}
In other words, the parameter $c$ is the maximum of the absolute
eigenvalues computed from the left and right state. Though diffusive,
the Lax flux is the simplest in the sense that it requires the minimum
amount of information about the equation system being solved: all one
needs (besides the flux function) is an \emph{estimate} of the maximum
eigenvalue. Note that any $c$ greater than the one computed by
\eqr{\ref{eqn:lax_c}} can be used. More complex numerical flux
functions that incorporate more information about the equation system
can also be used. These flux functions can reduce diffusion as the
cost of greater complexity.

\end{document}

In other words, for each cell $\script{C}_i$ compute $L(Q_i)$, project
the differences on this matrix and then use the averaging
procedure. Finally, compute $R(Q_i)$ and reconstruct the slopes.

In this note I will use the 1D Euler equations as an example. The
conservative form of these equations are
\begin{align}
  \pfraca{t}
  \left[
    \begin{matrix}
      \rho \\
      \rho u \\
      E
    \end{matrix}
  \right]
  +
  \pfraca{x}
  \left[
    \begin{matrix}
      \rho u \\
      \rho u^2 + p \\
      (E+p)u
    \end{matrix}
  \right]
  =
  0
\end{align}
where
\begin{align}
  E = \frac{p}{\gamma-1} + \frac{1}{2}\rho u^2
\end{align}
and $\gamma$ is the gas adiabatic index. The non-conservative form of
these equations are
\begin{align}
  &\pfrac{\rho}{t} + \rho\pfrac{u}{x} + u \pfrac{\rho}{x} = 0 \\
  &\pfrac{u}{t} + u\pfrac{u}{x} + \frac{1}{\rho} \pfrac{p}{x} = 0 \\
  &\pfrac{p}{t} + u\pfrac{p}{x} + \gamma p \pfrac{u}{x} = 0 \\
\end{align}

