\documentclass[reqno]{amsart}
%% AMS packages and font files
\usepackage{amsmath}
\usepackage{amsfonts}
\usepackage{amsthm}
\usepackage[dvips]{graphicx}
\usepackage[usenames,dvipsnames]{color}
\usepackage{setspace}
\usepackage{fancyhdr}
\usepackage{url}
% \pagestyle{fancyplain}

\DeclareMathAlphabet{\mathpzc}{OT1}{pzc}{m}{it}

%% Set page size properly
% \oddsidemargin  0.0in
% \evensidemargin 0.0in
% \textwidth      6.5in
% \textheight     9.0in
% \leftmargin     1.0in
% \rightmargin    1.0in

%% Autoscaled figures
\newcommand{\incfig}{\centering\includegraphics}
\setkeys{Gin}{width=0.9\linewidth,keepaspectratio}

%% Commonly used macros
\newcommand{\eqr}[1]{Eq.\thinspace(#1)}
\newcommand{\pfrac}[2]{\frac{\partial #1}{\partial #2}}
\newcommand{\pfracc}[2]{\frac{\partial^2 #1}{\partial #2^2}}
\newcommand{\pfraca}[1]{\frac{\partial}{\partial #1}}
\newcommand{\pfracb}[2]{\partial #1/\partial #2}
\newcommand{\pfracbb}[2]{\partial^2 #1/\partial #2^2}
\newcommand{\spfrac}[2]{{\partial_{#1}} {#2}}
\newcommand{\mvec}[1]{\mathbf{#1}}
\newcommand{\gvec}[1]{\boldsymbol{#1}}
\newcommand{\script}[1]{\mathpzc{#1}}
\newcommand{\eep}{\mvec{e}_\phi}
\newcommand{\eer}{\mvec{e}_r}
\newcommand{\eez}{\mvec{e}_z}

\newtheorem{thm}{Theorem}
\newtheorem{lem}{Lemma}
\newtheorem{prop}{Proposition}

\theoremstyle{definition}
\newtheorem{dfn}{Definition}

\title[Ideal MHD Equilibrium]{Formulation of ideal MHD equilibrum problems}%
\author{Ammar H. Hakim}%
\date{}

\begin{document}
% header text
\lhead{Tech-Note 1008}
\maketitle

\appendix
\section{Useful Identities}

A set of useful vector identities can be found in the NRL Plasma
Formulary\footnote{See \url{http://wwwppd.nrl.navy.mil/nrlformulary/}
  for a PDF version of the Formulary and for ordering free printed
  copies.}. In this Appendix additional identities useful in the
derivation of axisymmetric static and flowing equilibria are listed.

Let $\mvec{a}$ be an axisymmetric vector field satisfying
$\nabla\cdot\mvec{a} = 0$. Then, in cylindrical coordinates, it can be
written as
\begin{align}
  \mvec{a} = a_\phi \eep + \frac{1}{r}\nabla\psi \times \eep,
\end{align}
where $\eep$, $\eer$ and $\eez$ are unit vectors and $\psi =
\psi(r,z)$ is an arbitrary function. In component form
\begin{align}
  a_r = -\frac{1}{r} \pfrac{\psi}{z}, \quad 
  a_z = \frac{1}{r} \pfrac{\psi}{r}.
\end{align}
The curl of $\mvec{a}$ is given by
\begin{align}
  \nabla\times\mvec{a} = -\frac{\triangle^*\psi}{r}\eep
  + \frac{1}{r} \nabla(ra_\phi)\times\eep, \label{eq:curla}
\end{align}
where $\triangle^*$ is the \emph{Grad-Shafranov} operator defined by
\begin{align}
  \triangle^*\psi \equiv \frac{\partial^2 \psi}{\partial z^2}
  + r \frac{\partial}{\partial r}\left(\frac{1}{r} \pfrac{\psi}{r}\right).
\end{align}

If $\mvec{a}$ is a axisymmetric vector field and $f(r,z)$ is a
scalar function and $\nabla \cdot (f\mvec{a}) = 0$, then
\begin{align}
  \mvec{a} = a_\phi \eep + \frac{1}{rf}\nabla\psi \times \eep.
\end{align}
The curl of $\mvec{a}$ is given by
\begin{align}
  \nabla\times\mvec{a} = -\frac{\triangle^*_f\psi}{r}\eep
  + \frac{1}{r} \nabla(ra_\phi)\times\eep, \label{eq:curlaf}
\end{align}
where $\triangle^*_f$ is a \emph{f-weighted Grad-Shafranov} operator
defined by
\begin{align}
  \triangle^*_f\psi \equiv 
  \frac{\partial}{\partial z}\left(\frac{1}{f} \pfrac{\psi}{z}\right)
  + r \frac{\partial}{\partial r}\left(\frac{1}{rf} \pfrac{\psi}{r}\right).
\end{align}
Let $\mvec{a}=a_\phi\eep + \nabla\psi_a \times \eep/r$ and $\mvec{b}=b_\phi\eep +
\nabla\psi_b \times \eep/r$. Then 
\begin{align}
  \mvec{a}\times\mvec{b} &=
  \frac{a_\phi}{r}\nabla\psi_b
  -
  \frac{b_\phi}{r}\nabla\psi_a
  -\frac{1}{r^2}(\nabla\psi_a \times \eep \cdot \nabla\psi_b)\eep \\
  &= \frac{a_\phi}{r}\nabla\psi_b
  -
  \frac{b_\phi}{r}\nabla\psi_a
  +\frac{1}{r^2}\nabla\psi_a \times \nabla\psi_b.
\end{align}
Let $\mvec{a}=a_\phi\eep + \nabla\psi \times \eep/r$ and $f=f(r,z)$ is
a scalar field. Then
\begin{align}
  \mvec{a}\cdot\nabla f = \frac{1}{r} \nabla f \times \nabla{\psi}.
\end{align}

\begin{prop}
  Let $f=f(r,z)$ be a scalar field such that $\nabla f = K(r,z) \nabla
  \psi$, where $K(r,z)$ is another scalar field. Then, $f(r,z)$ is a
  flux function, i.e. $f(r,z) = F(\psi(r,z))$.
\end{prop}

\end{document}