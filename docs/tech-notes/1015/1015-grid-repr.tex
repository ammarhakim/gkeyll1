\documentclass[11pt, reqno]{amsart}
\usepackage{hyperref}
%% AMS packages and font files
\usepackage{amsmath}
\usepackage{amsfonts}
\usepackage{amsthm}
\usepackage[dvips]{graphicx}
\usepackage[usenames,dvipsnames]{color}
\usepackage{setspace}
\usepackage{fancyhdr}
% \pagestyle{fancyplain}

\DeclareMathAlphabet{\mathpzc}{OT1}{pzc}{m}{it}

%% Set page size properly
% \oddsidemargin  0.0in
% \evensidemargin 0.0in
% \textwidth      6.5in
% \textheight     9.0in
% \leftmargin     1.0in
% \rightmargin    1.0in

%% Autoscaled figures
\newcommand{\incfig}{\centering\includegraphics}
\setkeys{Gin}{width=0.9\linewidth,keepaspectratio}

%% Commonly used macros
\newcommand{\eqr}[1]{Eq.\thinspace(#1)}
\newcommand{\pfrac}[2]{\frac{\partial #1}{\partial #2}}
\newcommand{\pfracc}[2]{\frac{\partial^2 #1}{\partial #2^2}}
\newcommand{\pfraca}[1]{\frac{\partial}{\partial #1}}
\newcommand{\pfracb}[2]{\partial #1/\partial #2}
\newcommand{\pfracbb}[2]{\partial^2 #1/\partial #2^2}
\newcommand{\spfrac}[2]{{\partial_{#1}} {#2}}
\newcommand{\mvec}[1]{\mathbf{#1}}
\newcommand{\gvec}[1]{\boldsymbol{#1}}
\newcommand{\script}[1]{\mathpzc{#1}}
\newcommand{\eep}{\mvec{e}_\phi}
\newcommand{\eer}{\mvec{e}_r}
\newcommand{\eez}{\mvec{e}_z}

\newtheorem{thm}{Theorem}
\newtheorem{lem}{Lemma}

\theoremstyle{definition}
\newtheorem{dfn}{Definition}

\title[Grid Representation]{Notes on grid representations}%
\author{Ammar H. Hakim}%
\date{}

\begin{document}
% header text
\lhead{Tech-Note 1015}
\maketitle

% General note: I do not know this stuff well and it is reflected in
% my writeup below. However, it seems to me that learning more
% topology and differential geometery will pay off even in the short
% run.

\section{Introduction}

To effectively develop production quality computational software it is
important to analyze both the mathematical structures as well as the
computational requirements of the domain under study. Selecting a
sufficiently general mathematical structure leads to logical
consistency and provides terminology that can be used in the
code. However, the mathematical analysis might not clearly indicate an
implementation. In fact, the question of representation or its
efficiency is not addressed in the abstract mathematical structure
\footnote{This does not mean that mathematics can not be used to
  analyze the computer representation or its efficiency. It simply
  means that the abstract mathematical structures and their possible
  representations in code need to be studied separately. This is
  specially true as the code depends on the features provided by the
  selected programming language.}. For example, for cell complexes we
can define the concept of incidence or adjacency precisely, but this
does not indicate \emph{how} such incidence and adjacency information
can be represented or accessed in code.

In this document I analyze the mathematical and computational
requirements for implementing a general purpose unstructured grid
library. Although restricted grid representations for specific
applications can be implemented in a relatively straightforward
manner, significant work is required to develop a flexible and
efficient library that can support the needs of diverse algorithms.

% Some algorithms need topological information while others need
% detailed geometrical information of each grid element. Various
% incidence and adjacency access might be required by different
% agorithms.

\section{An Informal Overview}

% In this section I first informally define the topological concepts
% needed to mathematical define grids.

A general mathematical framework for the analysis of grids is provided
by a type of topological space called a \emph{CW-complex}. A
CW-complex is a mathematical structure that is obtained by recursively
``connecting together'' 0-cells (vertices) to form 1-cells (edges),
1-cells to form 2-cells, 2-cells to form 3-cells etc. If the process
is stopped after a finite number of steps and there are finite number
of n-cells the resulting complex is termed \emph{finite}\footnote{A
  CW-complex allows a countably infinite number of n-cells for all or
  any $n$. Of course, such a structure is not useful for most
  computational physics applications and hence we restrict our
  attention to the case when the number of all n-cells is
  finite.}. The rather mysterious letters ``CW'' in front of the word
``complex'' in this case can then be dropped. We will use \emph{cell
  complex} or even just \emph{complex} for a finite CW-complex. Not
all possible connections are permitted in a CW-complex: all n-cells in
the complex must be topologically equivalent to the interior of a ball
in $\mathbb{R}^n$. This eliminates objects with ``holes'' or
``tunnels'' or, for example, shaped like doughnuts.

Although, as shown below, a CW-complex provides a general framework
for several data-structures commonly used in computer science it does
not have enough ``features'' for use in the solution of partial
differential equations (PDEs). We need to add additional structure to
allow representing the concept of a spatial domain which, with the
division provided by the cell complex, leads to a natural discrete
lattice on which the PDE is to be solved. This additional structure is
provided by assuming that the underlying topological space is a
\emph{manifold}, i.e., a space made up of patches that can be mapped
to a subset of Euclidean space. In many instances the manifold has a
boundary which itself is a manifold with one lower dimension.

% The difference between CW-complex and the underlying manifold is an
% important one. The manifold structure can be described independently
% of the cell complex.

% CW-complexes were introduced by J.H.C. Whitehead. See
% http://www.gap-system.org/~history/Biographies/Whitehead_Henry.html

% The theory of fiber bundles provides an abstract setting for
% considering the grid and the data stored on it as single
% mathematical object.

% An array is cell complex made up of just vertices. As we have
% assumed finiteness, vertices can be put into a sequence, i.e. put
% into a one-to-one correspondence with the integers {1,2,...,n},
% where n is the number of vertices in the complex. The same thing is
% true for a singly linked list. However, the 'next' pointer in a
% linked list adds edges, make the cell complex 1 dimensional. The
% edges now allow a traversal mechanism over the vertices using the
% 'next' pointer. The random access is still possible by constructing
% map from integers to the vertices. This map does not directly allow
% answering incidence queries, however.

% CW-complex is actually too general for grids needed in PDE solutions
% and much more additional structure needs to be imposed. In
% particular, the underlying space on which the complex lives needs to
% be a manifold or a manifold with boundary.

\end{document}

For example: consider the solution of partial differential equations
using finite-difference methods. The simplest representation appears
to be finite differences applied to data stored on a rectangular
Cartesian grid. However, a more general mathematical structure is
provided by discrete exterior calculus. This yields so called mimetic
schemes that are not only a generalization of the finite-difference
scheme but also ensures that the discretization of the differential
operators satisfies the discrete form of differential identities. Such
an analysis is quite complicated and perhaps explains why it is not
adopted in introductory exposition of the subject.

Another example: the use of Newton-Krylov-Schwarz methods as a
unifying concept for the implicit solution of PDEs. One more: the use
of Riemann solvers as a unifying framework for the solution of
hyperbolic PDEs.


