\documentclass[11pt, reqno]{amsart}
\usepackage{hyperref}
%% AMS packages and font files
\usepackage{amsmath}
\usepackage{amsfonts}
\usepackage{amsthm}
\usepackage[dvips]{graphicx}
\usepackage[usenames,dvipsnames]{color}
\usepackage{setspace}
\usepackage{fancyhdr}
% \pagestyle{fancyplain}

\DeclareMathAlphabet{\mathpzc}{OT1}{pzc}{m}{it}

%% Set page size properly
% \oddsidemargin  0.0in
% \evensidemargin 0.0in
% \textwidth      6.5in
% \textheight     9.0in
% \leftmargin     1.0in
% \rightmargin    1.0in

%% Autoscaled figures
\newcommand{\incfig}{\centering\includegraphics}
\setkeys{Gin}{width=0.9\linewidth,keepaspectratio}

%% Commonly used macros
\newcommand{\eqr}[1]{Eq.\thinspace(#1)}
\newcommand{\pfrac}[2]{\frac{\partial #1}{\partial #2}}
\newcommand{\pfracc}[2]{\frac{\partial^2 #1}{\partial #2^2}}
\newcommand{\pfraca}[1]{\frac{\partial}{\partial #1}}
\newcommand{\pfracb}[2]{\partial #1/\partial #2}
\newcommand{\pfracbb}[2]{\partial^2 #1/\partial #2^2}
\newcommand{\spfrac}[2]{{\partial_{#1}} {#2}}
\newcommand{\mvec}[1]{\mathbf{#1}}
\newcommand{\gvec}[1]{\boldsymbol{#1}}
\newcommand{\script}[1]{\mathpzc{#1}}
\newcommand{\eep}{\mvec{e}_\phi}
\newcommand{\eer}{\mvec{e}_r}
\newcommand{\eez}{\mvec{e}_z}

\newtheorem{thm}{Theorem}
\newtheorem{lem}{Lemma}

\theoremstyle{definition}
\newtheorem{dfn}{Definition}

\title[Grid Representation]{Notes on grid representations}%
\author{Ammar H. Hakim}%
\date{}

\begin{document}
% header text
\lhead{Tech-Note 1015}
\maketitle

In this document I analyze the mathematical and computational
requirements for implementing a general purpose unstructured grid
library. Although restricted grid representations can be implemented
in a straightforward manner, significant work is required in
developing a library that can support the needs of different
algorithms. 

% Some algorithms need topological information while others need
% detailed geometrical information of each grid element. Various
% incidence and adjacency access might be required by different
% agorithms.

For developing production quality computational software it is
important to analyze both the mathematical structures as well as the
computational requirements of the domain under study. Using correct
mathematical structures not only ensure logical consistency but also
provides terminology that can be used in the code. However, the
mathematical structures do not themselves indicate, at least not
clearly, the means of implementing them in computer code. In fact, the
question of representation or its efficiency is not addressed in the
abstract mathematical structures themselves\footnote{This does not
  mean that mathematics can not be used to analyze the computer
  representation or its efficiency. It simply means that the abstract
  mathematical structures and their possible representations in code
  need to be studied separately. This is specially true as the code
  depends on the abstraction provided by the selected programming
  language.}. For example, we can define the concept of incidence or
adjacency precisely using the language of topology, but this does not
indicate \emph{how} such incidence and adjacency information can be
represented or accessed in code.

\end{document}

For example: consider the solution of partial differential equations
using finite-difference methods. The simplest representation appears
to be finite differences applied to data stored on a rectangular
Cartesian grid. However, a more general mathematical structure is
provided by discrete exterior calculus. This yields so called mimetic
schemes that are not only a generalization of the finite-difference
scheme but also ensures that the discretization of the differential
operators satisfies the discrete form of differential identities. Such
an analysis is quite complicated and perhaps explains why it is not
adopted in introductory exposition of the subject.

Another example: the use of Newton-Krylov-Schwarz methods as a
unifying concept for the implicit solution of PDEs. One more: the use
of Riemann solvers as a unifying framework for the solution of
hyperbolic PDEs.


