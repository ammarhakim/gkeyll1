\documentclass[11pt, reqno]{amsart}
%% AMS packages and font files
\usepackage{amsmath}
\usepackage{amsfonts}
\usepackage[dvips]{graphicx}
\usepackage[usenames,dvipsnames]{color}
\usepackage{setspace}
\usepackage{fancyhdr}
% \pagestyle{fancyplain}

\DeclareMathAlphabet{\mathpzc}{OT1}{pzc}{m}{it}

%% Set page size properly
\oddsidemargin  0.0in
\evensidemargin 0.0in
\textwidth      6.5in
\textheight     9.0in
\headheight     1.0in
\headsep        0.0in
\leftmargin      1.0in
\rightmargin      1.0in
\topmargin      -.75in

%% Autoscaled figures
\newcommand{\incfig}{\centering\includegraphics}
\setkeys{Gin}{width=0.9\linewidth,keepaspectratio}

%% Commonly used macros
\newcommand{\eqr}[1]{Eq.\thinspace(#1)}
\newcommand{\pfrac}[2]{\frac{\partial #1}{\partial #2}}
\newcommand{\pfracc}[2]{\frac{\partial^2 #1}{\partial #2^2}}
\newcommand{\pfraca}[1]{\frac{\partial}{\partial #1}}
\newcommand{\pfracb}[2]{\partial #1/\partial #2}
\newcommand{\pfracbb}[2]{\partial^2 #1/\partial #2^2}
\newcommand{\spfrac}[2]{{\partial_{#1}} {#2}}
\newcommand{\mvec}[1]{\mathbf{#1}}
\newcommand{\gvec}[1]{\boldsymbol{#1}}
\newcommand{\script}[1]{\mathpzc{#1}}

\newtheorem{thm}{Theorem}
\newtheorem{lem}{Lemma}

\theoremstyle{definition}
\newtheorem{dfn}{Definition}


\title{Algorithms for solution of coupled atmosphere-ocean radiation
  transport equations}%
\author{Ammar H. Hakim}%
\date{}

\begin{document}
% header text
\lhead{Tech-Note 1009}
\maketitle

\section{The multi-layer, coupled atmosphere-ocean radiative transfer
  problem}

For many problems of interest, radiation in the coupled
atmosphere-ocean system can be described by the radiation transport
equation (RTE) for plane-parallel geometries. Vertical inhomogeneities
are handled by dividing the atmosphere and ocean into homogeneous
layers and enforcing continuity of radiance across layer
interfaces. The atmosphere-ocean boundary needs to be handled
carefully to account for the change in refractive index as well
surface waves on the ocean.

In each layer we need to solve the RTE
\begin{align}
  \mu\pfraca{\tau}L_k(\tau,\mu,\phi) + L_k(\tau,\mu,\phi)
  =
  \frac{\varpi_k}{4\pi}
  \int_{-1}^1 \int_0^{2\pi}
  p_k(\cos\Theta) L_k(\tau,\mu,\phi) d\mu d\phi
  \label{eqn:layer_rte}
\end{align}
for $\tau_{k-1} < \tau < \tau_k$ and
$k=1,\ldots,N_A,N_A+1,\ldots,N_A+N_O$ labeling the layers, where $N_A$
is the number of layers in the atmosphere and $N_O$ is the number of
layers in the ocean. Further, $L_k(\tau,\mu,\phi)$ is the radiance in
units of Watt m$^{-2}$ sr$^{-1}$ nm$^{-1}$, $\tau$ is the optical
depth, $\varpi_k$ is the albedo of single scattering, $\mu$ is the
cosine of the polar angle measured with the positive $Z$-axis and
$\phi$ is the azimuthal angle, $p_k(\cos\Theta) =
\sum_{l=0}^L\beta^k_lP_l(\cos\Theta)$ is the phase function, where
$\Theta$ is the scattering angle and $\beta^k_0=1$. Note that
$\tau_0=0$ is the optical depth of the top of the atmosphere and
$\tau_{N_A+N_O}=\tau_b$ is the optical depth of the bottom of the
ocean.

An incident beam illuminates the top of the atmosphere, while the
ocean bottom is assumed to be black. The boundary conditions to solve
\eqr{\ref{eqn:layer_rte}} are hence
\begin{align}
  L_0(0, \mu) &= \pi F \delta(\mu-\mu_0) \delta(\phi-\phi_0) \\
  L_b(\tau_b, -\mu) &= 0
\end{align}
for $\mu\in [0,1]$ where $\mu_0$ and $\phi_0$ are the cosine of the
polar angle and azimuthal angle of the incident beam and $\mu_0\pi F$
is the total downward solar irradiance.

\end{document}
