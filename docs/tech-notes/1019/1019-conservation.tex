\documentclass[11pt, reqno]{amsart}
\usepackage{hyperref}
%% AMS packages and font files
\usepackage{amsmath}
\usepackage{amsfonts}
\usepackage{amsthm}
\usepackage[dvips]{graphicx}
\usepackage[usenames,dvipsnames]{color}
\usepackage{setspace}
\usepackage{fancyhdr}
% \pagestyle{fancyplain}

\DeclareMathAlphabet{\mathpzc}{OT1}{pzc}{m}{it}

%% Set page size properly
% \oddsidemargin  0.0in
% \evensidemargin 0.0in
% \textwidth      6.5in
% \textheight     9.0in
% \leftmargin     1.0in
% \rightmargin    1.0in

%% Autoscaled figures
\newcommand{\incfig}{\centering\includegraphics}
\setkeys{Gin}{width=0.9\linewidth,keepaspectratio}

%% Commonly used macros
\newcommand{\eqr}[1]{Eq.\thinspace(#1)}
\newcommand{\pfrac}[2]{\frac{\partial #1}{\partial #2}}
\newcommand{\pfracc}[2]{\frac{\partial^2 #1}{\partial #2^2}}
\newcommand{\pfraca}[1]{\frac{\partial}{\partial #1}}
\newcommand{\pfracb}[2]{\partial #1/\partial #2}
\newcommand{\pfracbb}[2]{\partial^2 #1/\partial #2^2}
\newcommand{\spfrac}[2]{{\partial_{#1}} {#2}}
\newcommand{\mvec}[1]{\mathbf{#1}}
\newcommand{\gvec}[1]{\boldsymbol{#1}}
\newcommand{\script}[1]{\mathpzc{#1}}
\newcommand{\eep}{\mvec{e}_\phi}
\newcommand{\eer}{\mvec{e}_r}
\newcommand{\eez}{\mvec{e}_z}
\newcommand{\iprod}[2]{\langle{#1}\rangle_{#2}}

\newtheorem{thm}{Theorem}
\newtheorem{lem}{Lemma}
\newtheorem{example}{Example}

\theoremstyle{definition}
\newtheorem{dfn}{Definition}

\title[Conservation for Hamiltonian System]{Conservation properties of
  Discontinuous Galerkin Schemes for Hamiltonian Systems}
\author{Ammar H. Hakim}%
\date{}%

\begin{document}
% header text
\lhead{Tech-Note 1019}%
\maketitle

\section{Conservation Laws for the Continuous System}

Consider a scalar field described by a Hamiltonian evolution equation
\begin{align}
  \pfrac{f}{t} = \{H,f\} \label{eq:fevolve}
\end{align}
for the Hamiltonian $H(x,v)$ and where the canonical Poisson bracket is
\begin{align}
  \{g,h\} \equiv \pfrac{g}{x}\pfrac{h}{v} - \pfrac{g}{v}\pfrac{h}{x}.
\end{align}
Defining the phase-space velocity vector $\gvec{\alpha} = (\dot{x},
\dot{v})$, where the characteristic speeds are determined from
$\dot{x} = -\{H,x\}$ and $\dot{v} = -\{H,v\}$, allows rewriting
\eqr{\ref{eq:fevolve}} in an explicit conservation law form
\begin{align}
  \pfrac{f}{t} + \nabla\cdot\left(\gvec{\alpha}f\right) = 0.
\end{align}
Liouville theorem on phase-space incompressibility implies
$\nabla\cdot\gvec{\alpha} = 0$. Also, for any smooth function $g$,
$\{H,g\} = \nabla g\cdot\gvec{\alpha}$, hence $\{H,H\}=\nabla H \cdot
\gvec{\alpha} = 0$.

\begin{example}
  2D incompressible flow is given by the Hamiltonian $H(x,v)=\phi(x,v)$,
  where the potential is determined from $\nabla^2 \phi = -f$.
\end{example}

\begin{example}
  The Vlasov-Poisson system has the Hamiltonian
  \begin{align}
    H(x,v) = \frac{1}{2}v^2 - \frac{|e|}{m}\phi(x)
  \end{align}
  where $e$ is the charge on an electron and $m$ is electron mass. The
  potential is determined from
  \begin{align}
    \frac{\partial^2 \phi}{\partial x^2} = -\frac{\varrho_c}{\epsilon_0}
  \end{align}
  and where
  \begin{align}
    \varrho_c = |e| \left(Zn_{i0}(x) - n(x,t)\right) \label{eq:vp-poisson}
  \end{align}
  Here, $n(x,t) \equiv \int_{-\infty}^{\infty} f(x,v,t) dv$ and $Z$ is
  the ion charge number and $n_{i0}(x)$ is the fixed ion number density
  profile.
\end{example}

Multiplying \eqr{\ref{eq:fevolve}} equation by a smooth test function
$w(x,v)$ and integrating over an arbitrary volume element $K$ gives
the weak-form
\begin{align}
  \int_K w\pfrac{f}{t}d\Omega 
  + \int_{\partial K}w \gvec{\alpha}\cdot\mvec{n}f dS
  - \int_K \nabla w \cdot \gvec{\alpha} f d\Omega
 = 0.
\end{align}
The weak-form, although derived from the differential equation
\eqr{\ref{eq:fevolve}}, allows for a broader class of solutions as the
requirement on continuity of $f$ are weaker: $f$ need only be
piecewise continuous. However, the Hamiltonian must be continuous.

\subsection{Particle Conservation}

Various \emph{global} conservation laws can be obtained by
specializing the test function. In the following the domain is taken
to be periodic or extending to infinity. In either case the surface
integral term in the weak-form is assumed to vanish when the
integration is performed on the whole domain and boundary conditions
are applied.

The \emph{particle} conservation is obtained by selecting $w=1$. With
this
\begin{align}
  \pfraca{t}\int_K f d\Omega = 0
\end{align}

\subsection{Generalized Entropy Conservation}
If the solution is \emph{smooth} it can be used as a test
function. This gives
\begin{align}
  \int_K f\pfrac{f}{t}d\Omega 
  + \int_{\partial K}f \gvec{\alpha}\cdot\mvec{n}f dS
  - \int_K \nabla f \cdot \gvec{\alpha} f d\Omega
 = 0.
\end{align}
As $\nabla f \cdot \gvec{\alpha} f = \nabla\cdot (\gvec{\alpha}
f^2/2)$ the last term reduces to a surface integral, leading to the
conservation law
\begin{align}
  \pfraca{t}\int_K \frac{1}{2}f^2 d\Omega = 0.
\end{align}
Note that this results \emph{does not} hold if the solution is not
smooth.

\subsection{Energy Conservation}

Selecting the Hamiltonian as the test function gives, using the
identity $\nabla H \cdot \gvec{\alpha} = 0$,
\begin{align}
  \int_K H \pfrac{f}{t}d\Omega = 0.
\end{align}

\begin{example}
  For the incompressible Euler equations this becomes
  \begin{align}
    \int_K \phi \pfrac{f}{t}d\Omega = 0.
  \end{align}
  As ${\partial \nabla^2 \phi}/{\partial t} = -\partial f/{\partial
    t}$ we get, on integrating by parts and applying boundary
  conditions,
  \begin{align}
    \pfraca{t}\int_K \frac{1}{2} |\nabla\phi|^2  d\Omega = 0.
  \end{align}
\end{example}

\begin{example}
  For the Vlasov-Poisson system this becomes
  \begin{align}
    \int_K \left(\frac{1}{2}mv^2 - |e|\phi\right) \pfrac{f}{t}d\Omega = 0.
  \end{align}
  The second term becomes
  \begin{align}
    \int_K |e|\phi \pfrac{f}{t}d\Omega = \int |e|\phi \pfrac{n}{t} dx
  \end{align}
  Using the Poisson equation, \eqr{\ref{eq:vp-poisson}}, to eliminate
  ${\partial n}/{\partial t}$ leads to the energy conservation law
  \begin{align}
    \pfraca{t}\int \mathcal{E} +
    \frac{\epsilon_0}{2}\left(\pfrac{\phi}{x}\right)^2 dx = 0
  \end{align}
  where $\mathcal{E}(x,t) \equiv \frac{1}{2}\int_{-\infty}^{\infty} mv^2f dv$.
\end{example}

\subsection{Momentum Conservation}

For the Vlasov-Poisson system we can select the coordinate $v$ as the
test function. This leads to
\begin{align}
  \int_K v\pfrac{f}{t}d\Omega 
  + \int_{\partial K}v \gvec{\alpha}\cdot\mvec{n}f dS
  - \int_K \nabla v \cdot \gvec{\alpha} f d\Omega
 = 0.
\end{align}
As $\nabla v \cdot \gvec{\alpha} f = \dot{v}f$ the last term becomes
\begin{align}
  \int_K\dot{v} f d\Omega = \int \frac{|e|}{m} \pfrac{\phi}{x} n\thinspace dx.
\end{align}
Using the Poisson equation to eliminate $n(x,t)$, integrating by parts
and applying boundary condition leads to the momentum conservation law
\begin{align}
  \int_K v\pfrac{f}{t}d\Omega = 0.
\end{align}

\end{document}
