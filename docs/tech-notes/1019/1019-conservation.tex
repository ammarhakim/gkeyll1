\documentclass[11pt, reqno]{amsart}
\usepackage{hyperref}
%% AMS packages and font files
\usepackage{amsmath}
\usepackage{amsfonts}
\usepackage{amsthm}
\usepackage[dvips]{graphicx}
\usepackage[usenames,dvipsnames]{color}
\usepackage{setspace}
\usepackage{fancyhdr}
% \pagestyle{fancyplain}

\DeclareMathAlphabet{\mathpzc}{OT1}{pzc}{m}{it}

%% Set page size properly
% \oddsidemargin  0.0in
% \evensidemargin 0.0in
% \textwidth      6.5in
% \textheight     9.0in
% \leftmargin     1.0in
% \rightmargin    1.0in

%% Autoscaled figures
\newcommand{\incfig}{\centering\includegraphics}
\setkeys{Gin}{width=0.9\linewidth,keepaspectratio}

%% Commonly used macros
\newcommand{\eqr}[1]{Eq.\thinspace(#1)}
\newcommand{\pfrac}[2]{\frac{\partial #1}{\partial #2}}
\newcommand{\pfracc}[2]{\frac{\partial^2 #1}{\partial #2^2}}
\newcommand{\pfraca}[1]{\frac{\partial}{\partial #1}}
\newcommand{\pfracb}[2]{\partial #1/\partial #2}
\newcommand{\pfracbb}[2]{\partial^2 #1/\partial #2^2}
\newcommand{\spfrac}[2]{{\partial_{#1}} {#2}}
\newcommand{\mvec}[1]{\mathbf{#1}}
\newcommand{\gvec}[1]{\boldsymbol{#1}}
\newcommand{\script}[1]{\mathpzc{#1}}
\newcommand{\eep}{\mvec{e}_\phi}
\newcommand{\eer}{\mvec{e}_r}
\newcommand{\eez}{\mvec{e}_z}
\newcommand{\iprod}[2]{\langle{#1}\rangle_{#2}}

\newtheorem{thm}{Theorem}
\newtheorem{lem}{Lemma}
\newtheorem{example}{Example}

\theoremstyle{definition}
\newtheorem{dfn}{Definition}

\title[Conservation for Hamiltonian System]{Conservation properties of
  Discontinuous Galerkin Schemes for Hamiltonian Systems}
\author{Ammar H. Hakim}%
\date{}%

\begin{document}
% header text
\lhead{Tech-Note 1019}%
\maketitle

\section{Conservation Laws for the Continuous System}

Consider a scalar field described by a Hamiltonian evolution equation
\begin{align}
  \pfrac{f}{t} = \{H,f\} \label{eq:fevolve}
\end{align}
for the Hamiltonian $H(x,v)$ and where the canonical Poisson bracket is
\begin{align}
  \{g,h\} \equiv \pfrac{g}{x}\pfrac{h}{v} - \pfrac{g}{v}\pfrac{h}{x}.
\end{align}
Defining the phase-space velocity vector $\gvec{\alpha} = (\dot{x},
\dot{v})$, where the characteristic speeds are determined from
$\dot{x} = \{x,H\}$ and $\dot{v} = \{v,H\}$, allows rewriting
\eqr{\ref{eq:fevolve}} in an explicit conservation law form
\begin{align}
  \pfrac{f}{t} + \nabla\cdot\left(\gvec{\alpha}f\right) = 0.
\end{align}
Liouville theorem on phase-space incompressibility implies
$\nabla\cdot\gvec{\alpha} = 0$. Also, for any smooth function $g$,
$\{g,H\} = \nabla g\cdot\gvec{\alpha}$, hence $\{H,H\}=\nabla H \cdot
\gvec{\alpha} = 0$.

\begin{example}
  2D incompressible flow is given by the Hamiltonian $H(x,v)=\phi(x,v)$,
  where the potential is determined from $\nabla^2 \phi = -f$.
\end{example}

\begin{example}
  The Vlasov-Poisson system has the Hamiltonian
  \begin{align}
    H(x,v) = \frac{1}{2}v^2 - \frac{|e|}{m}\phi(x)
  \end{align}
  where $e$ is the charge on an electron and $m$ is electron mass. The
  potential is determined from
  \begin{align}
    \frac{\partial^2 \phi}{\partial x^2} = -\frac{\varrho_c}{\epsilon_0}
  \end{align}
  and where
  \begin{align}
    \varrho_c = |e| \left(Zn_{i0}(x) - n(x,t)\right) \label{eq:vp-poisson}
  \end{align}
  Here, $n(x,t) \equiv \int_{-\infty}^{\infty} f(x,v,t) dv$ and $Z$ is
  the ion charge number and $n_{i0}(x)$ is the fixed ion number density
  profile.
\end{example}

Multiplying \eqr{\ref{eq:fevolve}} equation by a smooth test function
$w(x,v)$ and integrating over an arbitrary volume element $K$ gives
the weak-form
\begin{align}
  \int_K w\pfrac{f}{t}d\Omega 
  + \int_{\partial K}w \gvec{\alpha}\cdot\mvec{n}f dS
  - \int_K \nabla w \cdot \gvec{\alpha} f d\Omega
 = 0. \label{eq:weak-form}
\end{align}
The weak-form, although derived from the differential equation
\eqr{\ref{eq:fevolve}}, allows for a broader class of solutions as the
requirement on continuity of $f$ are weaker: $f$ need only be
piecewise continuous. However, the Hamiltonian must be continuous.

\subsection{Particle Conservation}

Various \emph{global} conservation laws can be obtained by
specializing the test function. In the following the domain is taken
to be periodic or extending to infinity. In either case the surface
integral term in the weak-form is assumed to vanish when the
integration is performed on the whole domain and boundary conditions
are applied.

The \emph{particle} conservation is obtained by selecting $w=1$. With
this
\begin{align}
  \pfraca{t}\int_K f d\Omega = 0
\end{align}

\subsection{Generalized Entropy Conservation}

The solution itself can be used as a test function if it is
\emph{smooth}. This gives
\begin{align}
  \int_K f\pfrac{f}{t}d\Omega 
  + \int_{\partial K}f \gvec{\alpha}\cdot\mvec{n}f dS
  - \int_K \nabla f \cdot \gvec{\alpha} f d\Omega
 = 0.
\end{align}
As $\nabla f \cdot \gvec{\alpha} f = \nabla\cdot (\gvec{\alpha}
f^2/2)$ the last term reduces to a surface integral, leading to the
conservation law
\begin{align}
  \pfraca{t}\int_K \frac{1}{2}f^2 d\Omega = 0.
\end{align}
Note that this results \emph{does not} hold if the solution is not
smooth.

\subsection{Energy Conservation}

Selecting the Hamiltonian as the test function gives, using the
identity $\nabla H \cdot \gvec{\alpha} = 0$,
\begin{align}
  \int_K H \pfrac{f}{t}d\Omega = 0.
\end{align}

\begin{example}
  For the incompressible Euler equations this becomes
  \begin{align}
    \int_K \phi \pfrac{f}{t}d\Omega = 0.
  \end{align}
  As ${\partial \nabla^2 \phi}/{\partial t} = -\partial f/{\partial
    t}$ we get, on integrating by parts and applying boundary
  conditions,
  \begin{align}
    \pfraca{t}\int_K \frac{1}{2} |\nabla\phi|^2  d\Omega = 0.
  \end{align}
\end{example}

\begin{example}
  For the Vlasov-Poisson system this becomes
  \begin{align}
    \int_K \left(\frac{1}{2}mv^2 - |e|\phi\right) \pfrac{f}{t}d\Omega = 0.
  \end{align}
  The second term becomes
  \begin{align}
    \int_K |e|\phi \pfrac{f}{t}d\Omega = \int |e|\phi \pfrac{n}{t} dx
  \end{align}
  Using the Poisson equation, \eqr{\ref{eq:vp-poisson}}, to eliminate
  ${\partial n}/{\partial t}$ leads to the energy conservation law
  \begin{align}
    \pfraca{t}\int \mathcal{E} +
    \frac{\epsilon_0}{2}\left(\pfrac{\phi}{x}\right)^2 dx = 0
  \end{align}
  where $\mathcal{E}(x,t) \equiv \frac{1}{2}\int_{-\infty}^{\infty} mv^2f dv$.
\end{example}

\subsection{Momentum Conservation}

For the Vlasov-Poisson system we can select the coordinate $v$ as the
test function. This leads to
\begin{align}
  \int_K v\pfrac{f}{t}d\Omega 
  + \int_{\partial K}v \gvec{\alpha}\cdot\mvec{n}f dS
  - \int_K \nabla v \cdot \gvec{\alpha} f d\Omega
 = 0.
\end{align}
As $\nabla v \cdot \gvec{\alpha} = \{v,H\} = \dot{v}f$ the last term
becomes
\begin{align}
  \int_K\dot{v} f d\Omega = \int \frac{|e|}{m} \pfrac{\phi}{x} n\thinspace dx.
\end{align}
Using the Poisson equation to eliminate $n(x,t)$, integrating by parts
and applying boundary condition leads to the momentum conservation law
\begin{align}
  \frac{d}{dt}\int_K vf d\Omega = 0.
\end{align}

\section{Conservation Properties of the Discrete System}

The discontinuous Galerkin scheme discretizes the weak-form
\eqr{\ref{eq:weak-form}} by introducing a triangulation of the domain
into cells $K_j$ and a set of test functions $w$ belonging to some
finite-dimensional space, usually selected such that their restriction
on $K_j$ are polynomials. The discrete weak-form becomes
\begin{align}
  \int_{K_j} w\pfrac{f_h}{t}d\Omega + 
  \int_{\partial K_j}w^-
  \hat{F}dS 
  - \int_{K_j}
  \nabla w \cdot \gvec{\alpha}_h
  f_h d\Omega = 0. \label{eq:dis-weak-form}
\end{align}
where $\hat{F} = \hat{F}(\mvec{n}\cdot\gvec{\alpha}_h^- f_h^-,
\mvec{n}\cdot\gvec{\alpha}_h^+ f_h^+)$ is a numerical flux
function\footnote{If the Hamiltonian is continuous the normal
  component of phase-space velocity $\mvec{n}\cdot\gvec{\alpha}_h$ is
  also continuous (although the phase-space velocity itself may be
  discontinuous). In this case we can write $\hat{F} =
  \hat{F}(\mvec{n}\cdot\gvec{\alpha}_h f_h^-,
  \mvec{n}\cdot\gvec{\alpha}_h f_h^+)$, i.e., there is no need to
  distinguish between the normal of the phase-space velocity on either
  side of a cell interface. {\bf Note of 10/7/2012}: I do not know
  (yet) how to prove this for a general Hamiltonian system. This is
  evidently true of the incompressible Euler as well as the
  Vlasov-Poisson system. The proof is most like accomplished by
  transforming to local tangent-normal coordinates on the cell face
  and then showing that $\mvec{n}\cdot\gvec{\alpha}$ only depends on
  the restriction of the Hamiltonian to the cell surface.}. Here the
subscript $h$ indicates the discrete solution. The notation $w^-$
($w^+$) indicates that the function is evaluated just inside (outside)
on the location on the surface $\partial K_j$. In the following we
will assume periodic boundary conditions and/or that the solution
vanishes sufficiently rapidly if the boundaries are at infinity.

\subsection{Particle Conservation}

Selecting $w=1$ as a test function exact particle conservation follows
immediately.
\begin{align}
  \frac{\partial }{\partial t}\sum_j \int_{K_j} f_hd\Omega
  =
  0
\end{align}
where the sum is performed on all cells in the triangulation.

% \subsection{Generalized Entropy Conservation}

% Using the solution $f_h$ as a test function on the discrete weak-form
% we get
% \begin{align}
%   \frac{\partial}{\partial t} \frac{1}{2} \int_{K_j} f_h^2 d\Omega + 
%   \int_{\partial K_j}f_h^-
%   \mvec{n} \cdot \hat{F}dS 
%   - \frac{1}{2}\int_{K_j}
%   \nabla \cdot (\gvec{\alpha}_h f_h^2)
%   = 0.
% \end{align}
% where the incompressibility condition in each cell,
% $\nabla\cdot\gvec{\alpha}_h = 0$, was used.

\subsection{Energy Conservation}

We next \emph{assume} that the discrete Hamiltonian belongs to the
space of test functions. Using this in the discrete weak-form we get
\begin{align}
  \sum_j\int_{K_j} H_h\pfrac{f_h}{t}d\Omega + 
  \sum_j \int_{\partial K_j}H_h^-
  \hat{F}dS 
  = 0
\end{align}
where we have also assumed that the discrete Hamiltonian and
characteristics computed from it satisfy $\nabla H_h \cdot
\gvec{\alpha}_h = 0$. Now if the Hamiltonian is \emph{continuous} the
second term will vanish as the contribution from surfaces shared by
two cells will cancel, leaving us with
\begin{align}
  \sum_j\int_{K_j} H_h\pfrac{f_h}{t}d\Omega = 0.
\end{align}

For the Vlasov-Poisson system this can be rewritten as a discrete
integral over the spatial domain
\begin{align}
  \sum_{j}\int_{I_{j}} \pfrac{\mathcal{E}_h}{t}\thinspace dx
  -
  \sum_{j}
  |e|
  \int_{I_{j}}
  \phi_h(x)
  \pfrac{n_h}{t}
  \thinspace dx
  \label{eq:dis-energy-1}
\end{align}
where $\mathcal{E}_h$ is the discrete kinetic-energy and $n_h$ the
discrete number-density and $\phi_h$ the discrete potential. The
summation is now taken over all spatial cells $I_{j}$ where the index
$j$ now ranges over the spatial cell number.

To determine the potential a weak-form of the Poisson equation is
derived
\begin{align}
  w_x(x^-_{j+1/2})\hat{\phi}'_{hj+1/2} -  w_x(x^+_{j-1/2})\hat{\phi}'_{hj-1/2}
  -
  \int_{I_j} w_x'\phi_h'\thinspace dx
  =
  -\int_{I_j} w_x \frac{\varrho_{c}}{\epsilon_0}\thinspace dx.
  \label{eq:weak-poisson}
\end{align}
where primes denote derivatives with respect to $x$ and
$\hat{\phi}'_{hj+1/2}$ is some approximation to the (possibly)
discontinuous derivative of the potential at $x_{j+1/2}$. Also,
$w_x(x)$ belongs to projection of the test function function space to
the spatial domain.\footnote{Incidentally, this equation also shows,
  on using $w_x=1$ as the test function and summing over the domain,
  that the total charge in the domain will be exactly zero.}

Taking the time-derivative of this discrete weak-form, then using
$\phi_h$ as the test function, and summing over all cells $I_j$ leads
to
\begin{align}
  -\sum_j \int_{I_j} \phi_h'\pfrac{\phi_h'}{t}\thinspace dx
  =
  \frac{|e|}{\epsilon_0}
  \sum_j
  \int_{I_j}
  \phi_h
  \pfrac{n_h}{t}
  \thinspace dx.
\end{align}
In deriving this the continuity of $\phi_h$ was used to cancel, on
summation, the contribution from surface integration terms. Using this
in \eqr{\ref{eq:dis-energy-1}} gives the discrete energy conservation
law
\begin{align}
  \frac{\partial }{\partial t}
  \sum_{j}\int_{I_{j}} 
  \left(
  \mathcal{E}_h
  + \frac{\epsilon_0}{2}\phi_h'^2
  \right)
  \thinspace dx
  =
  0.
  \label{eq:dis-energy}
\end{align}

\section{Momentum Conservation}

To study if the scheme conserves momentum we use $mv$ as a test
function in the discrete weak-form to get
\begin{align}
  \sum_j \int_{K_j} mv\pfrac{f_h}{t}d\Omega
  -
  \sum_j
  \int_{K_j}
  m \nabla v \cdot \gvec{\alpha}_h
  f_h d\Omega = 0. \label{eq:dis-momentum-1}
\end{align}
where the continuity of $v$ was used to eliminate the surface terms on
summation. As $\nabla v \cdot \gvec{\alpha}_h = \{v,H_h\} = \dot{v}_h$
the last term becomes
\begin{align}
  -\sum_j \int_{K_j} m\dot{v}_h f_h d\Omega
  =
  \sum_j
  \int_{K_j} |e| \phi'_h f_h d\Omega
  =
  \sum_j
  \int_{I_j} |e| \phi'_h n_h \thinspace dx.
\end{align}
To simplify this we use $\phi_h'$ as a test function in the discrete
weak-form of the Poisson equation, \eqr{\ref{eq:weak-poisson}}, sum
over all cells to get
\begin{align}
  \sum_j 
  \left(
    \phi'_h(x^-_{j+1/2})\hat{\phi}'_{hj+1/2} -  \phi'_h(x^+_{j-1/2})\hat{\phi}'_{hj-1/2}
  \right)
  =
  \sum_j
  \int_{I_j}
  \phi'_h
  \frac{|e|}{\epsilon_0}
  n_h
  \thinspace dx.
\end{align}
In deriving this we have used the fact that the term $\phi_h''\phi'_h$
is a perfect derivative and so its integral summed over all cell
vanishes on application of boundary conditions. Shifting the indices
leads to
\begin{align}
  \sum_j 
  \hat{\phi}'_{hj+1/2}
  \left(
    \phi'_h(x^-_{j+1/2}) -  \phi'_h(x^+_{j+1/2})
  \right)
  =
  \sum_j
  \int_{I_j}
  \phi'_h
  \frac{|e|}{\epsilon_0}
  n_h
  \thinspace dx.
\end{align}
Using this we finally get the total momentum evolution equation
\begin{align}
  \frac{\partial }{\partial t}
  \sum_j \int_{I_j} \mathcal{M}_h \thinspace dx
  +
  \epsilon_0
  \sum_j 
  \hat{\phi}'_{hj+1/2}
  \left(
    \phi'_h(x^-_{j+1/2}) -  \phi'_h(x^+_{j+1/2})
  \right)
  =
  0
\end{align}
where $\mathcal{M}_h \equiv \int mv f_h(x,v,t) dv$ is the particle
momentum. This result shows that the total momentum \emph{is not
  conserved} as, in general, the derivative of the potential will not
be continuous at a cell interface.

\end{document}
