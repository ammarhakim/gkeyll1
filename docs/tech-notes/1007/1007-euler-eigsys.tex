\documentclass[11pt, reqno]{amsart}
%% AMS packages and font files
\usepackage{amsmath}
\usepackage{amsfonts}
\usepackage{amsthm}
\usepackage[dvips]{graphicx}
\usepackage[usenames,dvipsnames]{color}
\usepackage{setspace}
\usepackage{fancyhdr}
% \pagestyle{fancyplain}

\DeclareMathAlphabet{\mathpzc}{OT1}{pzc}{m}{it}

%% Set page size properly
% \oddsidemargin  0.0in
% \evensidemargin 0.0in
% \textwidth      6.5in
% \textheight     9.0in
% \leftmargin     1.0in
% \rightmargin    1.0in

%% Autoscaled figures
\newcommand{\incfig}{\centering\includegraphics}
\setkeys{Gin}{width=0.9\linewidth,keepaspectratio}

%% Commonly used macros
\newcommand{\eqr}[1]{Eq.\thinspace(#1)}
\newcommand{\pfrac}[2]{\frac{\partial #1}{\partial #2}}
\newcommand{\pfracc}[2]{\frac{\partial^2 #1}{\partial #2^2}}
\newcommand{\pfraca}[1]{\frac{\partial}{\partial #1}}
\newcommand{\pfracb}[2]{\partial #1/\partial #2}
\newcommand{\pfracbb}[2]{\partial^2 #1/\partial #2^2}
\newcommand{\spfrac}[2]{{\partial_{#1}} {#2}}
\newcommand{\mvec}[1]{\mathbf{#1}}
\newcommand{\gvec}[1]{\boldsymbol{#1}}
\newcommand{\script}[1]{\mathpzc{#1}}

\newtheorem{thm}{Theorem}
\newtheorem{lem}{Lemma}

\theoremstyle{definition}
\newtheorem{dfn}{Definition}

\title[Euler Eigensystem]{The eigensystem of the Euler equations}%
\author{Ammar H. Hakim}%
\date{}

\begin{document}
% header text
\lhead{Tech-Note 1007}
\maketitle

In this document I list the eigensystem of the Euler equations. The
formulas are taken from\cite{kulikovskii_2001}, Chapter 3, section
3.1. The Euler equations can be written in conservative form as
\begin{align}
  \pfraca{t}
  \left[
    \begin{matrix}
      \rho \\
      \rho u \\
      \rho v \\
      \rho w \\
      E
    \end{matrix}
  \right]
  +
  \pfraca{x}
  \left[
    \begin{matrix}
      \rho u \\
      \rho u^2 + p \\
      \rho uv \\
      \rho uw \\
      (E+p)u
    \end{matrix}
  \right]
  =
  0
\end{align}
where
\begin{align}
  E = \rho \varepsilon + \frac{1}{2}\rho q^2
\end{align}
is the total energy and $\varepsilon$ is the internal energy of the
fluid and $q^2=u^2 + v^2 + w^2$. The pressure is given by an equation
of state (EOS) $p=p(\varepsilon, \rho)$. For an ideal gas the EOS is
$p = (\gamma-1)\rho \varepsilon$.

The eigenvalues are $\{u-c, u, u, u, u+c\}$. The right
eigenvectors of the flux Jacobian are given by
\begin{align}
  R
  =
  \left[
    \begin{matrix}
      1 & 0 & 0 & 1 & 1 \\
      u-c & 0 & 0 & u & u+c \\
      v & 1 & 0 & v & v \\
      w & 0 & 1 & w & w \\
      h-uc & v & w & h-c^2/b & h+uc
    \end{matrix}
  \right]
  \label{eq:rev}
\end{align}
here
\begin{align}
  h &= (E+p)/\rho \\
  c &= \sqrt{\pfrac{p}{\rho} + \frac{p}{\rho^2}\pfrac{p}{\varepsilon}}
\end{align}
is the enthalpy and the sound speed respectively. Also,
\begin{align}
  b &= \frac{1}{\rho}\pfrac{p}{\varepsilon}.
\end{align}
Note that for ideal gas EOS we have
\begin{align}
  h &= \frac{c^2}{\gamma-1} + \frac{1}{2}q^2 \\
  c &= \sqrt{\frac{\gamma p}{\rho}}
\end{align}
and $b=\gamma-1$. Hence, in this case the term $h-c^2/b$ in
\eqr{\ref{eq:rev}} is just $q^2/2$. The left eigenvectors are
\begin{align}
  L
  =
  \frac{b}{2c^2}
  \left[
    \begin{matrix}
      \theta+uc/b & -u-c/b & -v & -w & 1 \\
      -2vc^2/b & 0 & 2c^2/b & 0 & 0 \\
      -2wc^2/b & 0 & 0 & 2c^2/b & 0 \\
      2h-2q^2 & 2u & 2v & 2w & -2 \\
      \theta-uc/b & -u+c/b & -v & -w & 1
    \end{matrix}
  \right]
  \label{eq:rev}
\end{align}
where
\begin{align}
  \theta = q^2 - \frac{E}{\rho} + \rho\frac{\pfracb{p}{\rho}}{\pfracb{p}{\varepsilon}}
\end{align}
which, for an ideal gas EOS reduces to $q^2/2$.

Now consider the problem of splitting a jump vector $\Delta \equiv
[\delta_0,\delta_1,\delta_2,\delta_3,\delta_4]^T$ into coefficients
neeeded in computing the Riemann problem. The coefficients are given
by $L\Delta$. For an ideal gas law EOS, after some algebra we can show
that an efficient way to compute these are
\begin{align}
  \alpha_3 &= \frac{\gamma-1}{c^2}
  \left[
    (h-q^2)\delta_0 + u\delta_1 + v\delta_2 + w\delta_3 -\delta_4
  \right] \\
  \alpha_1 &= -v\delta_0 + \delta_2 \\
  \alpha_2 &= -w\delta_0 + \delta_3 \\
  \alpha_4 &= \frac{1}{2c}
  \left[
    \delta_1 + (c-u)\delta_0 - c\alpha_3
  \right] \\
  \alpha_0 &= \delta_0 - \alpha_3 - \alpha_4.
\end{align}

\bibliography{../common/lucee}
\bibliographystyle{plain}

\end{document}