\documentclass[11pt, reqno]{amsart}
\usepackage{hyperref}
%% AMS packages and font files
\usepackage{amsmath}
\usepackage{amsfonts}
\usepackage{amsthm}
\usepackage[dvips]{graphicx}
\usepackage[usenames,dvipsnames]{color}
\usepackage{setspace}
\usepackage{fancyhdr}
% \pagestyle{fancyplain}

\DeclareMathAlphabet{\mathpzc}{OT1}{pzc}{m}{it}

%% Set page size properly
% \oddsidemargin  0.0in
% \evensidemargin 0.0in
% \textwidth      6.5in
% \textheight     9.0in
% \leftmargin     1.0in
% \rightmargin    1.0in

%% Autoscaled figures
\newcommand{\incfig}{\centering\includegraphics}
\setkeys{Gin}{width=0.9\linewidth,keepaspectratio}

%% Commonly used macros
\newcommand{\eqr}[1]{Eq.\thinspace(#1)}
\newcommand{\pfrac}[2]{\frac{\partial #1}{\partial #2}}
\newcommand{\pfracc}[2]{\frac{\partial^2 #1}{\partial #2^2}}
\newcommand{\pfraca}[1]{\frac{\partial}{\partial #1}}
\newcommand{\pfracb}[2]{\partial #1/\partial #2}
\newcommand{\pfracbb}[2]{\partial^2 #1/\partial #2^2}
\newcommand{\spfrac}[2]{{\partial_{#1}} {#2}}
\newcommand{\mvec}[1]{\mathbf{#1}}
\newcommand{\gvec}[1]{\boldsymbol{#1}}
\newcommand{\script}[1]{\mathpzc{#1}}
\newcommand{\eep}{\mvec{e}_\phi}
\newcommand{\eer}{\mvec{e}_r}
\newcommand{\eez}{\mvec{e}_z}

\newtheorem{thm}{Theorem}
\newtheorem{lem}{Lemma}

\theoremstyle{definition}
\newtheorem{dfn}{Definition}

\title[Multifluid Equilibrium]{Multifluid flowing equilibrium
  equations}%
\author{Ammar H. Hakim}%
\date{}

\begin{document}
% header text
\lhead{Tech-Note 1014}
\maketitle

\section{Governing Equations}

In this document I derive and list equations governing axisymmetric
multifluid equilibria. This derivation is largely based on the paper
by Steinhauer and Ishida\cite{Steinhauer:2006p526} except that I do
not normalize the equations. Also, I do not pursue the ``nearby
fluids'' concept introduced in that paper. 

The basic governing equations are the steady-state two-fluid equations
in which the electron mass is set to zero. For each fluid the
continuity and pressure equations are
\begin{align}
  \nabla\cdot(n_\alpha\mvec{u}_\alpha) &= 0 \label{eq:divNU} \\
  \mvec{u}_\alpha\cdot\nabla p_\alpha &= -\gamma p_\alpha \nabla\cdot\mvec{u}_\alpha
\end{align}
For each ion species the momentum equation is
\begin{align}
  \mvec{u}_\alpha\cdot\nabla \mvec{u}_\alpha &= -\frac{\nabla
    p_\alpha}{m_\alpha n_\alpha} 
  + \frac{q_\alpha}{m_\alpha}(\mvec{E} + \mvec{u}_\alpha\times
  \mvec{B})
\end{align}
while for the electrons, the momentum equation reduces to
\begin{align}
  0 &= -\nabla p_e - en_e(\mvec{E} + \mvec{u}_e\times\mvec{B}).
\end{align}
In these equations $m_\alpha$ and $q_\alpha$ are the species charge
and mass respectively, $n_\alpha$ is the number density,
$\mvec{u}_\alpha$ the velocity and $p_\alpha$ the pressure. For smooth
flows the pressure equation can be replaced by an advection equation
for the entropy that is obtained by setting $p_\alpha =
n_\alpha^\gamma e^{(\gamma-1)s_\alpha}$ to give
\begin{align}
  \mvec{u}_\alpha\cdot\nabla s_\alpha = 0. \label{eq:entropy}
\end{align}
The electromagnetic field is determined from the steady-state Maxwell
equations
\begin{align}
  \nabla\times \mvec{B} &= \mu_0 \mvec{J} \label{eq:curlB} \\
  \nabla\times \mvec{E} &= \mvec{0} \\
  \nabla\cdot\mvec{B} &= 0 \label{eq:divB}
\end{align}
where $\mvec{J}=\sum_\alpha q_\alpha n_\alpha\mvec{u}_\alpha$ is the
total plasma current. Finally, the condition of quasi-neutrality is
used to compute the electron density.

\section{Flux Functions}

As the divergence of the fluid momentum and magnetic field vanishes
(see \eqr{\ref{eq:divNU}} and \eqr{\ref{eq:divB}}), in axisymmetric
geometry we can write, using identities (\ref{id:divA}) and
(\ref{id:divAF}),
\begin{align}
  \mvec{u}_\alpha &= u_{\alpha \phi} \eep +
  \frac{1}{r n_\alpha}\nabla\psi_\alpha \times \eep \label{eq:uform} \\
  \mvec{B} &= B_{\phi} \eep +
  \frac{1}{r}\nabla\psi \times \eep \label{eq:bform}
\end{align}
where $u_{\alpha \phi}$ and $B_{\phi}$ are the toroidal fluid velocity
and magnetic field respectively and $\psi_\alpha(r,z)$ and $\psi(r,z)$
are scalar flux functions that determine the poloidal fluid velocity
and magnetic fields. The total plasma current can be hence expressed
as
\begin{align}
  \mvec{J} = \sum_\alpha q_\alpha n_\alpha u_{\alpha \phi}
  + \frac{1}{r}\sum_\alpha q_\alpha \nabla \psi_\alpha \times \eep
\end{align}
where the summation is over the electrons and all ion species in the
plasma. Using \eqr{\ref{eq:curlB}} and the identity (\ref{eq:curla})
in this equation for the current we get
\begin{align}
  rB_\phi &= \mu_0\sum_\alpha q_\alpha \psi_\alpha \\
  -\frac{\triangle^*\psi}{r} &=
  \mu_0 \sum_\alpha q_\alpha n_\alpha u_{\alpha \phi}.
  \label{eq:rBphi}
\end{align}
These first of these equations relates the toroidal magnetic field to
the ion flux functions while the second one relates the magnetic field
flux function to the total toroidal current. Alternately, the second
equation can be rewritten as an equation for the toroidal electron
current in terms of the ion toroidal currents and the magnetic field
flux function.

To simplify the fluid momentum equations we introduce the canonical
momentum defined by
\begin{align}
  \mvec{P}_\alpha = m_\alpha\mvec{u}_\alpha + q_\alpha \mvec{A}
\end{align}
where $\mvec{A}$ is the vector potential in terms of which $\mvec{B} =
\nabla\times \mvec{A}$. We also define the canonical vorticity as
$\mvec{\Omega}_\alpha = \nabla\times \mvec{P}_\alpha =
m_\alpha\gvec{\omega}_\alpha + q_\alpha \mvec{B}$ where
$\gvec{\omega}_\alpha=\nabla\times \mvec{u}_\alpha$ is the fluid
vorticity. As $\nabla\cdot\mvec{\Omega}_\alpha = 0$ we can write
\begin{align}
  \mvec{\Omega}_\alpha = \Omega_{\alpha \phi} \eep + \frac{1}{r}\nabla
  Y_{\alpha}\times \eep \label{eq:Omega}
\end{align}
where $Y_\alpha(r,z)$ is a canonical vorticity flux function. Using
the definition of canonical vorticity and the identity
(\ref{eq:curlaf}) to express $\gvec{\omega}_\alpha$ and using
\eqr{\ref{eq:bform}} to express the magnetic field we get, comparing
to \eqr{\ref{eq:Omega}},
\begin{align}
  \Omega_{\alpha\phi} &=
  -\frac{m_\alpha}{r}\triangle^*_{n_\alpha}\psi_\alpha
  + q_\alpha B_\phi \\
  Y_\alpha &=
  m_\alpha r u_{\alpha\phi} + q_\alpha \psi.
\end{align}

We next use the identity
\begin{align}
  \mvec{u}_\alpha\cdot\nabla\mvec{u_\alpha} 
  + \mvec{u}_\alpha\times\nabla\times\mvec{u}_\alpha
  = \nabla(\mvec{u}_\alpha^2/2)
\end{align}
and the thermodynamic relation
\begin{align}
  \frac{\nabla p_\alpha}{n_\alpha} = \nabla h_\alpha - T_\alpha\nabla s_\alpha
\end{align}
where $h_\alpha = \gamma T/(\gamma-1)$ is the fluid
enthalpy\footnote{The usual definition of enthalpy for an ideal fluid
  is $h = \gamma T/(\gamma-1) + m\mvec{u}^2/2$. In the definition
  adopted here the kinetic energy contribution is left out but taken
  into account (for the ions) in the momentum equation.} and
$T_\alpha$ is the fluid temperature (defined as $p_\alpha = n_\alpha
T_\alpha$) in the momentum equation we get
\begin{align}
  \nabla\left(
    h_\alpha
    +
    m_\alpha \mvec{u}_\alpha^2/2 + q_\alpha\phi
    \right)
    =
    T_\alpha \nabla s_\alpha + 
    \mvec{u}_\alpha\times \mvec{\Omega}_\alpha.
    \label{eq:ionMom}
\end{align}
This equation is only valid for the ions. For the electrons we need to
set $m_e=0$ and $q_e = -e$ to get the simplified electron momentum
equation as
\begin{align}
  \nabla\left(
    h_e - e\phi
    \right)
    =
    T_e \nabla s_e - e \mvec{u}_e\times \mvec{B}
    \label{eq:elcMom}
\end{align}
where we have use the fact that for electrons the canonical vorticity
is simply $\mvec{\Omega}_e = -e\mvec{B}$.

\section{Surface Functions. Components of Momentum Equations}

To derive the final set of equations governing the multifluid
equilibrium we need to look at the components of the ion and electron
momentum equations. First, we look at the toroidal component by taking
the dot product with $\eep$. As the toroidal component (in
axisymmetric geometry) of the gradient of a scalar vanishes, we get
the conditions
\begin{align}
  0 &= (\mvec{u}_\alpha\times \mvec{\Omega}_\alpha)\cdot\eep \\
  0 &= (\mvec{u}_e\times \mvec{B})\cdot\eep.
\end{align}
Using the identity (\ref{id:acrossb}) we can show that these reduce to
\begin{align}
  0 &= (\nabla \psi_\alpha \times \nabla Y_\alpha)\cdot\eep \\
  0 &= (\nabla \psi_e \times \nabla \psi)\cdot\eep.
\end{align}
This shows\footnote{Whenever we have scalar functions that are related
  by $\nabla \psi(r,z) = K(r,z) \nabla \phi(r,z)$ we can show
  $\psi(r,z) = \overline{\psi}(\phi(r,z))$ i.e., the function $\psi$
  is can be written as a \emph{surface} function of the scalar field
  $\phi$ instead of $(r,z)$. This also implies that $\nabla \psi =
  \overline{\psi}' \nabla\phi$, where the prime denotes
  differentiation with respect to $\phi$.} that the flux functions
$\psi_\alpha$ and $\psi_e$ are \emph{surface} functions, i.e.
\begin{align}
  \psi_\alpha &= \overline{\psi}_\alpha(Y_\alpha) \\
  \psi_e &= \overline{\psi}_e(\psi).
\end{align}

With these expressions we can write the poloidal component of the ion
velocity (see \eqr{\ref{eq:uform}}) as
\begin{align}
  \mvec{u}_{\alpha p} 
  = \frac{1}{r n_\alpha}\nabla\psi_\alpha \times\eep
  = \frac{\overline{\psi}_\alpha'}{r n_\alpha}\nabla Y_\alpha \times\eep
  = \frac{\overline{\psi}_\alpha'}{n_\alpha}\mvec{\Omega}_{\alpha p}
\end{align}
where $\mvec{\Omega}_{\alpha p}$ is the poloidal component of the
canonical vorticity. Similarly, the poloidal component of the electron
velocity is
\begin{align}
  \mvec{u}_{e p} 
  = \frac{1}{r n_e}\nabla\psi_e \times\eep
  = \frac{\overline{\psi}_e'}{r n_e}\nabla \psi \times\eep
  = \frac{\overline{\psi}_e'}{n_e}\mvec{B}_{p}
\end{align}
where $\mvec{B}_{p}$ is poloidal magnetic field. Hence, the ion and
electron poloidal flows are not, in general, parallel to each other.

We can show that the fluid entropy are surface functions by using
identity ({\ref{id:aAotGradF}}) in \eqr{\ref{eq:entropy}} to get
\begin{align}
   \frac{1}{r n_\alpha} \nabla s_\alpha \times \nabla \psi_\alpha = 0
\end{align}
which allows us to write $s_\alpha = \overline{s}_\alpha(Y_\alpha)$
for the ions and $s_e = \overline{s}_e(\psi)$ for the electrons.

We now take the component of the ion and electron momentum equations
along $\mvec{u}_\alpha$. The right hand side vanishes (see
\eqr{\ref{eq:entropy}}) which yields, upon using identity
({\ref{id:aAotGradF}})
\begin{align}
  h_\alpha + \frac{1}{2}m_\alpha \mvec{u}_\alpha^2 + q_\alpha\phi
  =
  \overline{H}_\alpha(Y_\alpha)
\end{align}
for the ions and
\begin{align}
  h_e - e\phi
  =
  \overline{H}_e(\psi)
\end{align}
for the electrons. These equations are a form of Bernoulli's equations
and state that the total enthalpy (including the electrostatic
potential energy) of each fluid is constant on a flux surface.

Finally, we take the component of the ion and electron momentum
equations along $\nabla Y_\alpha$ and $\nabla \psi$
respectively. These directions are perpendicular to the poloidal
components of the ion and electron velocities and hence will yield
independent set of equations. For ions taking the dot product with
$\nabla Y_\alpha$ we get
\begin{align}
  \overline{H}_\alpha' \nabla Y_\alpha \cdot \nabla Y_\alpha
  = T_\alpha \overline{s}_\alpha' \nabla Y_\alpha \cdot \nabla Y_\alpha
  +
  \nabla Y_\alpha \cdot (\mvec{u}_\alpha\times\mvec{\Omega}_\alpha).
\end{align}
The last term in this equation can be simplified using the identity
(\ref{id:acrossb}) to get, after some rearrangements, the differential
equation
\begin{align}
  m_\alpha\overline{\psi}_\alpha' 
  r^2\nabla\cdot
  \left(
    \frac{\overline{\psi}_\alpha'}{r^2 n_\alpha}\nabla Y_\alpha
  \right)
  =
  r\big(
  \overline{\psi}_\alpha' q_\alpha  B_\phi - n_\alpha u_{\alpha\phi}
  \big)
  + n_\alpha r^2 \big(
  \overline{H}_\alpha' - T_\alpha \overline{s}_\alpha'
  \big)
\end{align}
For electrons taking the dot product with $\nabla \psi$ we get
\begin{align}
  \overline{H}_e' \nabla \psi \cdot \nabla \psi
  = T_e \overline{s}_e' \nabla \psi \cdot \nabla \psi
  -
  \nabla \psi \cdot ( e \mvec{u}_e\times\mvec{B}).
\end{align}
As we did for the ions, the last term in this equation can be
simplified using the identity (\ref{id:acrossb}) to get, after some
rearrangements, the equation
\begin{align}
  r en_e u_{e\phi} = r J_{e\phi} =
  reB_\phi\overline{\psi}'
  -
  n_e r^2
  \big(
  \overline{H}_e' - T_e \overline{s}_e'  
  \big)
\end{align}
Note that the left hand side of this equation involves $J_{e\phi}$,
i.e., the toroidal component of the electron current. This can be
eliminated from \eqr{\ref{eq:rBphi}} to get a differential equation
for the magnetic field flux function $\psi(r,z)$.

\section{Summary}

% We have derived a set of equations that describes multifluid
% equilibrium configurations. The electron surface functions are
% $\overline{\psi}_e(\psi)$, $\overline{s}_e(\psi)$ and
% $\overline{H}_e(\psi)$.
{\bf There is something funcky about these equations. It seems that
  the PDEs are for surface functions which are supposed to be
  arbitrary. This is strange and needs to be thought out more
  carefully.}

\appendix

\section{Useful Identities}
\label{apdx:identities}

Let $\mvec{a}$ be an axisymmetric vector field satisfying
$\nabla\cdot\mvec{a} = 0$. Then, in cylindrical coordinates, it can be
written as
\begin{align}
  \mvec{a} = a_\phi \eep + \frac{1}{r}\nabla\psi \times \eep, \label{id:divA}
\end{align}
where $\eep$ are unit vectors and $\psi = \psi(r,z)$ is an arbitrary
function. In component form
\begin{align}
  a_r = -\frac{1}{r} \pfrac{\psi}{z}, \quad 
  a_z = \frac{1}{r} \pfrac{\psi}{r}.
\end{align}
The curl of $\mvec{a}$ is given by
\begin{align}
  \nabla\times\mvec{a} = -\frac{\triangle^*\psi}{r}\eep
  + \frac{1}{r} \nabla(ra_\phi)\times\eep, \label{eq:curla}
\end{align}
where $\triangle^*$ is the \emph{Grad-Shafranov} operator defined by
\begin{align}
  \triangle^*\psi \equiv \frac{\partial^2 \psi}{\partial z^2}
  + r \frac{\partial}{\partial r}
  \left(\frac{1}{r} \pfrac{\psi}{r}\right)
    =
  r^2\nabla\cdot\left(\frac{1}{r^2}\nabla\psi\right)
\end{align}

If $\mvec{a}$ is a axisymmetric vector field and $f(r,z)$ is a
scalar function and $\nabla \cdot (f\mvec{a}) = 0$, then
\begin{align}
  \mvec{a} = a_\phi \eep + \frac{1}{rf}\nabla\psi \times \eep. \label{id:divAF}
\end{align}
The curl of $\mvec{a}$ is given by
\begin{align}
  \nabla\times\mvec{a} = -\frac{\triangle^*_f\psi}{r}\eep
  + \frac{1}{r} \nabla(ra_\phi)\times\eep, \label{eq:curlaf}
\end{align}
where $\triangle^*_f$ is a \emph{f-weighted Grad-Shafranov} operator
defined by
\begin{align}
  \triangle^*_f\psi \equiv 
  \frac{\partial}{\partial z}\left(\frac{1}{f} \pfrac{\psi}{z}\right)
  + r \frac{\partial}{\partial r}
  \left(\frac{1}{r^2f}\pfrac{\psi}{r}\right)
  =
  r^2\nabla\cdot\left(\frac{1}{rf}\nabla\psi\right)
\end{align}
Let $\mvec{a}=a_\phi\eep + \nabla\psi_a \times \eep/f_ar$ and
$\mvec{b}=b_\phi\eep + \nabla\psi_b \times \eep/f_br$ where
$f_a=f_a(r,z)$ and $f_b=f_b(r,z)$ are scalar fields. Then
\begin{align}
  \mvec{a}\times\mvec{b} &=
  \frac{a_\phi}{rf_b}\nabla\psi_b
  -
  \frac{b_\phi}{rf_a}\nabla\psi_a
  -\frac{1}{r^2f_af_b}(\nabla\psi_a \times \eep \cdot \nabla\psi_b)\eep \\
  &= \frac{a_\phi}{rf_b}\nabla\psi_b
  -
  \frac{b_\phi}{rf_a}\nabla\psi_a
  +\frac{1}{r^2f_af_b}\nabla\psi_a \times \nabla\psi_b. \label{id:acrossb}
\end{align}
Let $\mvec{a}=a_\phi\eep + \nabla\psi \times \eep/r$ and $f=f(r,z)$ is
a scalar field. Then
\begin{align}
  \mvec{a}\cdot\nabla f = \frac{1}{r} \nabla f \times \nabla{\psi}.
  \label{id:aAotGradF}
\end{align}
%% THIS LAST IDENTITY DOES NOT LOOK CORRECT. THE LHS SHOULD BE A
%% SCALAR BUT LOOKS LIKE A VECTOR. PERHAPS IT NEEDS TO BE DOTTED WITH
%% EPP.

\bibliography{../common/lucee}
\bibliographystyle{plain}

\end{document}
