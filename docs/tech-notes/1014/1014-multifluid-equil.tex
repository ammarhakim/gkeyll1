\documentclass[11pt, reqno]{amsart}
%% AMS packages and font files
\usepackage{amsmath}
\usepackage{amsfonts}
\usepackage{amsthm}
\usepackage[dvips]{graphicx}
\usepackage[usenames,dvipsnames]{color}
\usepackage{setspace}
\usepackage{fancyhdr}
% \pagestyle{fancyplain}

\DeclareMathAlphabet{\mathpzc}{OT1}{pzc}{m}{it}

%% Set page size properly
% \oddsidemargin  0.0in
% \evensidemargin 0.0in
% \textwidth      6.5in
% \textheight     9.0in
% \leftmargin     1.0in
% \rightmargin    1.0in

%% Autoscaled figures
\newcommand{\incfig}{\centering\includegraphics}
\setkeys{Gin}{width=0.9\linewidth,keepaspectratio}

%% Commonly used macros
\newcommand{\eqr}[1]{Eq.\thinspace(#1)}
\newcommand{\pfrac}[2]{\frac{\partial #1}{\partial #2}}
\newcommand{\pfracc}[2]{\frac{\partial^2 #1}{\partial #2^2}}
\newcommand{\pfraca}[1]{\frac{\partial}{\partial #1}}
\newcommand{\pfracb}[2]{\partial #1/\partial #2}
\newcommand{\pfracbb}[2]{\partial^2 #1/\partial #2^2}
\newcommand{\spfrac}[2]{{\partial_{#1}} {#2}}
\newcommand{\mvec}[1]{\mathbf{#1}}
\newcommand{\gvec}[1]{\boldsymbol{#1}}
\newcommand{\script}[1]{\mathpzc{#1}}
\newcommand{\eep}{\mvec{e}_\phi}
\newcommand{\eer}{\mvec{e}_r}
\newcommand{\eez}{\mvec{e}_z}

\newtheorem{thm}{Theorem}
\newtheorem{lem}{Lemma}

\theoremstyle{definition}
\newtheorem{dfn}{Definition}

\title[Multifluid Equilibrium]{Multi-fluid flowing equilibrium equations}%
\author{Ammar H. Hakim}%
\date{}

\begin{document}
% header text
\lhead{Tech-Note 1014}
\maketitle

\section{Governing Equations}

In this document I derive and list equations governing axisymmetric
multi-fluid equilibria. This derivation is largely based on the paper
by Steinhauer and Ishida\cite{Steinhauer:2006p526} except that I do
not normalize the equations. Also, I do not pursue the ``nearby
fluids'' concept introduced in that paper. 

% Multi-fluid equations describing equilibrium of compact toroids like
% the field-reversed configurations (FRC) are significantly more
% complicated that static ideal-MHD equilibrium equations.
% In fact, retaining flow terms even in the ideal-MHD case makes the
% derivation and analysis of the equations more complicated.

The basic governing equations are the steady-state two-fluid equations
in which the electron mass is set to zero. For each fluid the
continuity and pressure equations are
\begin{align}
  \nabla\cdot(n_\alpha\mvec{u}_\alpha) &= 0 \label{eq:divNU} \\
  \mvec{u}_\alpha\cdot\nabla p_\alpha &= -\gamma p_\alpha \nabla\cdot\mvec{u}_\alpha
\end{align}
For each ion species the momentum equation is
\begin{align}
  \mvec{u}_\alpha\cdot\nabla \mvec{u}_\alpha &= -\frac{\nabla
    p_\alpha}{m_\alpha n_\alpha} 
  + \frac{q_\alpha}{m_\alpha}(\mvec{E} + \mvec{u}_\alpha\times
  \mvec{B})
\end{align}
while for the electrons, the momentum equation reduces to
\begin{align}
  0 &= -\nabla p_e - en_e(\mvec{E} + \mvec{u}_e\times\mvec{B}).
\end{align}
In these equations $m_\alpha$ and $q_\alpha$ are the species charge
and mass respectively, $n_\alpha$ is the number density,
$\mvec{u}_\alpha$ the velocity and $p_\alpha$ the pressure. For smooth
flows the pressure equation can be replaced by an advection equation
for the entropy that is obtained by setting $p_\alpha =
n_\alpha^\gamma e^{(\gamma-1)s_\alpha}$ to give
\begin{align}
  \mvec{u}_\alpha\cdot\nabla s_\alpha = 0.
\end{align}
The electromagnetic field is determined from the steady-state Maxwell
equations
\begin{align}
  \nabla\times \mvec{B} &= \mu_0 \mvec{J} \label{eq:curlB} \\
  \nabla\times \mvec{E} &= \mvec{0} \\
  \nabla\cdot\mvec{B} &= 0 \label{eq:divB}
\end{align}
where $\mvec{J}=\sum_\alpha q_\alpha n_\alpha\mvec{u}_\alpha$ is the
total plasma current.

\section{Axisymmetric Multi-Fluid Equilibria}

As the divergence of the fluid momentum and magnetic field vanishes
(see \eqr{\ref{eq:divNU}} and \eqr{\ref{eq:divB}}), in axisymmetric
geometry we can write, using identities (\ref{id:divA}) and
(\ref{id:divAF}),
\begin{align}
  \mvec{u}_\alpha &= u_{\alpha \phi} \eep +
  \frac{1}{r n_\alpha}\nabla\psi_\alpha \times \eep \label{eq:uform} \\
  \mvec{B} &= B_{\phi} \eep +
  \frac{1}{r}\nabla\psi \times \eep \label{eq:bform}
\end{align}
where $u_{\alpha \phi}$ and $B_{\phi}$ are the toroidal fluid velocity
and magnetic field respectively and $\psi_\alpha(r,z)$ and $\psi(r,z)$
are scalar flux functions that determine the poloidal fluid velocity
and magnetic fields. The total plasma current can be hence expressed
as
\begin{align}
  \mvec{J} = \sum_\alpha q_\alpha n_\alpha u_{\alpha \phi}
  + \frac{1}{r}\sum_\alpha q_\alpha \nabla \psi_\alpha \times \eep
\end{align}
where the summation is over the electrons all ion species in the
plasma. Using \eqr{\ref{eq:curlB}} and the identity (\ref{eq:curla})
in this equation for the current we get
\begin{align}
  rB_\phi &= \mu_0\sum_\alpha q_\alpha \psi_\alpha \\
  -\frac{\triangle^*\psi}{r} &=
  \mu_0 \sum_\alpha q_\alpha n_\alpha u_{\alpha \phi}.
\end{align}
These first of these equations relates the toroidal magnetic field to
the ion flux functions while the second one relates the magnetic field
flux function to the total toroidal current. Alternately, the second
equation can be thought of as an equation for the toroidal electron
current in terms of the ion toroidal currents and the magnetic field
flux function.

To simplify the fluid momentum equations we introduce the canonical
momentum defined by
\begin{align}
  \mvec{P}_\alpha = m_\alpha\mvec{u}_\alpha + q_\alpha \mvec{A}
\end{align}
where $\mvec{A}$ is the vector potential in terms of which $\mvec{B} =
\nabla\times \mvec{A}$. We also define the canonical vorticity as
$\mvec{\Omega}_\alpha = \nabla\times \mvec{P}_\alpha =
m_\alpha\gvec{\omega}_\alpha + q_\alpha \mvec{B}$ where
$\gvec{\omega}_\alpha=\nabla\times \mvec{u}_\alpha$ is the fluid
vorticity. As $\nabla\cdot\mvec{\Omega}_\alpha = 0$ we can write
\begin{align}
  \mvec{\Omega}_\alpha = \Omega_{\alpha \phi} \eep + \frac{1}{r}\nabla
  Y_{\alpha}\times \eep
\end{align}


\appendix

\section{Useful Identities}
\label{apdx:identities}

Let $\mvec{a}$ be an axisymmetric vector field satisfying
$\nabla\cdot\mvec{a} = 0$. Then, in cylindrical coordinates, it can be
written as
\begin{align}
  \mvec{a} = a_\phi \eep + \frac{1}{r}\nabla\psi \times \eep, \label{id:divA}
\end{align}
where $\eep$ are unit vectors and $\psi = \psi(r,z)$ is an arbitrary
function. In component form
\begin{align}
  a_r = -\frac{1}{r} \pfrac{\psi}{z}, \quad 
  a_z = \frac{1}{r} \pfrac{\psi}{r}.
\end{align}
The curl of $\mvec{a}$ is given by
\begin{align}
  \nabla\times\mvec{a} = -\frac{\triangle^*\psi}{r}\eep
  + \frac{1}{r} \nabla(ra_\phi)\times\eep, \label{eq:curla}
\end{align}
where $\triangle^*$ is the \emph{Grad-Shafranov} operator defined by
\begin{align}
  \triangle^*\psi \equiv \frac{\partial^2 \psi}{\partial z^2}
  + r \frac{\partial}{\partial r}\left(\frac{1}{r} \pfrac{\psi}{r}\right).
\end{align}

If $\mvec{a}$ is a axisymmetric vector field and $f(r,z)$ is a
scalar function and $\nabla \cdot (f\mvec{a}) = 0$, then
\begin{align}
  \mvec{a} = a_\phi \eep + \frac{1}{rf}\nabla\psi \times \eep. \label{id:divAF}
\end{align}
The curl of $\mvec{a}$ is given by
\begin{align}
  \nabla\times\mvec{a} = -\frac{\triangle^*_f\psi}{r}\eep
  + \frac{1}{r} \nabla(ra_\phi)\times\eep, \label{eq:curlaf}
\end{align}
where $\triangle^*_f$ is a \emph{f-weighted Grad-Shafranov} operator
defined by
\begin{align}
  \triangle^*_f\psi \equiv 
  \frac{\partial}{\partial z}\left(\frac{1}{f} \pfrac{\psi}{z}\right)
  + r \frac{\partial}{\partial r}\left(\frac{1}{rf} \pfrac{\psi}{r}\right).
\end{align}
Let $\mvec{a}=a_\phi\eep + \nabla\psi_a \times \eep/f_ar$ and
$\mvec{b}=b_\phi\eep + \nabla\psi_b \times \eep/f_br$ where
$f_a=f_a(r,z)$ and $f_b=f_b(r,z)$ are scalar fields. Then
\begin{align}
  \mvec{a}\times\mvec{b} &=
  \frac{a_\phi}{rf_b}\nabla\psi_b
  -
  \frac{b_\phi}{rf_a}\nabla\psi_a
  -\frac{1}{r^2f_af_b}(\nabla\psi_a \times \eep \cdot \nabla\psi_b)\eep \\
  &= \frac{a_\phi}{rf_b}\nabla\psi_b
  -
  \frac{b_\phi}{rf_a}\nabla\psi_a
  +\frac{1}{r^2f_af_b}\nabla\psi_a \times \nabla\psi_b.
\end{align}
Let $\mvec{a}=a_\phi\eep + \nabla\psi \times \eep/r$ and $f=f(r,z)$ is
a scalar field. Then
\begin{align}
  \mvec{a}\cdot\nabla f = \frac{1}{r} \nabla f \times \nabla{\psi}.
\end{align}
Let $\nabla \psi(r,z) = K(r,z) \nabla \phi(r,z)$ where $\psi$, $\phi$
and $K$ are scalar functions. Then we have
\begin{align}
  \psi(r,z) = \overline{\psi}(\phi)
\end{align}
i.e., the function $\psi$ is now a \emph{surface} function of a single
scalar field $\phi$ instead of $(r,z)$. This also implies that $\nabla
\psi = \overline{\psi}' \nabla\phi$, where the prime denotes
differentiation with respect to $\phi$.

\bibliography{../common/lucee}
\bibliographystyle{plain}

\end{document}
